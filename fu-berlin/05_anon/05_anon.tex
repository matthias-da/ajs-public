\PassOptionsToPackage{unicode=true}{hyperref} % options for packages loaded elsewhere
\PassOptionsToPackage{hyphens}{url}
%
\documentclass[ignorenonframetext,]{beamer}
\usepackage{pgfpages}
\setbeamertemplate{caption}[numbered]
\setbeamertemplate{caption label separator}{: }
\setbeamercolor{caption name}{fg=normal text.fg}
\beamertemplatenavigationsymbolsempty
% Prevent slide breaks in the middle of a paragraph:
\widowpenalties 1 10000
\raggedbottom
\setbeamertemplate{part page}{
\centering
\begin{beamercolorbox}[sep=16pt,center]{part title}
  \usebeamerfont{part title}\insertpart\par
\end{beamercolorbox}
}
\setbeamertemplate{section page}{
\centering
\begin{beamercolorbox}[sep=12pt,center]{part title}
  \usebeamerfont{section title}\insertsection\par
\end{beamercolorbox}
}
\setbeamertemplate{subsection page}{
\centering
\begin{beamercolorbox}[sep=8pt,center]{part title}
  \usebeamerfont{subsection title}\insertsubsection\par
\end{beamercolorbox}
}
\AtBeginPart{
  \frame{\partpage}
}
\AtBeginSection{
  \ifbibliography
  \else
    \frame{\sectionpage}
  \fi
}
\AtBeginSubsection{
  \frame{\subsectionpage}
}
\usepackage{lmodern}
\usepackage{amssymb,amsmath}
\usepackage{ifxetex,ifluatex}
\usepackage{fixltx2e} % provides \textsubscript
\ifnum 0\ifxetex 1\fi\ifluatex 1\fi=0 % if pdftex
  \usepackage[T1]{fontenc}
  \usepackage[utf8]{inputenc}
  \usepackage{textcomp} % provides euro and other symbols
\else % if luatex or xelatex
  \usepackage{unicode-math}
  \defaultfontfeatures{Ligatures=TeX,Scale=MatchLowercase}
\fi
% use upquote if available, for straight quotes in verbatim environments
\IfFileExists{upquote.sty}{\usepackage{upquote}}{}
% use microtype if available
\IfFileExists{microtype.sty}{%
\usepackage[]{microtype}
\UseMicrotypeSet[protrusion]{basicmath} % disable protrusion for tt fonts
}{}
\IfFileExists{parskip.sty}{%
\usepackage{parskip}
}{% else
\setlength{\parindent}{0pt}
\setlength{\parskip}{6pt plus 2pt minus 1pt}
}
\usepackage{hyperref}
\hypersetup{
            pdftitle={Advanced Survey Statistics: Disclosure Control},
            pdfauthor={Matthias Templ},
            pdfborder={0 0 0},
            breaklinks=true}
\urlstyle{same}  % don't use monospace font for urls
\newif\ifbibliography
\usepackage{color}
\usepackage{fancyvrb}
\newcommand{\VerbBar}{|}
\newcommand{\VERB}{\Verb[commandchars=\\\{\}]}
\DefineVerbatimEnvironment{Highlighting}{Verbatim}{commandchars=\\\{\}}
% Add ',fontsize=\small' for more characters per line
\usepackage{framed}
\definecolor{shadecolor}{RGB}{248,248,248}
\newenvironment{Shaded}{\begin{snugshade}}{\end{snugshade}}
\newcommand{\AlertTok}[1]{\textcolor[rgb]{0.94,0.16,0.16}{#1}}
\newcommand{\AnnotationTok}[1]{\textcolor[rgb]{0.56,0.35,0.01}{\textbf{\textit{#1}}}}
\newcommand{\AttributeTok}[1]{\textcolor[rgb]{0.77,0.63,0.00}{#1}}
\newcommand{\BaseNTok}[1]{\textcolor[rgb]{0.00,0.00,0.81}{#1}}
\newcommand{\BuiltInTok}[1]{#1}
\newcommand{\CharTok}[1]{\textcolor[rgb]{0.31,0.60,0.02}{#1}}
\newcommand{\CommentTok}[1]{\textcolor[rgb]{0.56,0.35,0.01}{\textit{#1}}}
\newcommand{\CommentVarTok}[1]{\textcolor[rgb]{0.56,0.35,0.01}{\textbf{\textit{#1}}}}
\newcommand{\ConstantTok}[1]{\textcolor[rgb]{0.00,0.00,0.00}{#1}}
\newcommand{\ControlFlowTok}[1]{\textcolor[rgb]{0.13,0.29,0.53}{\textbf{#1}}}
\newcommand{\DataTypeTok}[1]{\textcolor[rgb]{0.13,0.29,0.53}{#1}}
\newcommand{\DecValTok}[1]{\textcolor[rgb]{0.00,0.00,0.81}{#1}}
\newcommand{\DocumentationTok}[1]{\textcolor[rgb]{0.56,0.35,0.01}{\textbf{\textit{#1}}}}
\newcommand{\ErrorTok}[1]{\textcolor[rgb]{0.64,0.00,0.00}{\textbf{#1}}}
\newcommand{\ExtensionTok}[1]{#1}
\newcommand{\FloatTok}[1]{\textcolor[rgb]{0.00,0.00,0.81}{#1}}
\newcommand{\FunctionTok}[1]{\textcolor[rgb]{0.00,0.00,0.00}{#1}}
\newcommand{\ImportTok}[1]{#1}
\newcommand{\InformationTok}[1]{\textcolor[rgb]{0.56,0.35,0.01}{\textbf{\textit{#1}}}}
\newcommand{\KeywordTok}[1]{\textcolor[rgb]{0.13,0.29,0.53}{\textbf{#1}}}
\newcommand{\NormalTok}[1]{#1}
\newcommand{\OperatorTok}[1]{\textcolor[rgb]{0.81,0.36,0.00}{\textbf{#1}}}
\newcommand{\OtherTok}[1]{\textcolor[rgb]{0.56,0.35,0.01}{#1}}
\newcommand{\PreprocessorTok}[1]{\textcolor[rgb]{0.56,0.35,0.01}{\textit{#1}}}
\newcommand{\RegionMarkerTok}[1]{#1}
\newcommand{\SpecialCharTok}[1]{\textcolor[rgb]{0.00,0.00,0.00}{#1}}
\newcommand{\SpecialStringTok}[1]{\textcolor[rgb]{0.31,0.60,0.02}{#1}}
\newcommand{\StringTok}[1]{\textcolor[rgb]{0.31,0.60,0.02}{#1}}
\newcommand{\VariableTok}[1]{\textcolor[rgb]{0.00,0.00,0.00}{#1}}
\newcommand{\VerbatimStringTok}[1]{\textcolor[rgb]{0.31,0.60,0.02}{#1}}
\newcommand{\WarningTok}[1]{\textcolor[rgb]{0.56,0.35,0.01}{\textbf{\textit{#1}}}}
\setlength{\emergencystretch}{3em}  % prevent overfull lines
\providecommand{\tightlist}{%
  \setlength{\itemsep}{0pt}\setlength{\parskip}{0pt}}
\setcounter{secnumdepth}{0}

% set default figure placement to htbp
\makeatletter
\def\fps@figure{htbp}
\makeatother

\newcommand{\N}{\mathbb{N}}
\newcommand{\R}{\mathbb{R}}
\newcommand{\E}{\mathsf{E}}
\newcommand{\Ex}{\mathcal{E}}
\newcommand{\Y}{\mathcal{Y}}
\newcommand{\D}{\mathcal{D}}
\newcommand{\Var}{\mathsf{Var}}
\newcommand{\OS}{\mathsf{ord}}
\newcommand{\T}{\mathsf{T}}
\newcommand{\Cov}{\mathsf{Cov}}
\newcommand{\COV}{\boldsymbol{\Sigma}}
\newcommand{\vX}{\mathbf{X}}
\newcommand{\vx}{\mathbf{x}}
\newcommand{\vz}{\mathbf{z}}
\newcommand{\vB}{\mathbf{B}}
\newcommand{\vS}{\mathbf{S}}
\newcommand{\vU}{\mathbf{U}}
\newcommand{\vLm}{\boldsymbol{\Lambda}}
\newcommand{\vmu}{\boldsymbol{\mu}}
\newcommand{\vbeta}{\boldsymbol{\beta}}
\newcommand{\I}{\mathbf{I}}
\newcommand{\Det}{\mathsf{det}}
\newcommand{\tr}{\mathsf{tr}}

\usepackage{booktabs}

\usepackage[babel,german=quotes]{csquotes}
\usepackage[ngerman]{babel}
\usepackage{tikz}

\definecolor{zhawblue}{rgb}{0, 0.391, 0.648}
\setbeamercolor{structure}{fg=zhawblue}
\usecolortheme[named=zhawblue]{structure}

\usepackage{beamerthemesplit}
\expandafter\def\expandafter\insertshorttitle\expandafter{%
  \insertshorttitle\hfill%
  \insertframenumber\,/\,\inserttotalframenumber}
														

\beamertemplatenavigationsymbolsempty

\titlegraphic{\hfill \includegraphics[width=3.5cm]{../de-zhaw-idp-rgb.jpg}}
\institute{Institut für Datenanalyse und Prozessdesign\\School of Engineering\\Zürcher Hochschule für Angewandte Wissenschaften}

\title{Advanced Survey Statistics: Disclosure Control}
\providecommand{\subtitle}[1]{}
\subtitle{Part 5: Anonymisation Methods}
\author{Matthias Templ}
\date{FU-Berlin, 2019}

\begin{document}
\frame{\titlepage}

\begin{frame}{Traditional anonymisation methods}
\protect\hypertarget{traditional-anonymisation-methods}{}

We will discuss these standard methods

Methods that recode data or suppress values

\begin{itemize}
\tightlist
\item
  global recoding
\item
  local suppression
\item
  microaggregation
\end{itemize}

Methods that include a probability mechanism

\begin{itemize}
\tightlist
\item
  PRAM
\item
  adding noise
\end{itemize}

\end{frame}

\begin{frame}{Recoding}
\protect\hypertarget{recoding}{}

Recoding categorical key variables:

\begin{itemize}
\tightlist
\item
  achieving anonymity by mapping the values of the categorical key
  variables to generalized or altered categories.
\item
  Example: combine multiple levels of schooling (e.g., secondary,
  tertiary, postgraduate) into one level (e.g., secondary and above).
\end{itemize}

Recoding continuous variables

\begin{itemize}
\tightlist
\item
  means to discretize the variable
\item
  Example: income to income classes
\end{itemize}

\end{frame}

\begin{frame}[fragile]{Recoding}
\protect\hypertarget{recoding-1}{}

Test data from the Philippines (household income data):

\begin{Shaded}
\begin{Highlighting}[]
\KeywordTok{require}\NormalTok{(}\StringTok{"sdcMicro"}\NormalTok{)}
\KeywordTok{data}\NormalTok{(testdata, }\DataTypeTok{package=}\StringTok{"sdcMicro"}\NormalTok{)}
\NormalTok{testdata}\OperatorTok{$}\NormalTok{urbrur <-}\StringTok{ }\KeywordTok{factor}\NormalTok{(testdata}\OperatorTok{$}\NormalTok{urbrur)}
\NormalTok{testdata}\OperatorTok{$}\NormalTok{water <-}\StringTok{ }\KeywordTok{factor}\NormalTok{(testdata}\OperatorTok{$}\NormalTok{water)}
\NormalTok{testdata}\OperatorTok{$}\NormalTok{relat <-}\StringTok{ }\KeywordTok{factor}\NormalTok{(testdata}\OperatorTok{$}\NormalTok{relat)}
\NormalTok{testdata}\OperatorTok{$}\NormalTok{walls <-}\StringTok{ }\KeywordTok{factor}\NormalTok{(testdata}\OperatorTok{$}\NormalTok{walls)}
\NormalTok{sdc <-}\StringTok{ }\KeywordTok{createSdcObj}\NormalTok{(testdata,}
          \DataTypeTok{keyVars=}\KeywordTok{c}\NormalTok{(}\StringTok{'urbrur'}\NormalTok{,}\StringTok{'water'}\NormalTok{,}\StringTok{'sex'}\NormalTok{,}\StringTok{'age'}\NormalTok{,}\StringTok{'relat'}\NormalTok{), }
          \DataTypeTok{numVars=}\KeywordTok{c}\NormalTok{(}\StringTok{'expend'}\NormalTok{,}\StringTok{'income'}\NormalTok{,}\StringTok{'savings'}\NormalTok{),}
          \DataTypeTok{pramVars=}\KeywordTok{c}\NormalTok{(}\StringTok{"walls"}\NormalTok{), }
          \DataTypeTok{w=}\StringTok{'sampling_weight'}\NormalTok{, }
          \DataTypeTok{hhId=}\StringTok{'ori_hid'}\NormalTok{, }\DataTypeTok{alpha =} \FloatTok{0.7}\NormalTok{)}
\KeywordTok{summary}\NormalTok{(testdata}\OperatorTok{$}\NormalTok{age)}
\end{Highlighting}
\end{Shaded}

\begin{verbatim}
##    Min. 1st Qu.  Median    Mean 3rd Qu.    Max. 
##    0.00    9.00   19.00   24.11   36.00   95.00
\end{verbatim}

\end{frame}

\begin{frame}[fragile]{Recoding}
\protect\hypertarget{recoding-2}{}

\begin{Shaded}
\begin{Highlighting}[]
\NormalTok{labs <-}\StringTok{ }\KeywordTok{c}\NormalTok{(}\StringTok{"1-9"}\NormalTok{,}\StringTok{"10-19"}\NormalTok{,}\StringTok{"20-29"}\NormalTok{,}\StringTok{"30-39"}\NormalTok{,}
          \StringTok{"40-49"}\NormalTok{,}\StringTok{"50-59"}\NormalTok{,}\StringTok{"60-69"}\NormalTok{,}\StringTok{"70-79"}\NormalTok{,}\StringTok{"80-130"}\NormalTok{)}
\NormalTok{sdc <-}\StringTok{ }\KeywordTok{globalRecode}\NormalTok{(sdc, }\DataTypeTok{column=}\StringTok{"age"}\NormalTok{,}
                    \DataTypeTok{breaks=}\KeywordTok{c}\NormalTok{(}\DecValTok{0}\NormalTok{,}\DecValTok{9}\NormalTok{,}\DecValTok{19}\NormalTok{,}\DecValTok{29}\NormalTok{,}\DecValTok{39}\NormalTok{,}\DecValTok{49}\NormalTok{,}\DecValTok{59}\NormalTok{,}\DecValTok{69}\NormalTok{,}\DecValTok{79}\NormalTok{,}\DecValTok{130}\NormalTok{), }
                    \DataTypeTok{labels=}\NormalTok{labs)}
\KeywordTok{print}\NormalTok{(sdc)}
\end{Highlighting}
\end{Shaded}

\begin{verbatim}
## Infos on 2/3-Anonymity:
## 
## Number of observations violating
##   - 2-anonymity: 113 (2.467%) | in original data: 653 (14.258%)
##   - 3-anonymity: 188 (4.105%) | in original data: 1087 (23.734%)
##   - 5-anonymity: 362 (7.904%) | in original data: 1781 (38.886%)
## 
## ----------------------------------------------------------------------
\end{verbatim}

To combine specific categories use \texttt{groupAndRename}.

\end{frame}

\begin{frame}[fragile]{Recoding}
\protect\hypertarget{recoding-3}{}

\begin{Shaded}
\begin{Highlighting}[]
\NormalTok{sdc <-}\StringTok{ }\KeywordTok{groupAndRename}\NormalTok{(sdc, }\DataTypeTok{var=}\StringTok{"urbrur"}\NormalTok{, }
         \DataTypeTok{before=}\KeywordTok{c}\NormalTok{(}\StringTok{"1"}\NormalTok{,}\StringTok{"2"}\NormalTok{), }\DataTypeTok{after=}\StringTok{"1"}\NormalTok{)}
\NormalTok{sdc <-}\StringTok{ }\KeywordTok{groupAndRename}\NormalTok{(sdc, }\DataTypeTok{var=}\StringTok{"water"}\NormalTok{, }
         \DataTypeTok{before=}\KeywordTok{levels}\NormalTok{(testdata}\OperatorTok{$}\NormalTok{water), }
         \DataTypeTok{after=}\KeywordTok{c}\NormalTok{(}\StringTok{"1"}\NormalTok{,}\StringTok{"2"}\NormalTok{,}\StringTok{"3"}\NormalTok{,}\StringTok{"4"}\NormalTok{,}\StringTok{"5"}\NormalTok{,}\StringTok{"6-9"}\NormalTok{,}\StringTok{"6-9"}\NormalTok{,}\StringTok{"6-9"}\NormalTok{))}
\NormalTok{sdc <-}\StringTok{ }\KeywordTok{groupAndRename}\NormalTok{(sdc, }\DataTypeTok{var=}\StringTok{"relat"}\NormalTok{, }
         \DataTypeTok{before=}\KeywordTok{levels}\NormalTok{(testdata}\OperatorTok{$}\NormalTok{relat), }
         \DataTypeTok{after=}\KeywordTok{c}\NormalTok{(}\StringTok{"1"}\NormalTok{,}\StringTok{"2"}\NormalTok{,}\StringTok{"3"}\NormalTok{,}\StringTok{"4"}\NormalTok{,}\StringTok{"5"}\NormalTok{,}\StringTok{"6"}\NormalTok{,}\StringTok{"7"}\NormalTok{,}\StringTok{"8-9"}\NormalTok{,}\StringTok{"8-9"}\NormalTok{))}
\KeywordTok{print}\NormalTok{(sdc, }\StringTok{"kAnon"}\NormalTok{)}
\end{Highlighting}
\end{Shaded}

\begin{verbatim}
## Infos on 2/3-Anonymity:
## 
## Number of observations violating
##   - 2-anonymity: 68 (1.485%) | in original data: 653 (14.258%)
##   - 3-anonymity: 113 (2.467%) | in original data: 1087 (23.734%)
##   - 5-anonymity: 210 (4.585%) | in original data: 1781 (38.886%)
## 
## ----------------------------------------------------------------------
\end{verbatim}

\end{frame}

\begin{frame}[fragile]{Top- and Bottom Coding}
\protect\hypertarget{top--and-bottom-coding}{}

Top (Bottom) Coding

\begin{itemize}
\tightlist
\item
  continuous variables are cut off by a given upper (lower) threshold
\item
  values above (below) are replaced with, e.g., the mean of values above
  (below) the threshold
\end{itemize}

\begin{Shaded}
\begin{Highlighting}[]
\NormalTok{sdc <-}\StringTok{ }\KeywordTok{topBotCoding}\NormalTok{(sdc, }\DataTypeTok{value=}\DecValTok{500000}\NormalTok{, }\DataTypeTok{replacement=}\DecValTok{500000}\NormalTok{, }
                    \DataTypeTok{column=}\StringTok{"income"}\NormalTok{)}
\end{Highlighting}
\end{Shaded}

\begin{itemize}
\tightlist
\item
  advantage: easy to explain and thus often applied
\item
  disadvantage: the highest (lowest) values are then identical
\item
  extension to the multivariate case: see presentation on Wednesday
\end{itemize}

\end{frame}

\begin{frame}{Local suppression}
\protect\hypertarget{local-suppression}{}

\begin{itemize}
\tightlist
\item
  Typically used after recoding to minimize residual risk.
\item
  Heuristic optimization methods to find specific patterns in
  categorical key variables. Replace this pattern with missing values.
\item
  One aim: to \textbf{suppress a minimum amount of values} and in the
  same time \textbf{guarantee $k$-anonymity}.
\item
  Additional complexity: frequency counts with missing Values.
\item
  Weight the variables according to their importance (in some variables
  you may want to end with less suppresions than in some others)
\end{itemize}

\end{frame}

\begin{frame}{Local suppression - approaches}
\protect\hypertarget{local-suppression---approaches}{}

\begin{description}
\item[\textbf{Mondrian}:] combine categories to achieve $k$-anonymity by a recoding strategy based on counts of categories. Too over-simplistic approach, not very promising results because the algorithm combines categories without asking their meaning. 
\item[\textbf{all-$M$ approach}:] whenever $k$-anonymity cannot be provided because of having too many key variables, then $k$-anonymity is provided in all subsets of size $M$ of the key variables. More precisely, the algorithm will provide $k$-anonymity for each combination of $M$ key variables. 
\item[\textbf{$k$-anonymity approach}:] ensures $k$-anonymity for the combination of all key variables. If the number of key variables is too high: all-$M$ approach or PRAM (next method to be explained) for specific key variables. 
\end{description}

\end{frame}

\begin{frame}[fragile]{Local suppression}
\protect\hypertarget{local-suppression-1}{}

\begin{Shaded}
\begin{Highlighting}[]
\CommentTok{# all M approach}
\NormalTok{combs <-}\StringTok{ }\DecValTok{5}\OperatorTok{:}\DecValTok{3}
\NormalTok{k <-}\StringTok{ }\KeywordTok{c}\NormalTok{(}\DecValTok{10}\NormalTok{,}\DecValTok{20}\NormalTok{,}\DecValTok{30}\NormalTok{)}
\NormalTok{sdc <-}\StringTok{ }\KeywordTok{kAnon}\NormalTok{(sdc, }\DataTypeTok{k =}\NormalTok{ k, }\DataTypeTok{combs =}\NormalTok{ combs, }\DataTypeTok{importance =} \KeywordTok{c}\NormalTok{(}\DecValTok{3}\NormalTok{,}\DecValTok{4}\NormalTok{,}\DecValTok{2}\NormalTok{,}\DecValTok{1}\NormalTok{,}\DecValTok{5}\NormalTok{)) }
\CommentTok{# print(sdc, "kAnon")}
\KeywordTok{print}\NormalTok{(sdc, }\StringTok{"ls"}\NormalTok{)}
\end{Highlighting}
\end{Shaded}

\begin{verbatim}
## Local suppression:
##  KeyVar | Suppressions (#) | Suppressions (%)
##  urbrur |                0 |            0.000
##   water |              281 |            6.135
##     sex |               13 |            0.284
##     age |                0 |            0.000
##   relat |              429 |            9.367
## ----------------------------------------------------------------------
\end{verbatim}

\begin{Shaded}
\begin{Highlighting}[]
\NormalTok{sdc <-}\StringTok{ }\KeywordTok{undolast}\NormalTok{(sdc)}
\end{Highlighting}
\end{Shaded}

\end{frame}

\begin{frame}[fragile]{Local suppression}
\protect\hypertarget{local-suppression-2}{}

\begin{Shaded}
\begin{Highlighting}[]
\CommentTok{# k-anonymity for all key variables}
\NormalTok{sdc <-}\StringTok{ }\KeywordTok{kAnon}\NormalTok{(sdc, }\DataTypeTok{k =} \DecValTok{3}\NormalTok{, }\DataTypeTok{importance =} \KeywordTok{c}\NormalTok{(}\DecValTok{3}\NormalTok{,}\DecValTok{4}\NormalTok{,}\DecValTok{2}\NormalTok{,}\DecValTok{1}\NormalTok{,}\DecValTok{5}\NormalTok{)) }
\CommentTok{# print(sdc, "kAnon")}
\KeywordTok{print}\NormalTok{(sdc, }\StringTok{"ls"}\NormalTok{)}
\end{Highlighting}
\end{Shaded}

\begin{verbatim}
## Local suppression:
##  KeyVar | Suppressions (#) | Suppressions (%)
##  urbrur |                0 |            0.000
##   water |               11 |            0.240
##     sex |                0 |            0.000
##     age |                0 |            0.000
##   relat |              105 |            2.293
## ----------------------------------------------------------------------
\end{verbatim}

\end{frame}

\begin{frame}{kAnon in subsets}
\protect\hypertarget{kanon-in-subsets}{}

Stratification

\begin{itemize}
\tightlist
\item
  The methods can also be applied on each strata separately as long as
  the strata is specified in \texttt{createSdcObj}.
\item
  Automatically then the algorithm ensures \(k\)-anonymity in all
  strata.
\end{itemize}

\end{frame}

\begin{frame}{PRAM}
\protect\hypertarget{pram}{}

Especially if the number of categorical key variables is large or many
of these variables have a high number of different categories - recoding
and local suppression would modify the data too much.

\begin{itemize}
\tightlist
\item
  PRAM is applied to one variable at a time.
\item
  We swap values between categories with pre-defined probabilities.
\item
  An attacker can never be sure if a value is true or has been swapped.
\item
  Probabilities for swapping values are chosen to be small in practice.
\item
  In practice: often the geographical information is swapped using PRAM
\end{itemize}

\end{frame}

\begin{frame}{PRAM Example}
\protect\hypertarget{pram-example}{}

Consider a variable \textit{location} with categories east, middle,
west.

\begin{itemize}
\tightlist
\item
  We define a 3-by-3 transition matrix with \(p_{ij}\) the probabilities
  for swapping category \(i\) to \(j\).
  \(\sum_{j=1}^3 p_{ij} = 1 \quad , \forall i \in \{1,2,3\}\).
\item
  For example, the matrix could look like this:
\end{itemize}

\[ \mathbf{P} = \left( \begin{array}{ccc}
0.9 & 0.05 & 0.05 \\
0.05 & 0.9 & 0.05 \\
0.05 & 0.05 & 0.9 \end{array} \right)\]

\begin{itemize}
\tightlist
\item
  \(\rightarrow\) the probability that a value stays the same is 0.9,
  because \(p_{11}=p_{22}=p_{33}\)
\item
  The probability that east will become middle is \(p_{12} = 0.05\)
\item
  \ldots{}
\end{itemize}

\end{frame}

\begin{frame}[fragile]{PRAM Example}
\protect\hypertarget{pram-example-1}{}

\begin{Shaded}
\begin{Highlighting}[]
\NormalTok{sdc <-}\StringTok{ }\KeywordTok{pram}\NormalTok{(sdc) }\CommentTok{# with standard defaults}
\KeywordTok{print}\NormalTok{(sdc, }\StringTok{"pram"}\NormalTok{)}
\end{Highlighting}
\end{Shaded}

\begin{verbatim}
## Post-Randomization (PRAM):
## Variable: walls 
## --> final Transition-Matrix:
##             2          3            9
## 2 0.931318360 0.06843691 0.0002447332
## 3 0.024745897 0.96939230 0.0058617993
## 9 0.005888281 0.39004413 0.6040675916
## 
## Changed observations:
##   variable nrChanges percChanges
## 1    walls       189        4.13
## ----------------------------------------------------------------------
\end{verbatim}

How to build an own transition matrix: \texttt{?pram}

\end{frame}

\begin{frame}{Microaggregation}
\protect\hypertarget{microaggregation}{}

For continuous key variables. Clustering of observations into groups and
replace values with group means.

\begin{itemize}
\tightlist
\item
  Observations should be as similar as possible within a group.
\item
  Problem is NP-hard. Heuristic algorithm: MDAV
\item
  Assign an aggregated value in each group.

  \begin{itemize}
  \tightlist
  \item
    arithmetic mean or e.g.~robust means
  \end{itemize}
\item
  Application typically applied independently in subgroups, eg,
  independent in all regions.
\item
  also version for mixed-scaled variables available (Gower)
\end{itemize}

\end{frame}

\begin{frame}{Microaggregation Example}
\protect\hypertarget{microaggregation-example}{}

Example with aggregation level 2

\begin{small}
% latex table generated in R 3.6.0 by xtable 1.8-4 package
% Thu Jun 27 13:37:19 2019
\begin{table}[ht]
\centering
\caption{Example of micro-aggregation. Columns 1-3 contain the original variables, columns 4-6 the micro-aggregated values (rounded on two digits).} 
\label{listingMicroaggregation}
\begin{tabular}{|l|lll|lll|}
  \hline
 & Num1 & Num2 & Num3 & Mic1 & Mic2 & Mic3 \\ 
  \hline
1 & 0.30 & 0.400 & 4 & 0.65 & 0.85 & 8.5 \\ 
  2 & 0.12 & 0.220 & 22 & 0.15 & 0.51 & 15.0 \\ 
  3 & 0.18 & 0.800 & 8 & 0.15 & 0.51 & 15.0 \\ 
  4 & 1.90 & 9.000 & 91 & 1.45 & 5.20 & 52.5 \\ 
  5 & 1.00 & 1.300 & 13 & 0.65 & 0.85 & 8.5 \\ 
  6 & 1.00 & 1.400 & 14 & 1.45 & 5.20 & 52.5 \\ 
  7 & 0.10 & 0.010 & 1 & 0.12 & 0.26 & 3.0 \\ 
  8 & 0.15 & 0.500 & 5 & 0.12 & 0.26 & 3.0 \\ 
   \hline
\end{tabular}
\end{table}
\end{small}

\end{frame}

\begin{frame}{Example MDAV, 2-dim}
\protect\hypertarget{example-mdav-2-dim}{}

\includegraphics[width=0.45\textwidth]{/Users/teml/workspace/sdc-springer/book/figures/slidesmt-017}
\includegraphics[width=0.45\textwidth]{/Users/teml/workspace/sdc-springer/book/figures/slidesmt-018}

\end{frame}

\begin{frame}{Example MDAV, 2-dim}
\protect\hypertarget{example-mdav-2-dim-1}{}

\includegraphics[width=0.45\textwidth]{/Users/teml/workspace/sdc-springer/book/figures/slidesmt-019}
\includegraphics[width=0.45\textwidth]{/Users/teml/workspace/sdc-springer/book/figures/slidesmt-020}

\end{frame}

\begin{frame}{Example MDAV, 2-dim}
\protect\hypertarget{example-mdav-2-dim-2}{}

\includegraphics[width=0.45\textwidth]{/Users/teml/workspace/sdc-springer/book/figures/slidesmt-022}
\includegraphics[width=0.45\textwidth]{/Users/teml/workspace/sdc-springer/book/figures/slidesmt-026}

\end{frame}

\begin{frame}[fragile]{Example microaggregation with R}
\protect\hypertarget{example-microaggregation-with-r}{}

\begin{Shaded}
\begin{Highlighting}[]
\NormalTok{sdc <-}\StringTok{ }\KeywordTok{microaggregation}\NormalTok{(sdc, }\DataTypeTok{aggr =} \DecValTok{3}\NormalTok{, }\DataTypeTok{method=}\StringTok{"mdav"}\NormalTok{)}
\KeywordTok{print}\NormalTok{(sdc, }\StringTok{"numrisk"}\NormalTok{)}
\end{Highlighting}
\end{Shaded}

\begin{verbatim}
## Numerical key variables: expend, income, savings
## 
## Disclosure risk (~100.00% in original data):
##   modified data: [0.00%; 93.58%]
## 
## Current Information Loss in modified data (0.00% in original data):
##   IL1: 543198.41
##   Difference of Eigenvalues: 4.530%
## ----------------------------------------------------------------------
\end{verbatim}

Risk measure: use it only for comparison

Utility: we will use better measures later on

\end{frame}

\begin{frame}{Univariate microaggregation}
\protect\hypertarget{univariate-microaggregation}{}

\begin{itemize}
\tightlist
\item
  The individual ranking method (univariate microaggregation) is not
  recommended to use, but often applied because of its simplicity.
\item
  The method replaces values by its aggregates column by column
  independently.
\item
  First, the first column is sorted and the index of sorting is
  memorized to be able to sort the values back in the original order.
  Then the first \(k\) values are replaced by their aggregate (usually
  the arithmetic mean), the next \(k\) values are replaced by their
  aggregate, and so on, until all values are aggregated from the first
  variable. The variable is then back-sorted.
\item
  This procedure is then applied on the other variables independently.
\end{itemize}

Clearly destroys the multivariate structure of the data set.

\end{frame}

\begin{frame}[fragile]{Univariate microaggregation in R}
\protect\hypertarget{univariate-microaggregation-in-r}{}

\begin{Shaded}
\begin{Highlighting}[]
\NormalTok{sdc <-}\StringTok{ }\KeywordTok{undolast}\NormalTok{(sdc)}
\NormalTok{sdc <-}\StringTok{ }\KeywordTok{microaggregation}\NormalTok{(sdc, }\DataTypeTok{aggr =} \DecValTok{3}\NormalTok{, }\DataTypeTok{method=}\StringTok{"onedims"}\NormalTok{)}
\KeywordTok{print}\NormalTok{(sdc, }\StringTok{"numrisk"}\NormalTok{)}
\end{Highlighting}
\end{Shaded}

\begin{verbatim}
## Numerical key variables: expend, income, savings
## 
## Disclosure risk (~100.00% in original data):
##   modified data: [0.00%; 100.00%]
## 
## Current Information Loss in modified data (0.00% in original data):
##   IL1: 444503.76
##   Difference of Eigenvalues: 4.450%
## ----------------------------------------------------------------------
\end{verbatim}

\end{frame}

\begin{frame}{Adding uncorrelated (additive) noise}
\protect\hypertarget{adding-uncorrelated-additive-noise}{}

Normal noise:

\begin{equation}
\mathbf{z}_j = \mathbf{x}_j + \mathbf{\epsilon}_j \quad ,
\end{equation} where vector \(\mathbf{x}_j\) represents the original
values of variable \(j\), \(\mathbf{z}_j\) represents the perturbed
values of variable \(j\) and \(\mathbf{\epsilon}_j\) (uncorrelated
noise, or white noise) denotes normally distributed errors with
\(\epsilon_j \sim N(0, c \cdot s_{x_j})\) with \(c\) a constant and
\(s\) the standard deviation,
\(Cov(\mathbf{\epsilon}_l,\mathbf{\epsilon}_k) = 0\) for all
\(k \neq l\).

Uniform noise: \ldots{}

Mulitplicative noise: \ldots{}

\end{frame}

\begin{frame}{Adding correlated noise}
\protect\hypertarget{adding-correlated-noise}{}

Often the better choice than uncorrelated noise, because the
multivariate structure will not be completely changed.

\begin{itemize}
\item
  The difference to the uncorrelated noise method is that the covariance
  matrix of the errors is now designed to be proportional to the
  covariance of the original data, i.e.
  \(\mathbf{\epsilon} \sim N(0,\Sigma_{\epsilon}=c\Sigma_{\mathbf{X}})\).
\item
  There are several variants of methods available how to achieve this
  (we will not go into details here)
\end{itemize}

\end{frame}

\begin{frame}{Random orthogonal matrix masking}
\protect\hypertarget{random-orthogonal-matrix-masking}{}

Multiplicative correlated noise

\textbf{R}OMM (\textbf{R}andom \textbf{O}rthogonal \textbf{M}atrix
\textbf{M}asking):

perturbed data are obtained by

\[\mathbf{Z} = \mathbf{A} \mathbf{X} ,\]

\begin{itemize}
\tightlist
\item
  whereby \(\mathbf{A}\) is randomly generated and
\item
  fulfils \(\mathbf{A}^{-1} = \mathbf{A}^T\) (orthogonality condition).
\end{itemize}

\end{frame}

\begin{frame}[fragile]{Adding uncorrelated noise in R}
\protect\hypertarget{adding-uncorrelated-noise-in-r}{}

\begin{Shaded}
\begin{Highlighting}[]
\NormalTok{sdc <-}\StringTok{ }\KeywordTok{undolast}\NormalTok{(sdc)}
\NormalTok{sdc <-}\StringTok{ }\KeywordTok{addNoise}\NormalTok{(sdc, }\DataTypeTok{noise =} \DecValTok{5}\NormalTok{, }\DataTypeTok{method=}\StringTok{"additive"}\NormalTok{)}
\KeywordTok{print}\NormalTok{(sdc, }\StringTok{"numrisk"}\NormalTok{)}
\end{Highlighting}
\end{Shaded}

\begin{verbatim}
## Numerical key variables: expend, income, savings
## 
## Disclosure risk (~100.00% in original data):
##   modified data: [0.00%; 46.14%]
## 
## Current Information Loss in modified data (0.00% in original data):
##   IL1: 553879.14
##   Difference of Eigenvalues: 4.280%
## ----------------------------------------------------------------------
\end{verbatim}

\end{frame}

\begin{frame}[fragile]{Adding correlated noise in R}
\protect\hypertarget{adding-correlated-noise-in-r}{}

\begin{Shaded}
\begin{Highlighting}[]
\NormalTok{sdc <-}\StringTok{ }\KeywordTok{undolast}\NormalTok{(sdc)}
\NormalTok{sdc <-}\StringTok{ }\KeywordTok{addNoise}\NormalTok{(sdc, }\DataTypeTok{noise =} \DecValTok{5}\NormalTok{, }\DataTypeTok{method=}\StringTok{"correlated2"}\NormalTok{)}
\KeywordTok{print}\NormalTok{(sdc, }\StringTok{"numrisk"}\NormalTok{)}
\end{Highlighting}
\end{Shaded}

\begin{verbatim}
## Numerical key variables: expend, income, savings
## 
## Disclosure risk (~100.00% in original data):
##   modified data: [0.00%; 15.24%]
## 
## Current Information Loss in modified data (0.00% in original data):
##   IL1: 710940.47
##   Difference of Eigenvalues: 4.160%
## ----------------------------------------------------------------------
\end{verbatim}

See illustrative figures on tollerance ellipses in the book.

\end{frame}

\begin{frame}[fragile]{Multiplicative correlated noise (ROMM)}
\protect\hypertarget{multiplicative-correlated-noise-romm}{}

Loooong computation time\ldots{}

\begin{Shaded}
\begin{Highlighting}[]
\NormalTok{sdc <-}\StringTok{ }\KeywordTok{undolast}\NormalTok{(sdc)}
\NormalTok{sdc <-}\StringTok{ }\KeywordTok{addNoise}\NormalTok{(sdc, }\DataTypeTok{noise =} \DecValTok{5}\NormalTok{, }\DataTypeTok{method=}\StringTok{"ROMM"}\NormalTok{)}
\end{Highlighting}
\end{Shaded}

\end{frame}

\begin{frame}{Adding Noise und Shuffling}
\protect\hypertarget{adding-noise-und-shuffling}{}

\begin{itemize}
\tightlist
\item
  additiver Noise

  \begin{itemize}
  \tightlist
  \item
    unabh"angig auf jede stetige Schl"usselvariable angewandt
  \item
    don't do it (zerst"ort Korrelationen)
  \end{itemize}
\item
  korrelierter Noise

  \begin{itemize}
  \tightlist
  \item
    ber"ucksichtigt die Korrelationsstruktur der Daten
  \item
    Mittel und Varianzen bleiben erhalten \textbackslash{}end\{itemize\}
  \end{itemize}
\item
  Shuffling

  \begin{itemize}
  \tightlist
  \item
    Regressionsmodell: stetige Schl"usselvariablen als Response, andere
    Variablen f"ur Prediktoren
  \item
    Vertauschen von Werten basierend auf R"angen von erwarteten Werten
    und Originalwerten
  \item
    Rangkorrelationen bleiben erhalten
  \item
    Wir wollen nicht genauer auf Shuffling eingehen, da speziell
    Ausreisser nicht verschmutzt werden.
  \end{itemize}
\end{itemize}

\end{frame}

\end{document}
