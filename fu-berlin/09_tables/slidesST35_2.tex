\documentclass[pdfpagelabels=false, usepdftitle=false]{beamer}
\usetheme{CambridgeUS}
\usepackage[ngerman]{babel}
\usecolortheme{default}
\usepackage[latin1]{inputenc}
\usepackage[T1]{fontenc}
\usepackage{ulem}
\usepackage{hyperref}
\usepackage{graphics}
\usepackage{colortbl}
\hypersetup{colorlinks=true, breaklinks=true, linkcolor=STATred, menucolor=STATred, urlcolor=STATred}
\usepackage{listings}
\usepackage[longnamesfirst]{natbib}
\bibpunct{(}{)}{,}{a}{}{,}
%\newcommand*{\newblock}{}  % necessary to make beamer work with natbib
\newcommand{\Rlogo}{\raisebox{-.1mm}{\includegraphics[width=1.2em]{Rlogo.pdf}}}
\usepackage{amsmath}
\newcommand{\mymat}[1]{\boldsymbol{#1}}
\newcommand{\m}[1]{\boldsymbol{#1}}
\newcommand{\R}{\textsf{R}}
\newcommand{\code}[1]{\texttt{#1}}

\newcommand{\w}[1]{\textcolor{white}{#1}}
\newcommand{\wb}[1]{\textcolor{white}{{\bf#1}}}
\newcommand{\bl}[1]{\textcolor{pddblue}{#1}}  % blue
\newcommand{\blb}[1]{\textcolor{pddblue}{{\bf#1}}} % blue bold
\newcommand{\red}[1]{\textcolor{STATred}{#1}}  % red
\newcommand{\redb}[1]{\textcolor{STATred}{{\bf#1}}} % red bold


\newcommand{\mb}[1]{\mathbf{#1}}
\newcommand{\cbwmb}[1] {
	$\cellcolor{pddblue}\textcolor{white}{\mathbf{#1}}$
}
\newcommand{\cbw}[1] {
	\cellcolor{pddblue}\textcolor{white}{#1}
}
\newcommand{\cbwb}[1] {
	\cellcolor{pddblue}\textcolor{white}{{\bf #1}}
}

% grey row bold
\newcommand{\rowgblb}[5] {
	\rowcolor{Gray}
	\textcolor{black}{{\bf #1}} &
	\textcolor{black}{{\bf #2}} &
	\textcolor{black}{{\bf #3}} &
	\textcolor{black}{{\bf #4}} &
	\textcolor{black}{{\bf #5}}
}
\newcommand{\rowbwb}[5] {
	\rowcolor{pddblue}
	\textcolor{white}{{\bf #1}} &
	\textcolor{white}{{\bf #2}} &
	\textcolor{white}{{\bf #3}} &
	\textcolor{white}{{\bf #4}} &
	\textcolor{white}{{\bf #5}}
}

\newcommand{\crw}[1] {
	\cellcolor{STATred}\textcolor{white}{#1}
}
\newcommand{\crwb}[1] {
	\cellcolor{STATred}\textcolor{white}{{\bf #1}}
}
\newcommand{\invisibleRow} {
\w{\mb{h_x}} &\w{=}& \w{h_x (x) + h_x (x) + h_x (x) = xx} \\
}

\newcommand{\noninvisibleRow} {
\bl{\mb{h_x}} &\bl{=}& \bl{h_x (x) + h_x (x) + h_x (x) = xx} \\
}
\newcommand{\noninvisibleRowBold} {
\bl{\mb{h_x}} &\bl{{\bf =}}& \bl{{\bf h_x (x) + h_x (x) + h_x (x) = xx }} \\
}
\newcommand{\tabInvisible}[1] {
	\begin{center}
		\begin{tabular}{rllll}
		\wb{#1} & \wb{A} & \wb{B} & \wb{C} & \wb{Total} \\
		\wb{I} & \w{20} & \w{50} & \w{10} & \wb{80} \\
		\wb{II} & \w{8} & \w{19} & \w{22} & \wb{49} \\
		\wb{III} & \w{17} & \w{32} & \w{12} & \wb{61} \\
		\wb{Total} & \wb{45} & \wb{101} & \wb{44} & \wb{190} \\
		\end{tabular}
	\end{center}
}

\lstloadlanguages{R}
 \lstdefinelanguage{Renhanced}[]{R},%
   alsoother={:_\$}}
 \lstset{language=Renhanced,extendedchars=true,
   basicstyle=\small\ttfamily,
   commentstyle=\color{STATgrey},
   keywordstyle=\color{STATred}\bfseries,
   showstringspaces=false,
   index=[1][keywords],
   indexstyle=\indexfonction
}

\newcommand{\indexfonction}[1]{\index{#1@\texttt{#1}}}

% Farben wie fuer TU
\definecolor{pddblue}{rgb}{.17,.31,.44}
\definecolor{pddblue2}{rgb}{.17,.36,.49}
\definecolor{pddblue3}{rgb}{.2,.35,.60}
\definecolor{tublau}{rgb}{0,0.4,0.66} % ca. #0066AA
\definecolor{rotgold}{rgb}{0.8,0.3,0}
\definecolor{pdlblue}{rgb}{.75,.85,.92}
\definecolor{pdllblue}{rgb}{.9,.95,.98}
\definecolor{pdlparchment}{rgb}{.96,.94,.89}
\definecolor{pddparchment}{rgb}{.89,.85,.69}
\definecolor{Gray}{gray}{0.75}
\definecolor{Gray2}{gray}{0.5}
\definecolor{digreen}{rgb}{.1,.31,.1}
\definecolor{diblue}{rgb}{.0,.2,.35}
\definecolor{diblue2}{rgb}{.0,.3,.2}

% Farben wie bei STAT Webauftritt
\definecolor{STATred}{rgb}{0.694,0.0,0.149}
\definecolor{STATgrey}{rgb}{0.8,0.8,0.8}

\setbeamercolor{frametitle}{fg=white,bg=STATred} %STAT
\setbeamercolor{structure}{fg=STATred}

%STAT
\setbeamercolor{palette primary}{fg=STATred, bg=STATgrey}  %oben rechts
\setbeamercolor{palette secondary}{fg=black, bg=white} % unten mitte
\setbeamercolor{palette tertiary}{fg=white, bg=STATred}
\setbeamercolor{palette quaternary}{fg=white, bg=STATred}
%STAT
\setbeamercolor{palette sidebar primary}{fg=white, bg=blue}
\setbeamercolor{palette sidebar secondary}{fg=black, bg=STATred} % kein Effekt
\setbeamercolor{palette sidebar tertiary}{fg=white, bg=blue}
\setbeamercolor{palette sidebar quaternary}{fg=white, bg=STATred}


\setbeamercolor{normal text}{fg=pddblue, bg=white}
\setbeamercolor{alerted text}{fg=black} %STAT:black

\setbeamercolor{button}{fg=white, bg=black} %STAT:black
\setbeamercolor{button border}{use=button, fg=black} %STAT:black

\setbeamercolor{navigation symbols}{fg=white, bg=black} %STAT:black
\setbeamercolor{navigation symbols dimmed}{fg=white, bg=black} %STAT:black

\setbeamertemplate{navigation symbols}{}    % Unterdrueckt die Navigationssymbole
\definecolor{pddblue}{rgb}{.17,.31,.44}
\setbeamercolor{alerted text}{fg=pddblue}

\beamertemplateballitem

\usepackage{Sweave}

% ---------------------
% begin of presentation
% ---------------------

\begin{document}
\Sconcordance{concordance:slidesST35_2.tex:slidesST35_2.rnw:%
1 304 1 1 2 1 0 1 1 1 2 10 0 1 2 4 1 1 5 7 0 1 2 6 1 1 3 2 0 1 1 8 0 1 %
2 7 1 1 2 1 0 1 1 3 0 1 2 6 1 1 6 5 0 1 1 8 0 1 2 3 1 1 6 8 0 1 2 6 1 1 %
3 2 0 1 1 21 0 1 2 5 1 1 3 5 0 1 2 5 1 1 2 26 0 1 1 21 0 1 2 24 1}

% define title and authors
\title[Part 9: Tabular data protection]{
%\raisebox{-.1mm}{\includegraphics[height=50mm]{logo}} \\
Tabular data protection}
%\pgfdeclareimage[height=0.6cm]{logo}{logo}
%\logo{\pgfuseimage{logo}}
\author[Templ]{Matthias Templ}
\institute[ZHAW, FU Berlin]{}
\date[2019]{2019}

% slides
\begin{frame}
\titlepage
\end{frame}

\setcounter{tocdepth}{1}
\section{}
\begin{frame}{Content}
\tableofcontents
\end{frame}

\color{pddblue}
\section{Introduction}
% \begin{frame}\frametitle{Notwendigkeit statistischer Geheimhaltung}
% 	\begin{itemize}
% 	    \item {\bf unterschiedliche Gr?nde} statistische Geheimhaltung (auch) in Tabellen zu betreiben\pause
% 	    \begin{itemize}
% 	        \item gesetzliche Vorgaben
% 	        \item Respondentenschutz \pause
% 	   	\end{itemize}
% 		\item {\bf Data privacy} ist ein wichtiges Thema\pause
% 	    \begin{itemize}
% 	        \item wem ''geh?ren'' meine Daten
% 	        \item kann ich mich sicher sein, dass Information nicht zu meinem Nachteil verwendet wird \pause
% 	   	\end{itemize}
% 		\item {\bf Kursziele:}
% 	    \begin{itemize}
% 	        \item Bewusst machen, wie Angreifer vorgehen
% 	        \item welche M?glichkeiten gibt es, (tabellarische) Daten zu sch?tzen
% 	  	\end{itemize}
% 	\end{itemize}
% \end{frame}

\begin{frame}\frametitle{Examples}
	\begin{itemize}
	    \item {\bf Example 1: ( Location = Municipality x;  interest in soccer x sex) }:
		\begin{scriptsize}
		\begin{center}
			\begin{tabular}{|r|ll|l|}
			\hline
			& {\bf men} 		& {\bf women} & {\bf total} \\ \hline
			{\bf soccer fan} 		& 12 & 10 & {\bf 22}\\
			{\bf not a soccer fan} 	& 93 & 85 & {\bf 178} \\ \hline
			{\bf total} 			& {\bf 105} & {\bf 95} & {\bf 200} \\ \hline
			\end{tabular}
		\end{center}
		\end{scriptsize} \pause
	    \item {\bf Example 1 (cont.): ( Location = Municipality x; team preference x sex) }:
		\begin{scriptsize}
		\begin{center}
			\begin{tabular}{|r|ll|l|}
			\hline
			& {\bf men} & {\bf women} & {\bf total} \\ \hline
			{\bf Sk Rapid Wien} & 12 & 4 & {\bf 16}\\
			{\bf Sturm Graz} 	& 0 & 3 & {\bf 3} \\
			{\bf SV Ried} 		& 0 & 3 &  {\bf 3} \\ \hline
			{\bf total} 		& {\bf 12} & {\bf 10} & {\bf 22} \\ \hline
			\end{tabular}
		\end{center}
		\end{scriptsize}
			    \item {\bf Example 2: ( Location = Municipality x; education x sex) }:
		\begin{scriptsize}
		\begin{center}
			\begin{tabular}{|r|ll|l|}
			\hline
			& {\bf men} & {\bf women} & {\bf total} \\ \hline
			{\bf primary} 	& 49 & 53 & {\bf 102}\\
			{\bf apprenticeship} 				& 34 & 23 & {\bf 56} \\
			{\bf secondary} 				& 22 & 14 &  {\bf 37} \\
			{\bf university} 	&  0 &  5 & {\bf 5} \\ \hline
			{\bf total} 				& {\bf 105} & {\bf 95} & {\bf 200} \\ \hline
			\end{tabular}
		\end{center}
		\end{scriptsize} \pause
	\end{itemize}
\end{frame}


\section{Methods}
\subsection{Statistical tables}
\begin{frame}\frametitle{tables - intro}
	\begin{itemize}
	    \item tabular data: basis are {\bf micro data} \pause
	    \item {\bf magnitude tables} and {\bf frequency tables} \pause
	    \begin{itemize}
	    	\item {\bf frequency table:} Counts of categories\pause
	    	\item {\bf magnitude table:} sum of all values of a variable. \pause
	    \end{itemize}
	    \item {\bf Generally...}
	    \begin{itemize}
	    	\item tabular data have {\bf linear dependencies} between its cells. \pause
	    	\item tabular data can be  {\bf one- or multi-dimensional}, {\bf hierarchical} and/or {\bf linked}.
		\end{itemize}
	\end{itemize}
\end{frame}


%%%%%%%%%%%%%%%%%%%%%%%%%%%%%%%%%%%%%%
%%%%% Von Mikrodaten zur Tabelle %%%%%
\begin{frame}\frametitle{View on a table}
	\vspace{-0.2cm}
	\begin{center}
	\begin{tabular}{|r|ccc|}
	\hline
	{\bf ID} & {\bf DIM1} & {\bf DIM2} & {\bf VALUE}  \\ \hline
	1 & I & A & 5 \\ 
	2 & I & A & 7 \\ 
	3 & I & A & 4 \\	
	4 & I & A & 4 \\
	5 & I & B & 13 \\ 
	6 & I & B & 5 \\ 	
	. & . & . & . \\ 	
	. & . & . & . \\ 			
	\hline
	\end{tabular}
	\end{center}
	
	\vspace{-1cm}
	
	\begin{columns}
	\begin{column}{5cm}
		\tabInvisible{H}
	\end{column}
	\begin{column}{5cm}
		\tabInvisible{W}
		\end{column}
	\end{columns}
\end{frame}

\begin{frame}\frametitle{From microdata to tables}
	\vspace{-0.2cm}
	\begin{center}
	\begin{tabular}{|r|ccc|}
	\hline
	{\bf ID} & {\bf DIM1} & {\bf DIM2} & {\bf VALUE}  \\ \hline
	1 & I & A & 5 \\ 
	2 & I & A & 7 \\ 
	3 & I & A & 4 \\	
	4 & I & A & 4 \\
	5 & I & B & 13 \\ 
	6 & I & B & 5 \\ 	
	. & . & . & . \\ 	
	. & . & . & . \\ 			
	\hline
	\end{tabular}
	\end{center}
	
	\vspace{-1cm}
	
	\begin{columns}
	\begin{column}{5cm}
		\begin{center}
			\begin{tabular}{|r|lll|l|}
			\hline
			{\bf H} & {\bf A} & {\bf B} & {\bf C} & {\bf Total} \\ 
			\hline
			{\bf I} & $h_1$ & $h_2$ & $h_3$ & $\mb{h_4}$ \\ 
			{\bf II} & $h_5$ & $h_6$ & $h_7$ & $\mb{h_8}$ \\ 
			{\bf III} & $h_9$ & $h_{10}$ & $h_{11}$ & $\mb{h_{12}}$ \\ 
			\hline
			{\bf Total} & $\mb{h_{13}}$ & $\mb{h_{14}}$ & $\mb{h_{15}}$ & $\mb{h_{16}}$ \\
			\hline
			\end{tabular}
		\end{center}	
	\end{column}
	\begin{column}{5cm}
		\tabInvisible{W}
	\end{column}
	\end{columns}
\end{frame}

\begin{frame}\frametitle{From microdata to tables}
	\vspace{-0.2cm}
	\begin{center}
	\begin{tabular}{|r|ccc|}
	\hline
	{\bf ID} & {\bf DIM1} & {\bf DIM2} & {\bf VALUE}  \\ \hline
	1 & I & A & 5 \\ 
	2 & I & A & 7 \\ 
	3 & I & A & 4 \\	
	4 & I & A & 4 \\
	5 & I & B & 13 \\ 
	6 & I & B & 5 \\ 	
	. & . & . & . \\ 	
	. & . & . & . \\ 			
	\hline
	\end{tabular}
	\end{center}
	
	\vspace{-1cm}
	
	\begin{columns}
	\begin{column}{5cm}
		\begin{center}
			\begin{tabular}{|r|lll|l|}
			\hline
			{\bf H} & {\bf A} & {\bf B} & {\bf C} & {\bf Total} \\ 
			\hline
			{\bf I} & $h_1$ & $h_2$ & $h_3$ & $\mb{h_4}$ \\ 
			{\bf II} & $h_5$ & $h_6$ & $h_7$ & $\mb{h_8}$ \\ 
			{\bf III} & $h_9$ & $h_{10}$ & $h_{11}$ & $\mb{h_{12}}$ \\ 
			\hline
			{\bf Total} & $\mb{h_{13}}$ & $\mb{h_{14}}$ & $\mb{h_{15}}$ & $\mb{h_{16}}$ \\
			\hline
			\end{tabular}
		\end{center}		
	\end{column}
	\begin{column}{5cm}
		\begin{center}
			\begin{tabular}{|r|lll|l|}
			\hline
			{\bf W} & {\bf A} & {\bf B} & {\bf C} & {\bf Total} \\ 
			\hline
			{\bf I} & $y_1$ & $y_2$ & $y_3$ & $\mb{y_4}$ \\ 
			{\bf II} & $y_5$ & $y_6$ & $y_7$ & $\mb{y_8}$ \\ 
			{\bf III} & $y_9$ & $y_{10}$ & $y_{11}$ & $\mb{y_{12}}$ \\ 
			\hline
			{\bf Total} & $\mb{y_{13}}$ & $\mb{y_{14}}$ & $\mb{y_{15}}$ & $\mb{y_{16}}$ \\
			\hline
			\end{tabular}
		\end{center}			
		\end{column}
	\end{columns}
\end{frame}

\begin{frame}\frametitle{From microdata to tables}
	\vspace{-0.2cm}
	\begin{center}
	\begin{tabular}{|r|ccc|}
	\hline
	{\bf ID} & {\bf DIM1} & {\bf DIM2} & {\bf VALUE}  \\ \hline
	1 & I & A & 5 \\ 
	2 & I & A & 7 \\ 
	3 & I & A & 4 \\	
	4 & I & A & 4 \\
	5 & I & B & 13 \\ 
	6 & I & B & 5 \\ 	
	. & . & . & . \\ 	
	. & . & . & . \\ 			
	\hline
	\end{tabular}
	\end{center}
	
	\vspace{-1cm}
	
	\begin{columns}
	\begin{column}{5cm}
		\begin{center}
			\begin{tabular}{|r|lll|l|}
			\hline
			{\bf H} & {\bf A} & {\bf B} & {\bf C} & {\bf Total} \\ 
			\hline
			{\bf I} & \cbw{$h_1$} & $h_2$ & $h_3$ & $\mb{h_4}$ \\ 
			{\bf II} & $h_5$ & $h_6$ & $h_7$ & $\mb{h_8}$ \\ 
			{\bf III} & $h_9$ & $h_{10}$ & $h_{11}$ & $\mb{h_{12}}$ \\ 
			\hline
			{\bf Total} & $\mb{h_{13}}$ & $\mb{h_{14}}$ & $\mb{h_{15}}$ & $\mb{h_{16}}$ \\
			\hline
			\end{tabular}
		\end{center}		
	\end{column}
	\begin{column}{5cm}
		\begin{center}
			\begin{tabular}{|r|lll|l|}
			\hline
			{\bf W} & {\bf A} & {\bf B} & {\bf C} & {\bf Total} \\ 
			\hline
			{\bf I} & $y_1$ & $y_2$ & $y_3$ & $\mb{y_4}$ \\ 
			{\bf II} & $y_5$ & $y_6$ & $y_7$ & $\mb{y_8}$ \\ 
			{\bf III} & $y_9$ & $y_{10}$ & $y_{11}$ & $\mb{y_{12}}$ \\ 
			\hline
			{\bf Total} & $\mb{y_{13}}$ & $\mb{y_{14}}$ & $\mb{y_{15}}$ & $\mb{y_{16}}$ \\
			\hline
			\end{tabular}
		\end{center}			
		\end{column}
	\end{columns}
\end{frame}

\begin{frame}\frametitle{From microdata to tables}
	\vspace{-0.2cm}
	\begin{center}
	\begin{tabular}{|r|ccc|}
	\hline
	{\bf ID} & {\bf DIM1} & {\bf DIM2} & {\bf VALUE}  \\ \hline
	1 & I & A & 5 \\ 
	2 & I & A & 7 \\ 
	3 & I & A & 4 \\	
	4 & I & A & 4 \\
	5 & I & B & 13 \\ 
	6 & I & B & 5 \\ 	
	. & . & . & . \\ 	
	. & . & . & . \\ 			
	\hline
	\end{tabular}
	\end{center}
	
	\vspace{-1cm}
	
	\begin{columns}
	\begin{column}{5cm}
		\begin{center}
			\begin{tabular}{|r|lll|l|}
			\hline
			{\bf H} & {\bf A} & {\bf B} & {\bf C} & {\bf Total} \\ 
			\hline
			{\bf I} & \cbw{$h_1$} & $h_2$ & $h_3$ & $\mb{h_4}$ \\ 
			{\bf II} & $h_5$ & $h_6$ & $h_7$ & $\mb{h_8}$ \\ 
			{\bf III} & $h_9$ & $h_{10}$ & $h_{11}$ & $\mb{h_{12}}$ \\ 
			\hline
			{\bf Total} & $\mb{h_{13}}$ & $\mb{h_{14}}$ & $\mb{h_{15}}$ & $\mb{h_{16}}$ \\
			\hline
			\end{tabular}
		\end{center}		
	\end{column}
	\begin{column}{5cm}
		\begin{center}
			\begin{tabular}{|r|lll|l|}
			\hline
			{\bf W} & {\bf A} & {\bf B} & {\bf C} & {\bf Total} \\ 
			\hline
			{\bf I} & \cbw{$y_1$} & $y_2$ & $y_3$ & $\mb{y_4}$ \\ 
			{\bf II} & $y_5$ & $y_6$ & $y_7$ & $\mb{y_8}$ \\ 
			{\bf III} & $y_9$ & $y_{10}$ & $y_{11}$ & $\mb{y_{12}}$ \\ 
			\hline
			{\bf Total} & $\mb{y_{13}}$ & $\mb{y_{14}}$ & $\mb{y_{15}}$ & $\mb{y_{16}}$ \\
			\hline
			\end{tabular}
		\end{center}			
		\end{column}
	\end{columns}
\end{frame}

\begin{frame}\frametitle{From microdata to tables}
	\vspace{-0.2cm}
	\begin{center}
	\begin{tabular}{|r|ccc|}
	\hline
	{\bf ID} & {\bf DIM1} & {\bf DIM2} & {\bf VALUE}  \\ \hline
	\rowcolor{pddblue} \w{1} & \w{I} & \w{A} & \w{5} \\ 
	\rowcolor{pddblue} \w{2} & \w{I} & \w{A} & \w{7} \\ 
	\rowcolor{pddblue} \w{3} & \w{I} & \w{A} & \w{4} \\	
	\rowcolor{pddblue} \w{4} & \w{I} & \w{A} & \w{4} \\
	5 & I & B & 13 \\ 
	6 & I & B & 5 \\ 	
	. & . & . & . \\ 	
	. & . & . & . \\ 			
	\hline
	\end{tabular}
	\end{center}
	
	\vspace{-1cm}
	
	\begin{columns}
	\begin{column}{5cm}
		\begin{center}
			\begin{tabular}{|r|lll|l|}
			\hline
			{\bf H} & {\bf A} & {\bf B} & {\bf C} & {\bf Total} \\ 
			\hline
			{\bf I} & \cbw{$h_1$} & $h_2$ & $h_3$ & $\mb{h_4}$ \\ 
			{\bf II} & $h_5$ & $h_6$ & $h_7$ & $\mb{h_8}$ \\ 
			{\bf III} & $h_9$ & $h_{10}$ & $h_{11}$ & $\mb{h_{12}}$ \\ 
			\hline
			{\bf Total} & $\mb{h_{13}}$ & $\mb{h_{14}}$ & $\mb{h_{15}}$ & $\mb{h_{16}}$ \\
			\hline
			\end{tabular}
		\end{center}		
	\end{column}
	\begin{column}{5cm}
		\begin{center}
			\begin{tabular}{|r|lll|l|}
			\hline
			{\bf W} & {\bf A} & {\bf B} & {\bf C} & {\bf Total} \\ 
			\hline
			{\bf I} & \cbw{$y_1$} & $y_2$ & $y_3$ & $\mb{y_4}$ \\ 
			{\bf II} & $y_5$ & $y_6$ & $y_7$ & $\mb{y_8}$ \\ 
			{\bf III} & $y_9$ & $y_{10}$ & $y_{11}$ & $\mb{y_{12}}$ \\ 
			\hline
			{\bf Total} & $\mb{y_{13}}$ & $\mb{y_{14}}$ & $\mb{y_{15}}$ & $\mb{y_{16}}$ \\
			\hline
			\end{tabular}
		\end{center}			
		\end{column}
	\end{columns}
\end{frame}

\begin{frame}\frametitle{From microdata to tables}
	\vspace{-0.2cm}
	\begin{center}
	\begin{tabular}{|r|ccc|}
	\hline
	{\bf ID} & {\bf DIM1} & {\bf DIM2} & {\bf VALUE}  \\ \hline
	\rowcolor{pddblue} \w{1} & \w{I} & \w{A} & \w{5} \\ 
	\rowcolor{pddblue} \w{2} & \w{I} & \w{A} & \w{7} \\ 
	\rowcolor{pddblue} \w{3} & \w{I} & \w{A} & \w{4} \\	
	\rowcolor{pddblue} \w{4} & \w{I} & \w{A} & \w{4} \\
	5 & I & B & 13 \\ 
	6 & I & B & 5 \\ 	
	. & . & . & . \\ 	
	. & . & . & . \\ 			
	\hline
	\end{tabular}
	\end{center}
	
	\vspace{-1cm}
	
	\begin{columns}
	\begin{column}{5cm}
		\begin{center}
			\begin{tabular}{|r|lll|l|}
			\hline
			{\bf H} & {\bf A} & {\bf B} & {\bf C} & {\bf Total} \\ 
			\hline
			{\bf I} & \bl{4} & $h_2$ & $h_3$ & $\mb{h_4}$ \\ 
			{\bf II} & $h_5$ & $h_6$ & $h_7$ & $\mb{h_8}$ \\ 
			{\bf III} & $h_9$ & $h_{10}$ & $h_{11}$ & $\mb{h_{12}}$ \\ 
			\hline
			{\bf Total} & $\mb{h_{13}}$ & $\mb{h_{14}}$ & $\mb{h_{15}}$ & $\mb{h_{16}}$ \\
			\hline
			\end{tabular}
		\end{center}		
	\end{column}
	\begin{column}{5cm}
		\begin{center}
			\begin{tabular}{|r|lll|l|}
			\hline
			{\bf W} & {\bf A} & {\bf B} & {\bf C} & {\bf Total} \\ 
			\hline
			{\bf I} & \cbw{$y_1$} & $y_2$ & $y_3$ & $\mb{y_4}$ \\ 
			{\bf II} & $y_5$ & $y_6$ & $y_7$ & $\mb{y_8}$ \\ 
			{\bf III} & $y_9$ & $y_{10}$ & $y_{11}$ & $\mb{y_{12}}$ \\ 
			\hline
			{\bf Total} & $\mb{y_{13}}$ & $\mb{y_{14}}$ & $\mb{y_{15}}$ & $\mb{y_{16}}$ \\
			\hline
			\end{tabular}
		\end{center}			
		\end{column}
	\end{columns}
\end{frame}

\begin{frame}\frametitle{From microdata to tables}
	\vspace{-0.2cm}
	\begin{center}
	\begin{tabular}{|r|ccc|}
	\hline
	{\bf ID} & {\bf DIM1} & {\bf DIM2} & {\bf VALUE}  \\ \hline
	\rowcolor{pddblue} \w{1} & \w{I} & \w{A} & \w{5} \\ 
	\rowcolor{pddblue} \w{2} & \w{I} & \w{A} & \w{7} \\ 
	\rowcolor{pddblue} \w{3} & \w{I} & \w{A} & \w{4} \\	
	\rowcolor{pddblue} \w{4} & \w{I} & \w{A} & \w{4} \\
	5 & I & B & 13 \\ 
	6 & I & B & 5 \\ 	
	. & . & . & . \\ 	
	. & . & . & . \\ 			
	\hline
	\end{tabular}
	\end{center}
	
	\vspace{-1cm}
	
	\begin{columns}
	\begin{column}{5cm}
		\begin{center}
			\begin{tabular}{|r|lll|l|}
			\hline
			{\bf H} & {\bf A} & {\bf B} & {\bf C} & {\bf Total} \\ 
			\hline
			{\bf I} & \bl{4} & $h_2$ & $h_3$ & $\mb{h_4}$ \\ 
			{\bf II} & $h_5$ & $h_6$ & $h_7$ & $\mb{h_8}$ \\ 
			{\bf III} & $h_9$ & $h_{10}$ & $h_{11}$ & $\mb{h_{12}}$ \\ 
			\hline
			{\bf Total} & $\mb{h_{13}}$ & $\mb{h_{14}}$ & $\mb{h_{15}}$ & $\mb{h_{16}}$ \\
			\hline
			\end{tabular}
		\end{center}		
	\end{column}
	\begin{column}{5cm}
		\begin{center}
			\begin{tabular}{|r|lll|l|}
			\hline
			{\bf W} & {\bf A} & {\bf B} & {\bf C} & {\bf Total} \\ 
			\hline
			{\bf I} & \bl{20} & $y_2$ & $y_3$ & $\mb{y_4}$ \\ 
			{\bf II} & $y_5$ & $y_6$ & $y_7$ & $\mb{y_8}$ \\ 
			{\bf III} & $y_9$ & $y_{10}$ & $y_{11}$ & $\mb{y_{12}}$ \\ 
			\hline
			{\bf Total} & $\mb{y_{13}}$ & $\mb{y_{14}}$ & $\mb{y_{15}}$ & $\mb{y_{16}}$ \\
			\hline
			\end{tabular}
		\end{center}			
		\end{column}
	\end{columns}
\end{frame}

\begin{frame}\frametitle{From microdata to tables}
	\begin{itemize}
		\item proceed ... 
		\begin{center}
		\begin{columns}
			\begin{column}{5cm}
				\begin{tabular}{|r|lll|l|}
				\hline
				{\bf H} & {\bf A} & {\bf B} & {\bf C} & {\bf Total} \\
				\hline
				{\bf I} 	& 4 & 6 & 3 & $\mb{h_4}$ \\
				{\bf II} 	& 2 & 5 & 7 & $\mb{h_8}$ \\
				{\bf III}   & 4 & 5 & 3 & $\mb{h_{12}}$ \\
				\hline
				{\bf Total} & $\mb{h_{13}}$ & $\mb{h_{14}}$ & $\mb{h_{15}}$ & $\mb{h_{16}}$ \\
				\hline
				\end{tabular}			
			\end{column}
			\begin{column}{5cm}
				\begin{tabular}{|r|lll|l|}
				\hline
				{\bf H} & {\bf A} & {\bf B} & {\bf C} & {\bf Total} \\
				\hline
				{\bf I} 	& 20 & 50 & 10 & $\mb{y_4}$ \\
				{\bf II} 	& 8 & 19 & 22 & $\mb{y_8}$ \\
				{\bf III}   & 17 & 32 & 12 & $\mb{y_{12}}$ \\
				\hline
				{\bf Total} & $\mb{y_{13}}$ & $\mb{y_{14}}$ & $\mb{y_{15}}$ & $\mb{y_{16}}$
				\\
				\hline
				\end{tabular}				
			\end{column}			
		\end{columns}
		\end{center}	
		
		\pause
		\item common wording: {\bf marginal totals} \pause
		\item {\bf 2-dimensional case}: {\bf row-} and {\bf column sums}. \\
	\end{itemize}
\end{frame}
% 
%%% Randsummen (1) %%%
\begin{frame}\frametitle{From microdata to tables}
	\begin{columns}
	\begin{column}{5cm}
		\begin{center}
			\begin{tabular}{|r|lll|l|}
			\hline
			{\bf H} & {\bf A} & {\bf B} & {\bf C} & {\bf Total} \\
			\hline
			{\bf I} 	& \crw{4} & \crw{6} & \crw{3} & $\cbw{\mb{h_4}}$ \\
			{\bf II} 	& \crw{2} & \crw{5} & \crw{7} & $\cbw{\mb{h_8}}$ \\
			{\bf III}   & \crw{4} & \crw{5} & \crw{3} & $\cbw{\mb{h_{12}}}$ \\
			\hline
			{\bf Total} & $\mb{h_{13}}$ & $\mb{h_{14}}$ & $\mb{h_{15}}$ & $\mb{h_{16}}$ \\
			\hline
			\end{tabular}
		\end{center}

		\begin{scriptsize}
		\begin{eqnarray*}
			\invisibleRow \invisibleRow \invisibleRow \invisibleRow
			\invisibleRow \invisibleRow	\invisibleRow \invisibleRow
		\end{eqnarray*}
		\end{scriptsize}

	\end{column}
	\begin{column}{5cm}
		\begin{center}
			\begin{tabular}{|r|lll|l|}
			\hline
			{\bf W} & {\bf A} & {\bf B} & {\bf C} & {\bf Total} \\
			\hline
			{\bf I}   & \crw{20} & \crw{50} & \crw{10} & $\cbw{\mb{y_4}}$ \\
			{\bf II}  & \crw{8}  & \crw{19} & \crw{22} & $\cbw{\mb{y_8}}$ \\
			{\bf III} & \crw{17} & \crw{32} & \crw{12} & $\cbw{\mb{y_{12}}}$ \\
			\hline
			{\bf Total} & $\mb{y_{13}}$ & $\mb{y_{14}}$ & $\mb{y_{15}}$ & $\mb{y_{16}}$ \\
			\hline
			\end{tabular}
		\end{center}

		\begin{scriptsize}
		\begin{eqnarray*}
			\invisibleRow \invisibleRow \invisibleRow \invisibleRow
			\invisibleRow \invisibleRow	\invisibleRow \invisibleRow
		\end{eqnarray*}
		\end{scriptsize}
		\end{column}
	\end{columns}
\end{frame}

\begin{frame}\frametitle{From microdata to tables}
	\begin{columns}
	\begin{column}{5cm}
		\begin{center}
			\begin{tabular}{|r|lll|l|}
			\hline
			{\bf H} & {\bf A} & {\bf B} & {\bf C} & {\bf Total} \\
			\hline
			{\bf I} 	& \crw{4} & \crw{6} & \crw{3} & $\cbw{\mb{h_4}}$ \\
			{\bf II} 	& \crw{2} & \crw{5} & \crw{7} & $\cbw{\mb{h_8}}$ \\
			{\bf III}   & \crw{4} & \crw{5} & \crw{3} & $\cbw{\mb{h_{12}}}$ \\
			\hline
			{\bf Total} & $\mb{h_{13}}$ & $\mb{h_{14}}$ & $\mb{h_{15}}$ & $\mb{h_{16}}$ \\
			\hline
			\end{tabular}
		\end{center}

		\begin{scriptsize}
		\begin{eqnarray*}
			\mb{h_4} 	 &=& h_1 (4) + h_2 (6) + h_3 (3) = \redb{13} \\
			\mb{h_8} 	 &=& h_5 (2) + h_6 (5) + h_7 (7) = \redb{14} \\
			\mb{h_{12}}  &=& h_9 (4) + h_{10} (5) + h_{11} (3) = \redb{12} \\	
			\invisibleRow \invisibleRow \invisibleRow \invisibleRow \invisibleRow
		\end{eqnarray*}
		\end{scriptsize}

	\end{column}
	\begin{column}{5cm}
		\begin{center}
			\begin{tabular}{|r|lll|l|}
			\hline
			{\bf W} & {\bf A} & {\bf B} & {\bf C} & {\bf Total} \\
			\hline
			{\bf I}   & \crw{20} & \crw{50} & \crw{10} & $\cbw{\mb{y_4}}$ \\
			{\bf II}  & \crw{8}  & \crw{19} & \crw{22} & $\cbw{\mb{y_8}}$ \\
			{\bf III} & \crw{17} & \crw{32} & \crw{12} & $\cbw{\mb{y_{12}}}$ \\
			\hline
			{\bf Total} & $\mb{y_{13}}$ & $\mb{y_{14}}$ & $\mb{y_{15}}$ & $\mb{y_{16}}$ \\
			\hline
			\end{tabular}
		\end{center}

		\begin{scriptsize}
		\begin{eqnarray*}
			\mb{y_4} 	 &=& y_1 (20) + y_2 (50) + y_3 (10) = \redb{80} \\
			\mb{y_8} 	 &=& y_5 (8) + y_6 (19) + y_7 (22) = \redb{49} \\
			\mb{y_{12}}  &=& y_9 (17) + y_{10} (32) + y_{11} (12) = \redb{61} \\	
			\invisibleRow \invisibleRow \invisibleRow \invisibleRow \invisibleRow
		\end{eqnarray*}
		\end{scriptsize}
		\end{column}
	\end{columns}
\end{frame}

\begin{frame}\frametitle{From microdata to tables}
	\begin{columns}
	\begin{column}{5cm}
		\begin{center}
			\begin{tabular}{|r|lll|l|}
			\hline
			{\bf H} & {\bf A} & {\bf B} & {\bf C} & {\bf Total} \\
			\hline
			{\bf I} 	& 4 & 6 & 3 & \redb{13} \\
			{\bf II} 	& 2 & 5 & 7 & \redb{14} \\
			{\bf III}   & 4 & 5 & 3 & \redb{12} \\
			\hline
			{\bf Total} & $\mb{h_{13}}$ & $\mb{h_{14}}$ & $\mb{h_{15}}$ & $\mb{h_{16}}$ \\
			\hline
			\end{tabular}
		\end{center}

		\begin{scriptsize}
		\begin{eqnarray*}
			\mb{h_4} 	 &=& h_1 (4) + h_2 (6) + h_3 (3) = \redb{13} \\
			\mb{h_8} 	 &=& h_5 (2) + h_6 (5) + h_7 (7) = \redb{14} \\
			\mb{h_{12}}  &=& h_9 (4) + h_{10} (5) + h_{11} (3) = \redb{12} \\	
			\invisibleRow \invisibleRow \invisibleRow \invisibleRow \invisibleRow
		\end{eqnarray*}
		\end{scriptsize}

	\end{column}
	\begin{column}{5cm}
		\begin{center}
			\begin{tabular}{|r|lll|l|}
			\hline
			{\bf W} & {\bf A} & {\bf B} & {\bf C} & {\bf Total} \\
			\hline
			{\bf I}   & 20 & 50 & 10 & \redb{80} \\
			{\bf II}  & 8  & 19 & 22 & \redb{49} \\
			{\bf III} & 17 & 32 & 12 & \redb{61} \\
			\hline
			{\bf Total} & $\mb{y_{13}}$ & $\mb{y_{14}}$ & $\mb{y_{15}}$ & $\mb{y_{16}}$ \\
			\hline
			\end{tabular}
		\end{center}

		\begin{scriptsize}
		\begin{eqnarray*}
			\mb{y_4} 	 &=& y_1 (20) + y_2 (50) + y_3 (10) = \redb{80} \\
			\mb{y_8} 	 &=& y_5 (8) + y_6 (19) + y_7 (22) = \redb{49} \\
			\mb{y_{12}}  &=& y_9 (17) + y_{10} (32) + y_{11} (12) = \redb{61} \\	
			\invisibleRow \invisibleRow \invisibleRow \invisibleRow \invisibleRow
		\end{eqnarray*}
		\end{scriptsize}
		\end{column}
	\end{columns}
\end{frame}

%%% Randsummen (2) %%% 
\begin{frame}\frametitle{From microdata to tables}
	\begin{columns}
	\begin{column}{5cm}
		\begin{center}
			\begin{tabular}{|r|lll|l|}
			\hline
			{\bf H} & {\bf A} & {\bf B} & {\bf C} & {\bf Total} \\
			\hline
			{\bf I} 	& \crw{4} & \crw{6} & \crw{3} & \textbf{13} \\
			{\bf II} 	& \crw{2} & \crw{5} & \crw{7} & \textbf{14} \\
			{\bf III}   & \crw{4} & \crw{5} & \crw{3} & \textbf{12} \\
			\hline
			{\bf Total} & $\cbw{\mb{h_{13}}}$ & $\cbw{\mb{h_{14}}}$ & $\cbw{\mb{h_{15}}}$
			& $\mb{h_{16}}$
			\\
			\hline
			\end{tabular}
		\end{center}

		\begin{scriptsize}
		\begin{eqnarray*}
			\mb{h_4} 	 &=& h_1 (4) + h_2 (6) + h_3 (3) = \textbf{13} \\
			\mb{h_8} 	 &=& h_5 (2) + h_6 (5) + h_7 (7) = \textbf{14} \\
			\mb{h_{12}}  &=& h_9 (4) + h_{10} (5) + h_{11} (3) = \textbf{12} \\	
			\invisibleRow \invisibleRow \invisibleRow \invisibleRow \invisibleRow
		\end{eqnarray*}
		\end{scriptsize}

	\end{column}
	\begin{column}{5cm}
		\begin{center}
			\begin{tabular}{|r|lll|l|}
			\hline
			{\bf W} & {\bf A} & {\bf B} & {\bf C} & {\bf Total} \\
			\hline
			{\bf I}   & \crw{20} &  \crw{50} &  \crw{10} & $\textbf{80}$ \\
			{\bf II}  & \crw{8}  &  \crw{19} &  \crw{22} & $\textbf{49}$ \\
			{\bf III} & \crw{17} &  \crw{32} &  \crw{12} & $\textbf{61}$ \\
			\hline
			{\bf Total} & $\cbw{\mb{y_{13}}}$ & $\cbw{\mb{y_{14}}}$ & $\cbw{\mb{y_{15}}}$
			& $\mb{y_{16}}$
			\\
			\hline
			\end{tabular}
		\end{center}

		\begin{scriptsize}
		\begin{eqnarray*}
			\mb{y_4} 	 &=& y_1 (20) + y_2 (50) + y_3 (10) = \textbf{80} \\
			\mb{y_8} 	 &=& y_5 (8) + y_6 (19) + y_7 (22) = \textbf{49} \\
			\mb{y_{12}}  &=& y_9 (17) + y_{10} (32) + y_{11} (12) = \textbf{61} \\	
			\invisibleRow \invisibleRow \invisibleRow \invisibleRow \invisibleRow
		\end{eqnarray*}
		\end{scriptsize}
		\end{column}
	\end{columns}
\end{frame} 
 
\begin{frame}\frametitle{From microdata to tables}
	\begin{columns}
	\begin{column}{5cm}
		\begin{center}
			\begin{tabular}{|r|lll|l|}
			\hline
			{\bf H} & {\bf A} & {\bf B} & {\bf C} & {\bf Total} \\
			\hline
			{\bf I} 	& \crw{4} & \crw{6} & \crw{3} & \textbf{13} \\
			{\bf II} 	& \crw{2} & \crw{5} & \crw{7} & \textbf{14} \\
			{\bf III}   & \crw{4} & \crw{5} & \crw{3} & \textbf{12} \\
			\hline
			{\bf Total} & $\cbw{\mb{h_{13}}}$ & $\cbw{\mb{h_{14}}}$ & $\cbw{\mb{h_{15}}}$
			& $\mb{h_{16}}$
			\\
			\hline
			\end{tabular}
		\end{center}

		\begin{scriptsize}
		\begin{eqnarray*}
			\mb{h_4} 	 &=& h_1 (4) + h_2 (6) + h_3 (3) = \textbf{13} \\
			\mb{h_8} 	 &=& h_5 (2) + h_6 (5) + h_7 (7) = \textbf{14} \\
			\mb{h_{12}}  &=& h_9 (4) + h_{10} (5) + h_{11} (3) = \textbf{12} \\	
			\mb{h_{13}} &=& h_1 (4) + h_5 (2) + h_9 (4) = \redb{10} \\
 			\mb{h_{14}} &=& h_2 (6) + h_6 (5) + h_{10} (5) = \redb{16} \\
 			\mb{h_{15}} &=& h_3 (3) + h_7 (7) + h_{11} (3) = \redb{13} \\	
			\invisibleRow \invisibleRow
		\end{eqnarray*}
		\end{scriptsize}

	\end{column}
	\begin{column}{5cm}
		\begin{center}
			\begin{tabular}{|r|lll|l|}
			\hline
			{\bf W} & {\bf A} & {\bf B} & {\bf C} & {\bf Total} \\
			\hline
			{\bf I}   & \crw{20} &  \crw{50} &  \crw{10} & \textbf{80} \\
			{\bf II}  & \crw{8}  &  \crw{19} &  \crw{22} & \textbf{49} \\
			{\bf III} & \crw{17} &  \crw{32} &  \crw{12} & \textbf{61} \\
			\hline
			{\bf Total} & $\cbw{\mb{y_{13}}}$ & $\cbw{\mb{y_{14}}}$ & $\cbw{\mb{y_{15}}}$
			& $\mb{y_{16}}$
			\\
			\hline
			\end{tabular}
		\end{center}

		\begin{scriptsize}
		\begin{eqnarray*}
			\mb{y_4} 	 &=& y_1 (20) + y_2 (50) + y_3 (10) = \textbf{80} \\
			\mb{y_8} 	 &=& y_5 (8) + y_6 (19) + y_7 (22) = \textbf{49} \\
			\mb{y_{12}}  &=& y_9 (17) + y_{10} (32) + y_{11} (12) = \textbf{61} \\	
			\mb{y_{13}} &=& y_1 (20) + y_5 (8) + y_9 (17) = \redb{45} \\
 			\mb{y_{14}} &=& y_2 (50) + y_6 (19) + y_{32} (5) = \redb{101} \\
 			\mb{y_{15}} &=& y_3 (10) + y_7 (22) + y_{12} (3) = \redb{44} \\	
			\invisibleRow \invisibleRow
		\end{eqnarray*}
		\end{scriptsize}
		\end{column}
	\end{columns}
\end{frame}  
 
\begin{frame}\frametitle{From microdata to tables}
	\begin{columns}
	\begin{column}{5cm}
		\begin{center}
			\begin{tabular}{|r|lll|l|}
			\hline
			{\bf H} & {\bf A} & {\bf B} & {\bf C} & {\bf Total} \\
			\hline
			{\bf I} 	& 4 & 6 & 3 & \textbf{13} \\
			{\bf II} 	& 2 & 5 & 7 & \textbf{14} \\
			{\bf III}   & 4 & 5 & 3 & \textbf{12} \\
			\hline
			{\bf Total} & \redb{10} & \redb{16} & \redb{13}
			& $\mb{h_{16}}$
			\\
			\hline
			\end{tabular}
		\end{center}

		\begin{scriptsize}
		\begin{eqnarray*}
			\mb{h_4} 	 &=& h_1 (4) + h_2 (6) + h_3 (3) = \textbf{13} \\
			\mb{h_8} 	 &=& h_5 (2) + h_6 (5) + h_7 (7) = \textbf{14} \\
			\mb{h_{12}}  &=& h_9 (4) + h_{10} (5) + h_{11} (3) = \textbf{12} \\	
			\mb{h_{13}} &=& h_1 (4) + h_5 (2) + h_9 (4) = \redb{10} \\
 			\mb{h_{14}} &=& h_2 (6) + h_6 (5) + h_{10} (5) = \redb{16} \\
 			\mb{h_{15}} &=& h_3 (3) + h_7 (7) + h_{11} (3) = \redb{13} \\	
			\invisibleRow \invisibleRow
		\end{eqnarray*}
		\end{scriptsize}

	\end{column}
	\begin{column}{5cm}
		\begin{center}
			\begin{tabular}{|r|lll|l|}
			\hline
			{\bf W} & {\bf A} & {\bf B} & {\bf C} & {\bf Total} \\
			\hline
			{\bf I}   & 20 &  50 &  10 & \textbf{80} \\
			{\bf II}  & 8  &  19 &  22 & \textbf{49} \\
			{\bf III} & 17 &  32 &  12 & \textbf{61} \\
			\hline
			{\bf Total} & \redb{45} & \redb{101} & \redb{44}
			& $\mb{y_{16}}$
			\\
			\hline
			\end{tabular}
		\end{center}

		\begin{scriptsize}
		\begin{eqnarray*}
			\mb{y_4} 	 &=& y_1 (20) + y_2 (50) + y_3 (10) = \textbf{80} \\
			\mb{y_8} 	 &=& y_5 (8) + y_6 (19) + y_7 (22) = \textbf{49} \\
			\mb{y_{12}}  &=& y_9 (17) + y_{10} (32) + y_{11} (12) = \textbf{61} \\	
			\mb{y_{13}} &=& y_1 (20) + y_5 (8) + y_9 (17) = \redb{45} \\
 			\mb{y_{14}} &=& y_2 (50) + y_6 (19) + y_{32} (5) = \redb{101} \\
 			\mb{y_{15}} &=& y_3 (10) + y_7 (22) + y_{12} (3) = \redb{44} \\	
			\invisibleRow \invisibleRow
		\end{eqnarray*}
		\end{scriptsize}
		\end{column}
	\end{columns}
\end{frame}   
 
%%% Randsummen (3) %%% 
\begin{frame}\frametitle{From microdata to tables}
	\begin{columns}
	\begin{column}{5cm}
		\begin{center}
			\begin{tabular}{|r|lll|l|}
			\hline
			{\bf H} & {\bf A} & {\bf B} & {\bf C} & {\bf Total} \\
			\hline
			{\bf I} 	& 4 & 6 & 3 & \crwb{13} \\
			{\bf II} 	& 2 & 5 & 7 & \crwb{14} \\
			{\bf III}   & 4 & 5 & 3 & \crwb{12} \\
			\hline
			{\bf Total} & \crwb{10} & \crwb{16} & \crwb{13}
			& $\cbw{\mb{h_{16}}}$
			\\
			\hline
			\end{tabular}
		\end{center}

		\begin{scriptsize}
		\begin{eqnarray*}
			\mb{h_4} 	 &=& h_1 (4) + h_2 (6) + h_3 (3) = \textbf{13} \\
			\mb{h_8} 	 &=& h_5 (2) + h_6 (5) + h_7 (7) = \textbf{14} \\
			\mb{h_{12}}  &=& h_9 (4) + h_{10} (5) + h_{11} (3) = \textbf{12} \\	
			\mb{h_{13}} &=& h_1 (4) + h_5 (2) + h_9 (4) = \redb{10} \\
 			\mb{h_{14}} &=& h_2 (6) + h_6 (5) + h_{10} (5) = \redb{16} \\
 			\mb{h_{15}} &=& h_3 (3) + h_7 (7) + h_{11} (3) = \redb{13} \\	
			\invisibleRow \invisibleRow
		\end{eqnarray*}
		\end{scriptsize}

	\end{column}
	\begin{column}{5cm}
		\begin{center}
			\begin{tabular}{|r|lll|l|}
			\hline
			{\bf W} & {\bf A} & {\bf B} & {\bf C} & {\bf Total} \\
			\hline
			{\bf I}   & 20 &  50 &  10 & \crwb{80} \\
			{\bf II}  & 8  &  19 &  22 & \crwb{49} \\
			{\bf III} & 17 &  32 &  12 & \crwb{61} \\
			\hline
			{\bf Total} & \crwb{45} & \crwb{101} & \crwb{44}
			& $\cbw{\mb{y_{16}}}$
			\\
			\hline
			\end{tabular}
		\end{center}

		\begin{scriptsize}
		\begin{eqnarray*}
			\mb{y_4} 	 &=& y_1 (20) + y_2 (50) + y_3 (10) = \textbf{80} \\
			\mb{y_8} 	 &=& y_5 (8) + y_6 (19) + y_7 (22) = \textbf{49} \\
			\mb{y_{12}}  &=& y_9 (17) + y_{10} (32) + y_{11} (12) = \textbf{61} \\	
			\mb{y_{13}} &=& y_1 (20) + y_5 (8) + y_9 (17) = \redb{45} \\
 			\mb{y_{14}} &=& y_2 (50) + y_6 (19) + y_{32} (5) = \redb{101} \\
 			\mb{y_{15}} &=& y_3 (10) + y_7 (22) + y_{12} (3) = \redb{44} \\	
			\invisibleRow \invisibleRow
		\end{eqnarray*}
		\end{scriptsize}
		\end{column}
	\end{columns}
\end{frame}    
 
\begin{frame}\frametitle{From microdata to tables}
	\begin{columns}
	\begin{column}{5cm}
		\begin{center}
			\begin{tabular}{|r|lll|l|}
			\hline
			{\bf H} & {\bf A} & {\bf B} & {\bf C} & {\bf Total} \\
			\hline
			{\bf I} 	& 4 & 6 & 3 & \crwb{13} \\
			{\bf II} 	& 2 & 5 & 7 & \crwb{14} \\
			{\bf III}   & 4 & 5 & 3 & \crwb{12} \\
			\hline
			{\bf Total} & \crwb{10} & \crwb{16} & \crwb{13}
			& $\cbw{\mb{h_{16}}}$
			\\
			\hline
			\end{tabular}
		\end{center}

		\begin{scriptsize}
		\begin{eqnarray*}
			\mb{h_4} 	 &=& h_1 (4) + h_2 (6) + h_3 (3) = \textbf{13} \\
			\mb{h_8} 	 &=& h_5 (2) + h_6 (5) + h_7 (7) = \textbf{14} \\
			\mb{h_{12}}  &=& h_9 (4) + h_{10} (5) + h_{11} (3) = \textbf{12} \\	
			\mb{h_{13}} &=& h_1 (4) + h_5 (2) + h_9 (4) = \textbf{10} \\
 			\mb{h_{14}} &=& h_2 (6) + h_6 (5) + h_{10} (5) = \textbf{16} \\
 			\mb{h_{15}} &=& h_3 (3) + h_7 (7) + h_{11} (3) = \textbf{13} \\	
 			\mb{h_{16}} &=& \mb{h_4} (13) + \mb{h_8} (14) + \mb{h_{12}} (12) = \redb{39}
 			\\ \mb{h_{16}} &=& \mb{h_{13}} (10) + \mb{h_{14}} (16) + \mb{h_{15}} (13) =
 			\redb{39}\\
		\end{eqnarray*}
		\end{scriptsize}

	\end{column}
	\begin{column}{5cm}
		\begin{center}
			\begin{tabular}{|r|lll|l|}
			\hline
			{\bf W} & {\bf A} & {\bf B} & {\bf C} & {\bf Total} \\
			\hline
			{\bf I}   & 20 &  50 &  10 & \crwb{80} \\
			{\bf II}  & 8  &  19 &  22 & \crwb{49} \\
			{\bf III} & 17 &  32 &  12 & \crwb{61} \\
			\hline
			{\bf Total} & \crwb{45} & \crwb{101} & \crwb{44}
			& $\cbw{\mb{y_{16}}}$
			\\
			\hline
			\end{tabular}
		\end{center}

		\begin{scriptsize}
		\begin{eqnarray*}
			\mb{y_4} 	 &=& y_1 (20) + y_2 (50) + y_3 (10) = \textbf{80} \\
			\mb{y_8} 	 &=& y_5 (8) + y_6 (19) + y_7 (22) = \textbf{49} \\
			\mb{y_{12}}  &=& y_9 (17) + y_{10} (32) + y_{11} (12) = \textbf{61} \\	
			\mb{y_{13}} &=& y_1 (20) + y_5 (8) + y_9 (17) = \textbf{45} \\
 			\mb{y_{14}} &=& y_2 (50) + y_6 (19) + y_{32} (5) = \textbf{101} \\
 			\mb{y_{15}} &=& y_3 (10) + y_7 (22) + y_{12} (3) = \textbf{44} \\	
			\mb{y_{16}} &=& \mb{y_4} (80) + \mb{y_8} (49) + \mb{y_{12}} (61) = \redb{190}
			\\ \mb{y_{16}} &=& \mb{y_{13}} (45) + \mb{y_{14}} (101) + \mb{y_{15}} (44) =
			\redb{190} \\
		\end{eqnarray*}
		\end{scriptsize}
		\end{column}
	\end{columns}
\end{frame}  
 
\begin{frame}\frametitle{From microdata to tables}
	\begin{columns}
	\begin{column}{5cm}
		\begin{center}
			\begin{tabular}{|r|lll|l|}
			\hline
			{\bf H} & {\bf A} & {\bf B} & {\bf C} & {\bf Total} \\ 
			\hline
			{\bf I} 	& 4 & 6 & 3 & {\bf 13} \\ 
			{\bf II} 	& 2 & 5 & 7 & {\bf 14}\\ 
			{\bf III} & 4 & 5 & 3 & {\bf 12}\\ 
			\hline
			{\bf Total} & {\bf 10} & {\bf 16} & {\bf 13} & \redb{39} \\
			\hline
			\end{tabular}
		\end{center}	
		
		\begin{scriptsize}
		\begin{eqnarray*}		
			\mb{h_4} 	 &=& h_1 (4) + h_2 (6) + h_3 (3) = \textbf{13} \\
			\mb{h_8} 	 &=& h_5 (2) + h_6 (5) + h_7 (7) = \textbf{14} \\
			\mb{h_{12}} &=& h_9 (4) + h_{10} (5) + h_{11} (3) = \textbf{12} \\			
			\mb{h_{13}} &=& h_1 (4) + h_5 (2) + h_9 (4) = \textbf{10} \\
			\mb{h_{14}} &=& h_2 (6) + h_6 (5) + h_{10} (5) = \textbf{16} \\
			\mb{h_{15}} &=& h_3 (3) + h_7 (7) + h_{11} (3) = \textbf{13} \\				
			\mb{h_{16}} &=& \mb{h_4} (13) + \mb{h_8} (14) + \mb{h_{12}} (12) =
			\redb{39} \\ 
			\mb{h_{16}} &=& \mb{h_{13}} (10) + \mb{h_{14}} (16) + \mb{h_{15}} (13) =
			\redb{39}
		\end{eqnarray*}					
		\end{scriptsize}
			
	\end{column}
	\begin{column}{5cm}
		\begin{center}
			\begin{tabular}{|r|lll|l|}
			\hline
			{\bf W} & {\bf A} & {\bf B} & {\bf C} & {\bf Total} \\ 
			\hline
			{\bf I} 	& 20 & 50 & 10 & {\bf 80} \\ 
			{\bf II} 	& 8 & 19 & 22 & {\bf 49} \\ 
			{\bf III} & 17 & 32 & 12 & {\bf 61} \\ 
			\hline
			{\bf Total} & {\bf 45} & {\bf 101} & {\bf 44} & \redb{190} \\
			\hline
			\end{tabular}		
		\end{center}			

		\begin{scriptsize}
		\begin{eqnarray*}		
			\mb{y_4} 	&=& y_1 (20) + y_2 (50) + y_3 (10) = \textbf{80} \\
			\mb{y_8} 	&=& y_5 (8)  + y_6 (19) + y_7 (22) = \textbf{49} \\
			\mb{y_{12}} &=& y_9 (17) + y_{10} (32) + y_{11} (12) = \textbf{61} \\			
			\mb{y_{13}} &=& y_1 (20) + y_5 (8) + y_9 (17) = \textbf{45} \\
			\mb{y_{14}} &=& y_2 (50) + y_6 (19) + y_{32} (5) = \textbf{101} \\
			\mb{y_{15}} &=& y_3 (10) + y_7 (22) + y_{12} (3) = \textbf{44} \\				
			\mb{y_{16}} &=& \mb{y_4} (80) + \mb{y_8} (49) + \mb{y_{12}} (61) =
			\redb{190} \\ 
			\mb{y_{16}} &=& \mb{y_{13}} (45) + \mb{y_{14}} (101) +
			\mb{y_{15}} (44) = \redb{190}
		\end{eqnarray*}					
		\end{scriptsize}
		\end{column}
	\end{columns}	
\end{frame} 
 
 
%%% Vollstaendige Tabellen %%%
\begin{frame}\frametitle{From microdata to tables}
	\begin{columns}
	\begin{column}{5cm}
		\begin{center}
			\begin{tabular}{|r|lll|l|}
			\hline
			{\bf H} & {\bf A} & {\bf B} & {\bf C} & {\bf Total} \\ 
			\hline
			{\bf I} 	& 4 & 6 & 3 & {\bf 13} \\ 
			{\bf II} 	& 2 & 5 & 7 & {\bf 14}\\ 
			{\bf III} & 4 & 5 & 3 & {\bf 12}\\ 
			\hline
			{\bf Total} & {\bf 10} & {\bf 16} & {\bf 13} & {\bf 39} \\
			\hline
			\end{tabular}
		\end{center}	
		
		\begin{scriptsize}
		\begin{eqnarray*}		
			\mb{h_4} 	 &=& h_1 (4) + h_2 (6) + h_3 (3) = \textbf{13} \\
			\mb{h_8} 	 &=& h_5 (2) + h_6 (5) + h_7 (7) = \textbf{14} \\
			\mb{h_{12}} &=& h_9 (4) + h_{10} (5) + h_{11} (3) = \textbf{12} \\			
			\mb{h_{13}} &=& h_1 (4) + h_5 (2) + h_9 (4) = \textbf{10} \\
			\mb{h_{14}} &=& h_2 (6) + h_6 (5) + h_{10} (5) = \textbf{16} \\
			\mb{h_{15}} &=& h_3 (3) + h_7 (7) + h_{11} (3) = \textbf{13} \\				
			\mb{h_{16}} &=& \mb{h_4} (13) + \mb{h_8} (14) + \mb{h_{12}} (12) =
			\textbf{39} \\ 
			\mb{h_{16}} &=& \mb{h_{13}} (10) + \mb{h_{14}} (16) + \mb{h_{15}} (13) =
			\textbf{39}
		\end{eqnarray*}					
		\end{scriptsize}
			
	\end{column}
	\begin{column}{5cm}
		\begin{center}
			\begin{tabular}{|r|lll|l|}
			\hline
			{\bf W} & {\bf A} & {\bf B} & {\bf C} & {\bf Total} \\ 
			\hline
			{\bf I} 	& 20 & 50 & 10 & {\bf 80} \\ 
			{\bf II} 	& 8 & 19 & 22 & {\bf 49} \\ 
			{\bf III} & 17 & 32 & 12 & {\bf 61} \\ 
			\hline
			{\bf Total} & {\bf 45} & {\bf 101} & {\bf 44} & {\bf 190} \\
			\hline
			\end{tabular}		
		\end{center}			

		\begin{scriptsize}
		\begin{eqnarray*}		
			\mb{y_4} 	&=& y_1 (20) + y_2 (50) + y_3 (10) = \textbf{80} \\
			\mb{y_8} 	&=& y_5 (8)  + y_6 (19) + y_7 (22) = \textbf{49} \\
			\mb{y_{12}} &=& y_9 (17) + y_{10} (32) + y_{11} (12) = \textbf{61} \\			
			\mb{y_{13}} &=& y_1 (20) + y_5 (8) + y_9 (17) = \textbf{45} \\
			\mb{y_{14}} &=& y_2 (50) + y_6 (19) + y_{32} (5) = \textbf{101} \\
			\mb{y_{15}} &=& y_3 (10) + y_7 (22) + y_{12} (3) = \textbf{44} \\				
			\mb{y_{16}} &=& \mb{y_4} (80) + \mb{y_8} (49) + \mb{y_{12}} (61) =
			\textbf{190} \\ 
			\mb{y_{16}} &=& \mb{y_{13}} (45) + \mb{y_{14}} (101) +
			\mb{y_{15}} (44) = \textbf{190}
		\end{eqnarray*}					
		\end{scriptsize}
		\end{column}
	\end{columns}	
\end{frame}

%%% Formalisierung
\begin{frame}\frametitle{Formalization}
	\begin{itemize}
		\item {\bf Gerneralisation:} A (mulit-dimensional, hierarchical) table is given by: \pause
		\begin{itemize}
			\item a data vector: $a = [a_1,\ldots, a_n]$ \pause
			\item linear constraints of the form: $M a = b$ \pause
			%\item obere und untere Grenzen f?r jeden Tabellenwert, die einem \pause
			%Angreifer bekannt sind (zb. nicht-Negativit?t): $lb_i \leq a_i \leq ub_i$
		\end{itemize}		
		\item {\bf Remarks:}
		\begin{itemize}
			\item M is a matrix with $M_{ij} \in \{-1,0,1\}$ and $b$ is a vector containing $0$ \pause
			\begin{tiny}
			\[ M = \left( \begin{array}{cccccccccccccccc}
			1 & 1 & 1 & -1 & 0 & 0 & 0 & 0 & 0 & 0 & 0 & 0 & 0 & 0 & 0 & 0 \\
			0 & 0 & 0 & 0 & 1 & 1 & 1 & -1 & 0 & 0 & 0 & 0 & 0 & 0 & 0 & 0 \\
			0 & 0 & 0 & 0 & 0 & 0 & 0 & 0 & 1 & 1 & 1 & -1 & 0 & 0 & 0 & 0 \\
			0 & 0 & 0 & 0 & 0 & 0 & 0 & 0 & 0 & 0 & 0 & 0 & 1 & 1 & 1 & -1 \\			
			1 & 0 & 0 & 0 & 1 & 0 & 0 & 0 & 1 & 0 & 0 & 0 & -1 & 0 & 0 & 0 \\
			0 & 1 & 0 & 0 & 0 & 1 & 0 & 0 & 0 & 1 & 0 & 0 & 0 & -1 & 0 & 0 \\
			0 & 0 & 1 & 0 & 0 & 0 & 1 & 0 & 0 & 0 & 1 & 0 & 0 & 0 & -1 & 0 \\
			0 & 0 & 0 & 1 & 0 & 0 & 0 & 1 & 0 & 0 & 0 & 1 & 0 & 0 & 0 & -1 \end{array} \right)\] 
			\end{tiny} \pause
			\item Each row of $M~a=b$ referes to a constraint of a row or column sum. \pause
			\item the cells of a table are determined by its  (column) index: $i=1,\ldots,n$
		\end{itemize}			
	\end{itemize}
\end{frame}
 %%%%
%%%%%%%%%%%%%%%%%%%%%%%%%%%%%%%%%%%%%%


%%%%%%%%%%%%%%%%%%%%%%%%%%%%%%%%%%%%%%%%%%
%%%%% Methoden der Primaersperrungen %%%%%
\section{Identification of unsafe cells}
\begin{frame}\frametitle{Primary suppressions}
	\begin{itemize}
		\item Some rules for determining if a cell is ''unsafe'': \pause
		\begin{itemize}
			\item {\bf Frequency rule:} \\ Count of observations contributing to a cell. Unsafe if the count $<$ threshold $k$ (mostly $k$ is 3 or 4) \pause
			\item {\bf (n,k)-dominance rule:} \\ A cell must be protected if the total of $n$ largest of contributers to a cell is larger than $k\%$ of the total cell value.   \pause
			\item {\bf p-\% rule:} \\ total minus the sum of the two largest contributers is smaller than  $p \%$ of the largest contributor. (the largest contributor is again dominant) %\pause
		\end{itemize}
		%\item Beispiele!
	\end{itemize}
	
The later two rules are similar (but not the same). We will not go into details here.
\end{frame}
% 
% \begin{frame}\frametitle{Primary suppression}
% 	\begin{itemize}
% 		\item {\bf Zusammenhang} zwischen (n,k)-Dominanzregel und p-\%-Regel:
% 		\begin{itemize}
% 			\item Nach beiden Dominanzregeln m?ssen Zellwerte gesch?tzt werden, wenn ein oberer Sch?tztwert f?r den Wert des gr??ten Beitragenden konstruiert werden kann, der den wahren Wert {\bf nicht genug} ?bersch?tzt.  \pause
% 			\item Nach der $p-\%$-Regel wird das ''nicht genug'' als Rate ($p\%$) am wahren Wert der gr??ten beitragenden Einheit gemessen. \pause
% 			\item nach der (n,k)-Dominanzregel wird das ''nicht genug'' als Rate ($100-k$)\% am Zellwert gemessen. \pause
% 			\item {\bf Ausserdem gilt:}
% 			\begin{itemize}
% 				\item jeder Zellwert, der nach der (2,k)-Regel als ''sicher'' gilt, ist auch ''sicher'' nach der $p\%$-Regel.\pause
% 				\item nicht jeder Zellwert, der nach der $p\%$-Regel als ''sicher'' gilt ist auch ''sicher'' nach der (2,k)-Dominanzregel.\pause
% 					\item Es gilt f?r den Zusammenhang zwischen p\%-Regel und (2,k)-Regel: $p = 100 \cdot \dfrac{100-k}{k}$
% 			\end{itemize}
% 		\end{itemize}
% 	\end{itemize}
% \end{frame}

% \begin{frame}\frametitle{Prim?rsperrungen}
% 	\begin{itemize}
% 		\item {\bf Beispiel:} 5 Einheiten $F_1, F_2, F_3, F_4$ und $F_5$ tragen zu einer Zelle mit $uF_1=5000, uF_2=4900, uF_3=50, uF_4=30$ und $uF_5=20$ bei. Das Zelltotal ist daher $b_1+b_2+b_3+b_4+b_5=10000$. \pause
% 		\begin{itemize}
% 			\item {\bf Fallzahlregel:} \\ Die Anzahl der zur Zelle beitragenden Einheiten ist $5$. Die Zelle ist bei Anwendung der Fallzahlregel mit $n \leq 5$ {\bf nicht} sch?tzenswert. \pause
% 			\item {\bf (1,90)-Dominanzregel:} \\ Der Gesamtwert der $n=1$ gr??ten Beitragenden ?berschreitet $90 \%$ des Zellwertes nicht: $5000 < (90/100)*10000$. Die Zelle ist daher bei Anwendung der (1,90)-Dominanzregel ''sicher''. \pause \\
% 			\item {\bf (2,90)-Dominanzregel:} \\ Der Gesamtwert der $n=2$ gr??ten Beitragenden ?berschreitet $90 \%$ des Zellwertes deutlich: $5000+4900 > (90/100)*10000$. Die Zelle muss daher bei Anwendung der (2,90)-Dominanzregel als unsicher markiert werden. \pause
% 		\end{itemize}
% 	\end{itemize}
% \end{frame}

% \begin{frame}\frametitle{Prim?rsperrungen}
% 	\begin{itemize}
% 		\item {\bf Beispiel:} 5 Einheiten $F_1, F_2, F_3, F_4$ und $F_5$ tragen zu einer Zelle mit $uF_1=5000, uF_2=4900, uF_3=50, uF_4=30$ und $uF_5=20$ bei. Das Zelltotal ist daher $b_1+b_2+b_3+b_4+b_5=10000$. \pause
% 		\begin{itemize}
% 			\item {\bf Fallzahlregel:} \\ Die Anzahl der zur Zelle beitragenden Einheiten ist $5$. Die Zelle ist bei Anwendung der Fallzahlregel mit $n \leq 5$ {\bf nicht} sch?tzenswert. \pause
% 			\item {\bf (1,90)-Dominanzregel:} \\ Der Gesamtwert der $n=1$ gr??ten Beitragenden ?berschreitet $90 \%$ des Zellwertes nicht: $5000 < (90/100)*10000$. Die Zelle ist daher bei Anwendung der (1,90)-Dominanzregel ''sicher''. \pause \\
% 			{\bf Aber:} Der zweitgr??te Beitragende kann einen oberen Grenzwert f?r den Beitrag des gr??ten Beitragenden sch?tzen: $\hat{x_1} = 10000 - uF_2 (4900) = 5100$ und ?bersch?tzt damit den wahren Wert um nur 2\%!
% 		\end{itemize}
% 	\end{itemize}
% \end{frame}


% \begin{frame}\frametitle{Prim?rsperrungen (Fortsetzung)}
% 	\begin{itemize}
% 		\item {\bf Beispiel:} 5 Einheiten $F_1, F_2, F_3, F_4$ und $F_5$ tragen zu einer Zelle mit $uF_1=5000, uF_2=4900, uF_3=50, uF_4=30$ und $uF_5=20$ bei. Das Zelltotal ist daher $uF_1+uF_2+uF_3+uF_4+uF_5=10000$
% 		\begin{itemize}
% 			\item {\bf (2,90)-Dominanzregel:} \\ Der Gesamtwert der $n=2$ gr??ten Beitragenden ?berschreitet $90 \%$ des Zellwertes deutlich: $5000+4900 > (90/100)*10000$. Die Zelle muss daher bei Anwendung der (2,90)-Dominanzregel als unsicher markiert werden. \pause
% 			\item {\bf p-\% Regel:} \\ Wie die (2,k)-Dominanzregel wird auch bei der p-\% Regel Information ?ber die zwei gr??ten beitragenden Einheiten verwendet. Da
% 			\[ 1000 - uF_1 - uF_2 = 100 < \frac{p}{100} \cdot uF_1; \quad \forall p \geq 2 \]
% 			gilt die Zelle nach der p-\% Regel als unsicher f?r alle $p \geq 2$.
% 		\end{itemize}
% 	\end{itemize}
% \end{frame}

% \begin{frame}\frametitle{Prim?rsperrungen (Fortsetzung)}
% 	\begin{itemize}
% 		\item {\bf Zusammenhang} zwischen (n,k)-Dominanzregel und p-\%-Regel:
% 		\begin{itemize}
% 			\item Nach beiden Dominanzregeln m?ssen Zellwerte gesch?tzt werden, wenn ein oberer Sch?tztwert f?r den Wert des gr??ten Beitragenden konstruiert werden kann, der den wahren Wert {\bf nicht genug} ?bersch?tzt.
% 			\item Nach der $p-\%$-Regel wird das ''nicht genug'' als Rate ($p\%$) am wahren Wert der gr??ten beitragenden Einheit gemessen. \pause
% 			\item nach der (n,k)-Dominanzregel wird das ''nicht genug'' als Rate ($100-k\%$) am Zellwert gemessen. \pause
% 			\item {\bf Ausserdem gilt:}
% 			\begin{itemize}
% 				\item jeder Zellwert, der nach der (2,k)-Regel als ''sicher'' gilt, ist auch ''sicher'' nach der $p\%$-Regel.\pause
% 				\item nicht jeder Zellwert, der nach der $p\%$-Regel als ''sicher'' gilt ist auch ''sicher'' nach der (2,k)-Dominanzregel.\pause
% 					\item Es gilt f?r den Zusammenhang zwischen p\%-Regel und (2,k)-Regel: $p = 100 \cdot \dfrac{100-k}{k}$
% 			\end{itemize}
% 		\end{itemize}
% 	\end{itemize}
% \end{frame}

% \begin{frame}\frametitle{Identifizierung sensibler Zellen (Beispiele)}
% 	\begin{itemize}
% 		\item {\bf Beispiel:} Der Gesamtwert $T$ einer Tabellenzelle sei 1000. \\
% 		Der Wert des gr??ten Einzelbeitrages sei $B_1=851$. \\
% 		Der zweitgr??te Einzelbeitrag sei $B_2=120$. \pause
% 		\item Ist die Zelle nach Anwendung der (1,90)-Dominanzregel geheimzuhalten? \pause \\
% 		$\longrightarrow$ Nein. $B_1 < \frac{90}{100} \cdot T$ $\Longleftrightarrow$ $851 < 900$ \pause
% 		\item Ist die Zelle nach Anwendung der (2,85)-Dominanzregel geheimzuhalten? \pause \\
% 		$\longrightarrow$ Ja. $B_1+B_2 > \frac{85}{100} \cdot T$ $\Longleftrightarrow$ $971 > 850$ \pause
% 		\item welchem p entspricht die (2,85)-Dominanzregel? \pause \\
% 		$\longrightarrow$ $p = 100 \cdot \frac{100-85}{85} = \approx 17.6$
% 	\end{itemize}
% \end{frame}

\begin{frame}\frametitle{Identification of unsafe cells (examples)}
	\begin{itemize}
		\item {\bf Example:} The total $T$ of one cell in a table is 1000. \\
		The value of the larges contributor is $B_1=500$. \\
		The value of the second largest contributor is $B_2=400$. \pause
		\item Is the cell safe for the (2,80)-dominance rule? \pause \\
		$\longrightarrow$ Yes. $B_1+B_2 > \frac{80}{100} \cdot T$ $\Longleftrightarrow$ $900 > 800$ \pause
		% \item which $p$ equals the (2,80)-dominance rule? \pause \\
		% $\longrightarrow$ $p = 100 \cdot \frac{100-80}{80} = 25$ \pause
		\item let p=25. Is the cell unsafe when we apply the 25\%-rule? \pause \\
		$\longrightarrow$ Yes. $T-B_1-B_2 < \frac{25}{100} \cdot B_1$ $\Longleftrightarrow$ $100 < 125$ \pause
	\end{itemize}
	%\vspace{1cm}
	%$\longrightarrow$ {\bf ?bungen im TGUI}
\end{frame}


% \begin{frame}\frametitle{Identifizierung sensibler Zellen (Beispiele)}
% 	\begin{itemize}
% 		\item {\bf Beispiel:} Der Gesamtwert des Tabellenfeldes betrage 1000. \\Der Wert des gr??ten Einzelbeitrages sei 851. \\Der zweitgr??te Einzelbeitrag sei 120.
% 		\item Ist die Zelle nach Anwendung der (1,90)-Dominanzregel geheimzuhalten? \pause \\$\longrightarrow$ Nein: $851 <  \frac{90}{100}  \cdot 1000$. \pause
% 		\item Ist die Zelle nach Anwendung der (2,85)-Dominanzregel geheimzuhalten? \pause \\$\longrightarrow$ Ja. $851 + 120 >  \frac{85}{100}  \cdot 1000$
% 		\item welchem p entspricht die (2,85)-Dominanzregel? \pause \\$\longrightarrow$ $p = 100 \cdot \frac{100-85}{85} = \approx 17.6$ 	\end{itemize}
% \end{frame}

% \begin{frame}\frametitle{Identifizierung sensibler Zellen (Beispiele)}
% 	\begin{itemize}
% 		\item {\bf Beispiel:} Der Gesamtwert des Tabellenfeldes betrage 1000. \\Der Wert des gr??ten Einzelbeitrages sei 500. \\Der zweitgr??te Einzelbeitrag sei 400.
% 		\item Ist die Zelle nach Anwendung der (2,85)-Dominanzregel geheimzuhalten? \pause \\$\longrightarrow$ Ja. $500 + 400 >  \frac{85}{100} \cdot 1000$
% 		\item sei p=17.6. Ist die Zelle bei Anwendung der 17.6\%-Regel zu sch?tzen? \pause \\$\longrightarrow$ Nein. $1000-500-400 > \frac{17.6}{100} \cdot 500$.
% 	\end{itemize}
% \end{frame}%%%%%%%%%%
%%%%%%%%%%%%%%%%%%%%%%%%%%%%%%%%%%%%%%%%%%

%%%%%%%%%%%%%%%%%%%%%%%%%%%%%%%%%%%%%%%%%%%%%%%%%%%%
%%%%% Methoden zum Schutz von sensiblen Zellen %%%%%
\section{Methods}
\begin{frame}\frametitle{Protection of unsafe cells}
	\begin{itemize}
		\item {\bf Repitition:} A (multi-dimensional, hierarchical) table is given by: \pause
		\begin{itemize}
			\item a data vector: $a = [a_1,\ldots, a_n]$ \pause
			\item linear contraints of the form: $M y = b$ \pause
			\item upper and lower bounds for each cell value expressing the knowledge of an intruder: $lb_i \leq a_i \leq ub_i$	\pause
			\item A cell in a table is determined by its index: $i=1,\ldots,n$ \pause
		\end{itemize}
		\item {\bf Additionally:}
		\begin{itemize}
			\item given $p$ primary suppressions: $PS=\{i_1,\ldots,i_p\}$ \pause
		\end{itemize}
		\item {\bf Question:} How to protect primary suppressed cells?
		\end{itemize}
\end{frame}

\begin{frame}\frametitle{Protection of unsafe cells}
	\begin{itemize}
		\item {\bf Example:} \\
		\begin{center}
			\begin{tabular}{|r|lll|l|}
			\hline
			{\bf W} & {\bf A} & {\bf B} & {\bf C} & {\bf Total} \\
			\hline
			{\bf I} 	& 20 & 50 & 10 & {\bf 80} \\
			{\bf II} 	& 8 & 19 & \textcolor{red}{22} & {\bf 49} \\
			{\bf III} & 17 & 32 & 12 & {\bf 61} \\
			\hline
			{\bf Total} & {\bf 45} & {\bf 101} & {\bf 44} & {\bf 190} \\
			\hline
			\end{tabular}
		\end{center}		\pause
		\item Let cell $II/C ~ (PS=\{7\})$ be unsafe and to be protected
		\item Different possibilites to protect this cell, e.g.:
		\begin{itemize}
			\item \bf {cell suppression}
			\item \bf {rounding}
			\item \bf {reporting upper and lower bounds}
		\end{itemize}
	\end{itemize}
\end{frame}

\subsection{Cell suppression}
\begin{frame}\frametitle{Cell suppression}
	\begin{itemize}
		\item {\bf Example:} \\
		\begin{center}
			\begin{tabular}{|r|lll|l|}
			\hline
			{\bf W} & {\bf A} & {\bf B} & {\bf C} & {\bf Total} \\
			\hline
			{\bf I} 	& 20 & 50 & 10 & {\bf 80} \\
			{\bf II} 	& 8 & 19 & \textcolor{red}{NA} & {\bf 49} \\
			{\bf III}   & 17 & 32 & 12 & {\bf 61} \\
			\hline
			{\bf Total} & {\bf 45} & {\bf 101} & {\bf 44} & {\bf 190} \\
			\hline
			\end{tabular}
		\end{center}
		\item Most popular method
		\item {\bf However:} Because of the linear dependencies in tables, it is not enough to protect the unsafe cells only. \pause
		\item {\bf E.g.:} $44-10-12=22$ and $49-8-19=22$. \\ no protection for the primary suppressed cell. \pause
		\item $\longrightarrow$  {\bf secondary cell suppression}: suppressing additional cells.
	\end{itemize}
\end{frame}


\begin{frame}\frametitle{Cell suppression}
	\begin{itemize}
		\item {\bf Example:} suppression pattern \pause
		\begin{columns}
		\begin{column}{5cm}
			\begin{center}
				\begin{tabular}{|r|lll|l|}
				\hline
				{\bf W} & {\bf A} & {\bf B} & {\bf C} & {\bf Total} \\
				\hline
				{\bf I} 	& 20 & 50 & 10 & {\bf 80} \\
				{\bf II} 	& \textcolor{red}{S} & 19 & \textcolor{red}{NA} & {\bf 49} \\
				{\bf III} & \textcolor{red}{S} & 32 & \textcolor{red}{S} & {\bf 61} \\
				\hline
				{\bf Total} & {\bf 45} & {\bf 101} & {\bf 44} & {\bf 190} \\
				\hline
				\end{tabular}
			\end{center}
		\end{column}
		\pause
			\begin{column}{5cm}
			\begin{center}
				\begin{tabular}{|r|lll|l|}
				\hline
				{\bf W} & {\bf A} & {\bf B} & {\bf C} & {\bf Total} \\
				\hline
				{\bf I} 	& \textcolor{red}{S} & 50 & \textcolor{red}{S} & {\bf 80} \\
				{\bf II} 	& \textcolor{red}{S} & 19 & \textcolor{red}{NA} & {\bf 49} \\
				{\bf III} & 17 & 32 & 12 & {\bf 61} \\
				\hline
				{\bf Total} & {\bf 45} & {\bf 101} & {\bf 44} & {\bf 190} \\
				\hline
				\end{tabular}
			\end{center}
			\end{column}
		\end{columns}
		\pause
		\item When does a suppression pattern support enough protection to unsafe cells? \pause
		\item Is there a optimal suppression pattern? \pause
		\item {\bf Generally:} problem is NP-hard for hierarchical, multi-dimensional and linked tables
	\end{itemize}
\end{frame}

\begin{frame}\frametitle{Cell suppression}
	\begin{itemize}
		\item {\bf good news:} there exist algorithms for obtaining the optimal solution \pause
		\item {\bf bad news:} The optimal method is much too slow in practice.\pause
		\item Methods to find the optimal suppression pattern are based on {\bf linear optimization} \pause
		\begin{itemize}
			\item Aim: find a suppression pattern that, e.g.,  minimizes  the number of suppressed cells \textbf{and} guarantees protection of the unsafe cells.\pause
			\item Protection: A primary unsafe cell is protected, if the suppressed cell value cannot be estimated well enough, i.e. the attacker can only estimate an upper and lower bound of the cell value (attacker problem). This interval must be large enough. \pause
		\end{itemize}
	\end{itemize}
\end{frame}

\begin{frame}\frametitle{Cell suppression - Attacker  problem}
	\begin{itemize}
		\item {\bf Example:} Attacker problem \pause
		\item {\bf Attacker:} knows the suppression pattern $SUP = \{5,7,9,11\}$, the protected table,
		$M y = b; lb_i \leq y_i \leq ub_i ~ \forall i \in SUP; y_i=a_i ~ \forall i \notin SUP$
			\begin{scriptsize}
			\begin{center}
				\begin{tabular}{|r|lll|l|}
				\hline
				{\bf W} & {\bf A} & {\bf B} & {\bf C} & {\bf Total} \\
				\hline
				{\bf I} 	& 20 & 50 & 10 & {\bf 80} \\
				{\bf II} 	& \textcolor{red}{$y_5$} & 19 & \textcolor{red}{$y_7$} & {\bf 49} \\
				{\bf III} & \textcolor{red}{$y_9$} & 32 & \textcolor{red}{$y_{11}$} & {\bf 61} \\
				\hline
				{\bf Total} & {\bf 45} & {\bf 101} & {\bf 44} & {\bf 190} \\
				\hline
				\end{tabular}
			\end{center}
			\end{scriptsize}	\pause

		\item {\bf LP-problem:} $min/max ~ y_i ~ \forall i \in SUP$ unter contraints:\pause
			\begin{scriptsize}
			\begin{center}
				\begin{tabular}{|r|lll|l|}
				\hline
				{\bf W} & {\bf A} & {\bf B} & {\bf C} & {\bf Total} \\
				\hline
				{\bf I} 	& 20 & 50 & 10 & {\bf 80} \\
				{\bf II} 	& \textcolor{red}{[0:25]} & 19 & \textcolor{red}{[5:30]} & {\bf 49} \\
				{\bf III} & \textcolor{red}{[0:25]} & 32 & \textcolor{red}{[4:29]} & {\bf 61} \\
				\hline
				{\bf Total} & {\bf 45} & {\bf 101} & {\bf 44} & {\bf 190} \\
				\hline
				\end{tabular}
			\end{center}
			\end{scriptsize}	\pause
		\item the primary suppressed value $y_7$ is estimated by $[5:30]$.
	\end{itemize}
\end{frame}


\begin{frame}\frametitle{Cell suppression - attacker problem}
	\begin{itemize}
		\item {\bf Example:} Attackers Problem \pause
		\item {\bf Attacker:} has knowledge on the suppression pattern $SUP = \{1,3,5,7\}$, the protected table,
		$M y = b; lb_i \leq y_i \leq ub_i ~ \forall i \in SUP; y_i=a_i ~ \forall i \notin SUP$
			\begin{scriptsize}
		\begin{center}
			\begin{tabular}{|r|lll|l|}
			\hline
			{\bf W} & {\bf A} & {\bf B} & {\bf C} & {\bf Total} \\
			\hline
			{\bf I} 	& \textcolor{red}{$y_1$} & 50 & \textcolor{red}{$y_3$} & {\bf 80} \\
			{\bf II} 	& \textcolor{red}{$y_5$} & 19 & \textcolor{red}{$y_7$} & {\bf 49} \\
			{\bf III} & 17 & 32 & 12 & {\bf 61} \\
			\hline
			{\bf Total} & {\bf 45} & {\bf 101} & {\bf 44} & {\bf 190} \\
			\hline
			\end{tabular}
		\end{center}			\end{scriptsize}	\pause
		\item {\bf Lp-Problem:} $min/max ~ y_i ~ \forall i \in SUP$ under above given constraints:\pause
			\begin{scriptsize}
			\begin{center}
				\begin{tabular}{|r|lll|l|}
				\hline
				{\bf W} & {\bf A} & {\bf B} & {\bf C} & {\bf Total} \\
				\hline
				{\bf I} 	& \textcolor{red}{[0:28]} & 50 & \textcolor{red}{[2:30]} & {\bf 80} \\
				{\bf II} 	& \textcolor{red}{[0:28]} & 19 & \textcolor{red}{[2:30]} & {\bf 49} \\
				{\bf III} & 17 & 32 & 12 & {\bf 61} \\
				\hline
				{\bf Total} & {\bf 45} & {\bf 101} & {\bf 44} & {\bf 190} \\
				\hline
				\end{tabular}
			\end{center}				\end{scriptsize}	\pause
		\item the estimated primary suppressed cell value $y_7$ is  $[2:30]$.
	\end{itemize}
\end{frame}

\begin{frame}\frametitle{Cell suppression}
	\begin{itemize}
		\item {\bf Note:} Cell suppression is nothing else than a publication of intervals.\pause
		\item {\bf Protected?} When a cell is protected well enough? Use percentages of the upper and lower estimated cell value given the true value of the cell. \pause
		\item {\bf Example:} the attacker should not be able to estimate the primary suppressed cell value more precisely than $\pm 10\%$. \pause
		\item {\bf Information loss:} Cell suppression equals information loss. We want to find a suppression pattern that keeps the information loss as low as possible.\pause
		%\item {\bf Verlustfunktion:} $min \sum_{i=1}^n x_i * w_i$ mit $x_i \in \{0,1\}$ ($x_i=1$ wenn $a_i$ Teil des Unterdr?ckungsschemas ist) und $w_i$ ein Gewicht, z.B:\pause
		%\begin{itemize}
		%	\item $w_i = 1 \ldots$ Minimierung der Anzahl der Unterdr?ckungen
		%	\item $w_i = a_i \ldots$ Minimierung der unterdr?ckten Wertesumme
		%\end{itemize}	\pause
		\item We will show the \textbf{mathematical modell} for optimal cell suppression
		\end{itemize}
\end{frame}


\begin{frame}\frametitle{Cell suppression - model assumptions}
	\begin{itemize}
		\item Assumption: the intruder knows a lower and upper bound for each cell value $a_i$:
		\begin{scriptsize} \begin{eqnarray*}
			lb_i \leq a_i \leq ub_i ~ \forall i=1,\ldots,n
		\end{eqnarray*}	\end{scriptsize} \pause

		\vspace{-0.15cm}
		\item Relative outer bounds for each cell:
		\vspace{-0.15cm}
		\begin{scriptsize}	\begin{eqnarray*}
			LB_i &:=& a_i - lb_i \geq 0 ~ \forall i=1,\ldots,n \\
			UB_i &:=& ub_i - a_i \geq 0 ~ \forall i=1,\ldots,n
		\end{eqnarray*}	\end{scriptsize} \pause

		\vspace{-0.15cm}
		\item 
		For all sensible cells, define lower ($LPL_i$) and upper ($UPL_i$) protection levels, so that for the attackers intervals the following holds:
		\vspace{-0.15cm}
		\begin{scriptsize}
		\begin{eqnarray*}
			min(y_i) \leq a_i - LPL_i ~  \forall i \in PS \\
			max(y_i) \geq a_i + UPL_i ~  \forall i \in PS
		\end{eqnarray*}
		\end{scriptsize} \pause

		\vspace{-0.15cm}
		\item Lets define a binary variable $x_i, ~ i=1,\ldots,n$:
		\begin{scriptsize} \begin{eqnarray*}
			x_i = 0 ~  \forall i \notin SUP \\
			x_i = 1 ~  \forall i \in SUP
		\end{eqnarray*}	\end{scriptsize}
	\end{itemize}
\end{frame}


\begin{frame}\frametitle{Cell suppression - model assumptions (2)}
	\begin{itemize}
		\item For each cell $a_i$ we define a weight $w_i$, that is included in the objective function to be optimized:
		\vspace{-0.15cm}
		\begin{scriptsize} \begin{eqnarray*}
			w_i &=& a_i  \\
			w_i &=& 1  \\
			w_i &=& log(1+a_i)
	\end{eqnarray*}	\end{scriptsize} \pause

	\vspace{-0.3cm}
	\item The objective function to be optimized is given as:
	\vspace{-0.15cm}
		\begin{scriptsize} \begin{eqnarray*}
			min \sum_{i=1}^n w_i \cdot x_i
	\end{eqnarray*}	\end{scriptsize} \pause
	\vspace{-0.15cm}
	\item under these constraints:
	\vspace{-0.1cm}
			\begin{scriptsize}
			\begin{center}
				\begin{tabular}{cc}
				$M f = b$ & $M g = b$ \\
				$f_i \geq a_i - LB_i \cdot x_i ~ \forall i=1,\ldots,n$ &  $g_i \geq a_i - LB_i \cdot x_i \forall i=1,\ldots,n$\\
				$f_i \leq a_i + UB_i \cdot x_i ~ \forall i=1,\ldots,n$ &  $g_i \leq a_i + UB_i \cdot x_i \forall i=1,\ldots,n$ \\
				$f_i \leq a_i - LPL_i ~ \forall i \in PS$ & $g_i \geq a_i + UPL_i ~ \forall i \in PS$
				\end{tabular}
			\end{center}
			\end{scriptsize}
	\end{itemize}
\end{frame}


\begin{frame}\frametitle{Cell suppression - the model}
	\begin{itemize}
	\item Optimize: $min \sum_{i=1}^n w_i \cdot x_i$ under \pause
			\begin{scriptsize}
				\begin{eqnarray}
				M f = b & M g = b \label{consistency1}\\
				f_i \geq a_i - LB_i \cdot x_i ~ \forall i=1,\ldots,n & g_i \geq a_i - LB_i \cdot x_i \forall i=1,\ldots,n \label{consistency2}\\
				f_i \leq a_i + UB_i \cdot x_i ~ \forall i=1,\ldots,n & g_i \leq a_i + UB_i \cdot x_i \forall i=1,\ldots,n \label{consistency3}\\
				f_i \leq a_i - LPL_i ~ \forall i \in PS & g_i \geq a_i + UPL_i ~ \forall i \in PS	 \label{protection}
			\end{eqnarray}
		\end{scriptsize}	\pause
		\vspace{-0.2cm}
		\item We search for two possible tables $f=(f_1,\ldots,f_n)$	and $g=(g_1,\ldots,g_n)$.\pause
		\item The constraints (\ref{consistency1}, \ref{consistency2}, \ref{consistency3}) define that for $f$ and $g$ all linear dependencies holds and that:\pause
		\begin{scriptsize}
		\begin{eqnarray}
		f_i = g_i = a_i ~ \forall i \notin SUPP \\
		lb_i \leq f_i, g_i \leq ub_i ~ \forall i \in SUPP
		\end{eqnarray}
		\end{scriptsize} \pause
		\vspace{-0.2cm}
		\item The constraints (\ref{protection}) ensure the protection levels for all primary suppressed cells.
	\end{itemize}
\end{frame}

\begin{frame}\frametitle{Cell suppression - remarks on the model}
	\begin{itemize}
	\item The model result in a {\bf optimal suppression pattern} related to the objective function.
	\item But: in {\bf practise} {\bf it never works}, because the amount of utiltiy variables ($f_i, g_i, x_i)$ and the amount of contraints increases fastly. \pause
	\item Another formulation of the model allows to reduce the necessary variables using the duality principle.
	\pause
	% \item The model will be defined only through its binary variables $x_i$.\pause
	% \item Step by step new constraints are integrated in the model, such contraints that depend only on  $x_i$. \pause
	% \item $\rightarrow$ {\bf iterativ algorithm}, whereby several less complex linear problems must be solved.
	\item We will not go further into details, otherwise we need a lecture on linear mixed integer programming.
	\end{itemize}
\end{frame}

%%%%%%%%%%%%%
\begin{frame}\frametitle{Cell suppression in hierachical tables}
	\begin{itemize}
		\item Given the table: \\ \pause
		\begin{scriptsize}
		\begin{center}
			\begin{tabular}{|r|lll|l|}
			\hline
			{\bf } & {\bf R1} & {\bf R2} & {\bf R3} & {\bf Total} \\ \hline
			{\bf 55.1} & 20 & 50 & 10 & {\bf 80} \\
			{\bf 55.2} & 8 & 19 & 22 & {\bf 49} \\
			{\bf 55.3} & 17 & 32 & 12 & {\bf 61} \\ \hline
			\rowgblb{55}{45}{101}{44}{190} \\ \hline
			{\bf 56.11} & 9 & 28 & 5 & {\bf 42} \\
			{\bf 56.12} & 4 & 7 & 6 & {\bf 17} \\
			{\bf 56.13} & 27 & 15 & 9 & {\bf 51} \\ \hline
			\rowcolor{Gray}{\bf 56.1} & 40 & 50 & 20 & {\bf 110} \\ \hline
			{\bf 56.2} & 2 & 20 & 18 & {\bf 40} \\
			{\bf 56.3} & 20 & 30 & 25 & {\bf 75} \\ \hline
			\rowgblb{56}{62}{100}{53}{225} \\ \hline
			\rowbwb{Total}{107}{201}{97}{415} \\ \hline
			\end{tabular}
		\end{center}
		\end{scriptsize}
		\end{itemize}
\end{frame}


\begin{frame}\frametitle{Cell suppression in hierachical tables}
	\begin{itemize}
		\item Cells that needs protection are primary suppressed:\\
		\begin{scriptsize}
		\begin{center}
			\begin{tabular}{|r|lll|l|}
			\hline
			{\bf } & {\bf R1} & {\bf R2} & {\bf R3} & {\bf Total} \\ \hline
			{\bf 55.1} & 20 & 50 & 10 & {\bf 80} \\
			{\bf 55.2} & 8 & 19 & \textcolor{red}{NA} & {\bf 49} \\
			{\bf 55.3} & 17 & 32 & 12 & {\bf 61} \\ \hline
			\rowgblb{55}{45}{101}{44}{190} \\ \hline
			{\bf 56.11} & 9 & 28 & 5 & {\bf 42} \\
			{\bf 56.12} & \textcolor{red}{NA} & \textcolor{red}{NA} & 6 & {\bf \textcolor{red}{NA}} \\
			{\bf 56.13} & 27 & 15 & 9 & {\bf 51} \\ \hline
			\rowcolor{Gray}{\bf 56.1} & 40 & \textcolor{red}{NA} & 20 & {\bf 110} \\
			\hline
			{\bf 56.2} & \textcolor{red}{NA} & 20 & 18 & {\bf 40} \\
			{\bf 56.3} & 20 & 30 & 25 & {\bf 75} \\ \hline
			\rowgblb{56}{62}{100}{53}{225} \\ \hline
			\rowbwb{Total}{107}{201}{97}{415} \\ \hline
			\end{tabular}
		\end{center}
		\end{scriptsize}
		\end{itemize}
\end{frame}

\begin{frame}\frametitle{Cell suppression in hierachical tables}
	\begin{itemize}
		\item Task: find a secondary suppression pattern so that primary cells cannot be estimated well enough and with a minimal Amount of secondary suppressions:
		\begin{scriptsize}
		\begin{center}
			\begin{tabular}{|r|lll|l|}
			\hline
			{\bf } & {\bf R1} & {\bf R2} & {\bf R3} & {\bf Total} \\ \hline
			{\bf 55.1} & 20 & 50 & 10 & {\bf 80} \\
			{\bf 55.2} & 8 & 19 & \textcolor{red}{NA} & {\bf 49} \\
			{\bf 55.3} & 17 & 32 & 12 & {\bf 61} \\	\hline
			\rowgblb{55}{45}{101}{44}{190} \\ \hline
			{\bf 56.11} & 9 & 28 & 5 & {\bf 42} \\
			{\bf 56.12} & \textcolor{red}{NA} & \textcolor{red}{NA} & 6 & {\bf \textcolor{red}{NA}} \\
			{\bf 56.13} & 27 & 15 & 9 & {\bf 51} \\ \hline
			\rowcolor{Gray}{\bf 56.1} & 40 & \textcolor{red}{NA} & 20 & {\bf 110} \\
			\hline
			{\bf 56.2} & \textcolor{red}{NA} & 20 & 18 & {\bf 40} \\
			{\bf 56.3} & 20 & 30 & 25 & {\bf 75} \\ \hline
			\rowgblb{56}{62}{100}{53}{225} \\ \hline
			\rowbwb{Total}{107}{201}{97}{415} \\ \hline
			\end{tabular}
		\end{center}
		\end{scriptsize}
		\end{itemize}
\end{frame}

\begin{frame}\frametitle{Cell suppression in hierachical tables}
	\begin{itemize}
		\item Task: find a secondary suppression pattern so that primary cells cannot be estimated well enough and with a minimal Amount of secondary suppressions:
		\begin{scriptsize}
		\begin{center}
			\begin{tabular}{|r|lll|l|}
			\hline
			{\bf } & {\bf R1} & {\bf R2} & {\bf R3} & {\bf Total} \\ \hline
			{\bf 55.1} & 20 & 50 & \textcolor{red}{S} & \textcolor{red}{{\bf S}} \\
			{\bf 55.2} & 8 & 19 & \textcolor{red}{NA} & \textcolor{red}{{\bf S}} \\
			{\bf 55.3} & 17 & 32 & 12 & {\bf 61} \\ \hline
			\rowgblb{55}{45}{101}{44}{190} \\ \hline
			{\bf 56.11} & \textcolor{red}{S} & \textcolor{red}{S} & 5 & \textcolor{red}{{\bf S}} \\
			{\bf 56.12} & \textcolor{red}{NA} & \textcolor{red}{NA} & 6 &
			\textcolor{red}{{\bf S}} \\
			{\bf 56.13} & 27 & 15 & 9 & {\bf 51} \\ \hline
			\rowcolor{Gray}{\bf 56.1} & \textcolor{red}{S} & \textcolor{red}{NA} & 20 & \textcolor{red}{{\bf S}} \\ \hline
			{\bf 56.2} & \textcolor{red}{NA} & \textcolor{red}{S} & 18 & \textcolor{red}{{\bf S}} \\
			{\bf 56.3} & \textcolor{red}{S} & \textcolor{red}{S} & 25 & {\bf 75} \\ \hline
			\rowgblb{56}{62}{100}{53}{225} \\ \hline
			\rowbwb{Total}{107}{201}{97}{415} \\ \hline
			\end{tabular}
		\end{center}
		\end{scriptsize} \pause
		\item We suppressed 13 cells in addition to the primary suppressed ones. \pause
		\item The information loss from the seconardy suppressions is 485. \pause
		\item Is there a better suppression pattern?%$\longrightarrow$ {\bf TGUI}
		\end{itemize}
\end{frame}

\begin{frame}\frametitle{Cell suppression in hierachical tables}
	\begin{itemize}
		\item Solution: the optimal suppression pattern
		\begin{scriptsize}
		\begin{center}
			\begin{tabular}{|r|lll|l|}
			\hline
			{\bf } & {\bf R1} & {\bf R2} & {\bf R3} & {\bf Total} \\ \hline
			{\bf 55.1} & 20 & 50 & 10 & {\bf 80} \\
			{\bf 55.2} & \textcolor{red}{S} & 19 & \textcolor{red}{NA} & {\bf 49} \\
			{\bf 55.3} & \textcolor{red}{S} & 32 & \textcolor{red}{S} & {\bf 61} \\ \hline
			\rowgblb{55}{45}{101}{44}{190} \\ \hline
			{\bf 56.11} & \textcolor{red}{S} & 28 & 5 & \textcolor{red}{{\bf S}} \\
			{\bf 56.12} & \textcolor{red}{NA} & \textcolor{red}{NA} & 6 &
			\textcolor{red}{{\bf S}} \\
			{\bf 56.13} & 27 & 15 & 9 & {\bf 51} \\ \hline
			\rowcolor{Gray}{\bf 56.1} & \textcolor{red}{S} & \textcolor{red}{NA} & 20 & {\bf 110} \\ \hline
			{\bf 56.2} & \textcolor{red}{NA} & \textcolor{red}{S} & 18 & {\bf 40} \\
			{\bf 56.3} & 20 & 30 & 25 & {\bf 75} \\ \hline
			\rowgblb{56}{62}{100}{53}{225} \\ \hline
			\rowbwb{Total}{107}{201}{97}{415} \\ \hline
			\end{tabular}
		\end{center}
		\end{scriptsize} \pause
		\item We suppressed 7 cells in addition to the primary suppressed ones. \pause
		\item The information loss from the seconardy suppressions is 148. \pause
		\end{itemize}
\end{frame}


%%%%%%%%%%%%%%%%%%%%

% \begin{frame}\frametitle{Zellunterdr?ckung in hierarchischen Tabellen}
% 	\begin{itemize}
% 		\item Aufgabe: Auffinden eines g?ltigen Sperrmusters gegen exakte R?ckrechenbarkeit mit minimaler Anzahl von Sekund?rsperrungen:
% 		\begin{scriptsize}
% 		\begin{center}
% 			\begin{tabular}{|r|lll|l|}
% 			\hline
% 			{\bf } & {\bf R1} & {\bf R2} & {\bf R3} & {\bf Total} \\ \hline
% 			{\bf 55.1} & 20 & 50 & 10 & {\bf 80} \\
% 			{\bf 55.2} & 8 & 19 & \textcolor{red}{NA} & {\bf 49} \\
% 			{\bf 55.3} & 17 & 32 & 12 & {\bf 61} \\ \hline
% 			\rowgblb{55}{45}{101}{44}{190} \\ \hline
% 			{\bf 56.11} & 9 & 28 & 5 & {\bf 42} \\
% 			{\bf 56.12} & \textcolor{red}{NA} & 7 & 6 & {\bf \textcolor{red}{NA}} \\
% 			{\bf 56.13} & 27 & 15 & 9 & {\bf 51} \\ \hline
% 			\rowcolor{Gray}{\bf 56.1} & \textcolor{red}{S} & \textcolor{red}{NA} & 20 & {\bf 110} \\ \hline
% 			{\bf 56.2} & \textcolor{red}{NA} & \textcolor{red}{S} & 18 & {\bf 40} \\
% 			{\bf 56.3} & 20 & 30 & 25 & {\bf 75} \\ \hline
% 			\rowgblb{56}{62}{100}{53}{225} \\ \hline
% 			\rowbwb{Total}{107}{201}{97}{415} \\ \hline
% 			\end{tabular}
% 		\end{center}
% 		\end{scriptsize}
% 		\end{itemize}
% \end{frame}

% \begin{frame}\frametitle{Zellunterdr?ckung in hierarchischen Tabellen}
% 	\begin{itemize}
% 		\item Aufgabe: Auffinden eines g?ltigen Sperrmusters gegen exakte R?ckrechenbarkeit mit minimaler Anzahl von Sekund?rsperrungen:
% 		\begin{scriptsize}
% 		\begin{center}
% 			\begin{tabular}{|r|lll|l|}
% 			\hline
% 			{\bf } & {\bf R1} & {\bf R2} & {\bf R3} & {\bf Total} \\ \hline
% 			{\bf 55.1} & 20 & 50 & 10 & {\bf 80} \\
% 			{\bf 55.2} & 8 & 19 & \textcolor{red}{NA} & {\bf 49} \\
% 			{\bf 55.3} & 17 & 32 & 12 & {\bf 61} \\	\hline
% 			\rowgblb{55}{45}{101}{44}{190} \\ \hline
% 			{\bf 56.11} & 9 & 28 & 5 & {\bf 42} \\
% 			{\bf 56.12} & \textcolor{red}{NA} & 7 & 6 & {\bf \textcolor{red}{NA}} \\
% 			{\bf 56.13} & 27 & 15 & 9 & {\bf 51} \\ \hline
% 			\rowcolor{Gray}{\bf 56.1} & \textcolor{red}{S} & \textcolor{red}{NA} & 20 & {\bf 110} \\ \hline
% 			{\bf 56.2} & \textcolor{red}{NA} & \textcolor{red}{S} & 18 & {\bf 40} \\
% 			{\bf 56.3} & 20 & 30 & 25 & {\bf 75} \\ \hline
% 			\rowgblb{56}{62}{100}{53}{225} \\ \hline
% 			\rowbwb{Total}{107}{201}{97}{415} \\ \hline
% 			\end{tabular}
% 		\end{center}
% 		\end{scriptsize}
% 		\end{itemize}
% \end{frame}
%
% \begin{frame}\frametitle{Zellunterdr?ckung in hierarchischen Tabellen}
% 	\begin{itemize}
% 		\item Aufgabe: Auffinden eines g?ltigen Sperrmusters gegen exakte R?ckrechenbarkeit mit minimaler Anzahl von Sekund?rsperrungen:
% 		\begin{scriptsize}
% 		\begin{center}
% 			\begin{tabular}{|r|lll|l|}
% 			\hline
% 			{\bf } & {\bf R1} & {\bf R2} & {\bf R3} & {\bf Total} \\ \hline
% 			{\bf 55.1} & 20 & 50 & 10 & {\bf 80} \\
% 			{\bf 55.2} & 8 & 19 & \textcolor{red}{NA} & {\bf 49} \\
% 			{\bf 55.3} & 17 & 32 & 12 & {\bf 61} \\	\hline
% 			\rowgblb{55}{45}{101}{44}{190} \\ \hline
% 			{\bf 56.11} & 9 & 28 & 5 & {\bf 42} \\
% 			{\bf 56.12} & \textcolor{red}{NA} & \textcolor{red}{S} & 6 & {\bf \textcolor{red}{NA}} \\
% 			{\bf 56.13} & 27 & 15 & 9 & {\bf 51} \\ \hline
% 			\rowcolor{Gray}{\bf 56.1} & \textcolor{red}{S} & \textcolor{red}{NA} & 20 & {\bf \textcolor{red}{S}} \\ \hline
% 			{\bf 56.2} & \textcolor{red}{NA} & \textcolor{red}{S} & 18 & {\bf 40} \\
% 			{\bf 56.3} & 20 & 30 & 25 & {\bf 75} \\ \hline
% 			\rowgblb{56}{62}{100}{53}{225} \\ \hline
% 			\rowbwb{Total}{107}{201}{97}{415} \\ \hline
% 			\end{tabular}
% 		\end{center}
% 		\end{scriptsize}
% 		\end{itemize}
% \end{frame}
%
% \begin{frame}\frametitle{Zellunterdr?ckung in hierarchischen Tabellen}
% 	\begin{itemize}
% 		\item Aufgabe: Auffinden eines g?ltigen Sperrmusters gegen exakte R?ckrechenbarkeit mit minimaler Anzahl von Sekund?rsperrungen:
% 		\begin{scriptsize}
% 		\begin{center}
% 			\begin{tabular}{|r|lll|l|}
% 			\hline
% 			{\bf } & {\bf R1} & {\bf R2} & {\bf R3} & {\bf Total} \\ \hline
% 			{\bf 55.1} & 20 & 50 & 10 & {\bf 80} \\
% 			{\bf 55.2} & 8 & 19 & \textcolor{red}{NA} & {\bf 49} \\
% 			{\bf 55.3} & 17 & 32 & 12 & {\bf 61} \\	\hline
% 			\rowgblb{55}{45}{101}{44}{190} \\ \hline
% 			{\bf 56.11} & 9 & 28 & 5 & {\bf 42} \\
% 			{\bf 56.12} & \textcolor{red}{NA} & \textcolor{red}{S} & 6 & {\bf \textcolor{red}{NA}} \\
% 			{\bf 56.13} & 27 & 15 & 9 & {\bf 51} \\ \hline
% 			\rowcolor{Gray}{\bf 56.1} & \textcolor{red}{S} & \textcolor{red}{NA} & 20 & {\bf \textcolor{red}{S}} \\
% 			\hline
% 			{\bf 56.2} & \textcolor{red}{NA} & \textcolor{red}{S} & 18 & {\bf \textcolor{red}{S}} \\
% 			{\bf 56.3} & 20 & 30 & 25 & {\bf 75} \\ \hline
% 			\rowgblb{56}{62}{100}{53}{225} \\ \hline
% 			\rowbwb{Total}{107}{201}{97}{415} \\ \hline
% 			\end{tabular}
% 		\end{center}
% 		\end{scriptsize}
% 		\end{itemize}
% \end{frame}
%
% \begin{frame}\frametitle{Zellunterdr?ckung in hierarchischen Tabellen}
% 	\begin{itemize}
% 		\item Aufgabe: Auffinden eines g?ltigen Sperrmusters gegen exakte R?ckrechenbarkeit mit minimaler Anzahl von Sekund?rsperrungen:
% 		\begin{scriptsize}
% 		\begin{center}
% 			\begin{tabular}{|r|lll|l|}
% 			\hline
% 			{\bf } & {\bf R1} & {\bf R2} & {\bf R3} & {\bf Total} \\ \hline
% 			{\bf 55.1} & 20 & 50 & 10 & {\bf 80} \\
% 			{\bf 55.2} & \textcolor{red}{S} & 19 & \textcolor{red}{NA} & {\bf 49} \\
% 			{\bf 55.3} & \textcolor{red}{S} & 32 & \textcolor{red}{S} & {\bf 61} \\
% 			\hline
% 			\rowgblb{55}{45}{101}{44}{190} \\ \hline
% 			{\bf 56.11} & 9 & 28 & 5 & {\bf 42} \\
% 			{\bf 56.12} & \textcolor{red}{NA} & \textcolor{red}{S} & 6 & {\bf \textcolor{red}{NA}} \\
% 			{\bf 56.13} & 27 & 15 & 9 & {\bf 51} \\ \hline
% 			\rowcolor{Gray}{\bf 56.1} & \textcolor{red}{S} & \textcolor{red}{NA} & 20 & {\bf \textcolor{red}{S}} \\ \hline
% 			{\bf 56.2} & \textcolor{red}{NA} & \textcolor{red}{S} & 18 & {\bf \textcolor{red}{S}} \\
% 			{\bf 56.3} & 20 & 30 & 25 & {\bf 75} \\ \hline
% 			\rowgblb{56}{62}{100}{53}{225} \\ \hline
% 			\rowbwb{Total}{107}{201}{97}{415} \\ \hline
% 			\end{tabular}
% 		\end{center}
% 		\end{scriptsize} \pause
% 		\item Wir mussten insgesamt 8 zus?tzliche Zellen sperren. \pause
% 		\item Der Informationsverlust der Sekund?rsperrungen betr?gt 254.
% 		\end{itemize}
% \end{frame}

% \begin{frame}\frametitle{Zellunterdr?ckung in hierarchischen Tabellen}
% 	\begin{itemize}
% 		\item Gibt es alternative L?sungen?:\\
% 		\begin{scriptsize}
% 		\begin{center}
% 			\begin{tabular}{|r|lll|l|}
% 			\hline
% 			{\bf } & {\bf R1} & {\bf R2} & {\bf R3} & {\bf Total} \\ \hline
% 			{\bf 55.1} & 20 & 50 & 10 & {\bf 80} \\
% 			{\bf 55.2} & 8 & 19 & \textcolor{red}{NA} & {\bf 49} \\
% 			{\bf 55.3} & 17 & 32 & 12 & {\bf 61} \\	\hline
% 			\rowgblb{55}{45}{101}{44}{190} \\ \hline
% 			{\bf 56.11} & 9 & 28 & 5 & {\bf 42} \\
% 			{\bf 56.12} & \textcolor{red}{NA} & 7 & 6 & {\bf \textcolor{red}{NA}} \\
% 			{\bf 56.13} & 27 & 15 & 9 & {\bf 51} \\ \hline
% 			\rowcolor{Gray}{\bf 56.1} & 40 & \textcolor{red}{NA} & 20 & {\bf 110} \\
% 			\hline
% 			{\bf 56.2} & \textcolor{red}{NA} & 20 & 18 & {\bf 40} \\
% 			{\bf 56.3} & 20 & 30 & 25 & {\bf 75} \\ \hline
% 			\rowgblb{56}{62}{100}{53}{225} \\ \hline
% 			\rowbwb{Total}{107}{201}{97}{415} \\ \hline
% 			\end{tabular}
% 		\end{center}
% 		\end{scriptsize}
% 		\end{itemize}
% \end{frame}
%
% \begin{frame}\frametitle{Zellunterdr?ckung in hierarchischen Tabellen}
% 	\begin{itemize}
% 		\item Gibt es alternative L?sungen?:\\
% 		\begin{scriptsize}
% 		\begin{center}
% 			\begin{tabular}{|r|lll|l|}
% 			\hline
% 			{\bf } & {\bf R1} & {\bf R2} & {\bf R3} & {\bf Total} \\ \hline
% 			{\bf 55.1} & 20 & 50 & 10 & {\bf 80} \\
% 			{\bf 55.2} & 8 & 19 & \textcolor{red}{NA} & {\bf 49} \\
% 			{\bf 55.3} & 17 & 32 & 12 & {\bf 61} \\	\hline
% 			\rowgblb{55}{45}{101}{44}{190} \\ \hline
% 			{\bf 56.11} & 9 & 28 & 5 & {\bf 42} \\
% 			{\bf 56.12} & \textcolor{red}{NA} & 7 & 6 & {\bf \textcolor{red}{NA}} \\
% 			{\bf 56.13} & 27 & 15 & 9 & {\bf 51} \\ \hline
% 			\rowcolor{Gray}{\bf 56.1} & \textcolor{red}{S} & \textcolor{red}{NA} & 20 & {\bf 110} \\ \hline
% 			{\bf 56.2} & \textcolor{red}{NA} & \textcolor{red}{S} & 18 & {\bf 40} \\
% 			{\bf 56.3} & 20 & 30 & 25 & {\bf 75} \\ \hline
% 			\rowgblb{56}{62}{100}{53}{225} \\ \hline
% 			\rowbwb{Total}{107}{201}{97}{415} \\ \hline
% 			\end{tabular}
% 		\end{center}
% 		\end{scriptsize}
% 		\end{itemize}
% \end{frame}
%
% \begin{frame}\frametitle{Zellunterdr?ckung in hierarchischen Tabellen}
% 	\begin{itemize}
% 		\item Gibt es alternative L?sungen?:\\
% 		\begin{scriptsize}
% 		\begin{center}
% 			\begin{tabular}{|r|lll|l|}
% 			\hline
% 			{\bf } & {\bf R1} & {\bf R2} & {\bf R3} & {\bf Total} \\ \hline
% 			{\bf 55.1} & 20 & 50 & 10 & {\bf 80} \\
% 			{\bf 55.2} & 8 & 19 & \textcolor{red}{NA} & {\bf 49} \\
% 			{\bf 55.3} & 17 & 32 & 12 & {\bf 61} \\	\hline
% 			\rowgblb{55}{45}{101}{44}{190} \\ \hline
% 			{\bf 56.11} & \textcolor{red}{S} & 28 & 5 & {\bf \textcolor{red}{S}} \\
% 			{\bf 56.12} & \textcolor{red}{NA} & 7 & 6 & {\bf \textcolor{red}{NA}} \\
% 			{\bf 56.13} & 27 & 15 & 9 & {\bf 51} \\ \hline
% 			\rowcolor{Gray}{\bf 56.1} & \textcolor{red}{S} & \textcolor{red}{NA} & 20 & {\bf 110} \\ \hline
% 			{\bf 56.2} & \textcolor{red}{NA} & \textcolor{red}{S} & 18 & {\bf 40} \\
% 			{\bf 56.3} & 20 & 30 & 25 & {\bf 75} \\ \hline
% 			\rowgblb{56}{62}{100}{53}{225} \\ \hline
% 			\rowbwb{Total}{107}{201}{97}{415} \\ \hline
% 			\end{tabular}
% 		\end{center}
% 		\end{scriptsize}
% 		\end{itemize}
% \end{frame}
%
% \begin{frame}\frametitle{Zellunterdr?ckung in hierarchischen Tabellen}
% 	\begin{itemize}
% 		\item Gibt es alternative L?sungen?:\\
% 		\begin{scriptsize}
% 		\begin{center}
% 			\begin{tabular}{|r|lll|l|}
% 			\hline
% 			{\bf } & {\bf R1} & {\bf R2} & {\bf R3} & {\bf Total} \\ \hline
% 			{\bf 55.1} & 20 & 50 & 10 & {\bf 80} \\
% 			{\bf 55.2} & 8 & 19 & \textcolor{red}{NA} & {\bf 49} \\
% 			{\bf 55.3} & 17 & 32 & 12 & {\bf 61} \\ \hline
% 			\rowgblb{55}{45}{101}{44}{190} \\ \hline
% 			{\bf 56.11} & \textcolor{red}{S} & 28 & 5 & {\bf \textcolor{red}{S}} \\
% 			{\bf 56.12} & \textcolor{red}{NA} & \textcolor{red}{S} & 6 & {\bf \textcolor{red}{NA}} \\
% 			{\bf 56.13} & 27 & 15 & 9 & {\bf 51} \\ \hline
% 			\rowcolor{Gray}{\bf 56.1} & \textcolor{red}{S} & \textcolor{red}{NA} & 20 & {\bf 110} \\ \hline
% 			{\bf 56.2} & \textcolor{red}{NA} & \textcolor{red}{S} & 18 & {\bf 40} \\
% 			{\bf 56.3} & 20 & 30 & 25 & {\bf 75} \\ \hline
% 			\rowgblb{56}{62}{100}{53}{225} \\ \hline
% 			\rowbwb{Total}{107}{201}{97}{415} \\ \hline
% 			\end{tabular}
% 		\end{center}
% 		\end{scriptsize}
% 		\end{itemize}
% \end{frame}
%
% \begin{frame}\frametitle{Zellunterdr?ckung in hierarchischen Tabellen}
% 	\begin{itemize}
% 		\item Gibt es alternative L?sungen?:\\
% 		\begin{scriptsize}
% 		\begin{center}
% 			\begin{tabular}{|r|lll|l|}
% 			\hline
% 			{\bf } & {\bf R1} & {\bf R2} & {\bf R3} & {\bf Total} \\ \hline
% 			{\bf 55.1} & 20 & 50 & 10 & {\bf 80} \\
% 			{\bf 55.2} & \textcolor{red}{S} & 19 & \textcolor{red}{NA} & {\bf 49} \\
% 			{\bf 55.3} & \textcolor{red}{S} & 32 & \textcolor{red}{S} & {\bf 61} \\
% 			\hline
% 			\rowgblb{55}{45}{101}{44}{190} \\ \hline
% 			{\bf 56.11} & \textcolor{red}{S} & 28 & 5 & {\bf \textcolor{red}{S}} \\
% 			{\bf 56.12} & \textcolor{red}{NA} & \textcolor{red}{S} & 6 & {\bf \textcolor{red}{NA}} \\
% 			{\bf 56.13} & 27 & 15 & 9 & {\bf 51} \\ \hline
% 			\rowcolor{Gray}{\bf 56.1} & \textcolor{red}{S} & \textcolor{red}{NA} & 20 & {\bf 110} \\ \hline
% 			{\bf 56.2} & \textcolor{red}{NA} & \textcolor{red}{S} & 18 & {\bf 40} \\
% 			{\bf 56.3} & 20 & 30 & 25 & {\bf 75} \\ \hline
% 			\rowgblb{56}{62}{100}{53}{225} \\ \hline
% 			\rowbwb{Total}{107}{201}{97}{415} \\ \hline
% 			\end{tabular}
% 		\end{center}
% 		\end{scriptsize}
% 		\end{itemize}
% \end{frame}
%
% \begin{frame}\frametitle{Zellunterdr?ckung in hierarchischen Tabellen}
% 	\begin{itemize}
% 		\item Gibt es alternative L?sungen?:\\
% 		\begin{scriptsize}
% 		\begin{center}
% 			\begin{tabular}{|r|lll|l|}
% 			\hline
% 			{\bf } & {\bf R1} & {\bf R2} & {\bf R3} & {\bf Total} \\ \hline
% 			{\bf 55.1} & 20 & 50 & 10 & {\bf 80} \\
% 			{\bf 55.2} & \textcolor{red}{S} & 19 & \textcolor{red}{NA} & {\bf 49} \\
% 			{\bf 55.3} & \textcolor{red}{S} & 32 & \textcolor{red}{S} & {\bf 61} \\
% 			\hline
% 			\rowgblb{55}{45}{101}{44}{190} \\ \hline
% 			{\bf 56.11} & \textcolor{red}{S} & 28 & 5 & {\bf \textcolor{red}{S}} \\
% 			{\bf 56.12} & \textcolor{red}{NA} & \textcolor{red}{S} & 6 & {\bf \textcolor{red}{NA}} \\
% 			{\bf 56.13} & 27 & 15 & 9 & {\bf 51} \\ \hline
% 			\rowcolor{Gray}{\bf 56.1} & \textcolor{red}{S} & \textcolor{red}{NA} & 20 & {\bf 110} \\ \hline
% 			{\bf 56.2} & \textcolor{red}{NA} & \textcolor{red}{S} & 18 & {\bf 40} \\
% 			{\bf 56.3} & 20 & 30 & 25 & {\bf 75} \\ \hline
% 			\rowgblb{56}{62}{100}{53}{225} \\ \hline
% 			\rowbwb{Total}{107}{201}{97}{415} \\ \hline
% 			\end{tabular}
% 		\end{center}
% 		\end{scriptsize}
% 		\item Wir mussten insgesamt 8 zus?tzliche Zellen sperren. \pause
% 		\item Der Informationsverlust der Sekund?rsperrungen betr?gt allerdings nur 155.
% 		\end{itemize}
% \end{frame}


\begin{frame}\frametitle{Cell suppression - challenges}
	\begin{itemize}
		\item {\bf Hierarchical tables:} Variables (z.B NACE, NUTS,...) are usually hierarchical, which makes it difficult to model the linear dependencies ($M y = b$) in an automatized manner. \pause
		\item {\bf linked tables:} If certain cells can be found different tables. If such a cell is suppression by a secondary suppression, it must be checked if the cell cannot be estimated precisly in other tables.\pause
		\item {\bf Computational complexity:} the optimization problem is very hard to solve.\pause
		\item {\bf Heuristics} are necessary to find an almost optimal suppression pattern.\pause
		\begin{itemize}
			\item HITAS: Transformation of hierarchical tables to 2-dimensional tables and suppression of them given a certain schedule.\pause
		\end{itemize}
	\end{itemize}
\end{frame}

\subsection{Rounding}
\begin{frame}\frametitle{Rounding}
	\begin{itemize}
		\item {\bf Rounding} as an alternative to cell suppression. \pause
		\item {\bf Variants for rounding:}
		\pause
		\begin{itemize}
			\item rounding as usual
			\item random rounding
			\item controlled rounding \pause
		\end{itemize}
		\item All have in common a chosen {\bf rounding basis} (often 3 or 5).\pause
		\item {\bf rounding as usual} (rounding to the next multiple of the basis) is not the best approach \\ $\rightarrow$ we skip to apply this approach.
	\end{itemize}
\end{frame}

\begin{frame}\frametitle{Random rounding}
	\begin{itemize}
		\item {\bf Idea:} a cell value is round to a multiple of the basis, but ceiling or floor is decided randomly.\pause
		% \item {\bf Vielfache} der Basis werden nicht ver?ndert.\pause
		% \item {\bf Marinal totals} are werden ?blicherweise getrennt von den inneren Tabellenzellen behandelt.\pause
		% \item {\bf Wichtig:} unterschiedliche Gewichtungsschemata sind m?glich, jedoch soll keine Tendenz zum Auf- oder Abrunden durch das Gewichtungsschema implizit gegeben werden.\pause
		\item	{\bf Disadvantage:} hierarchical tables are no longer be additiv.
	\end{itemize}
\end{frame}


\begin{frame}\frametitle{Random rounding - example}
		\begin{scriptsize}
		\begin{center}
			\begin{tabular}{|r|lll|l|}
			\hline
			{\bf H} & {\bf A} & {\bf B} & {\bf C} & {\bf Total} \\
			\hline
			{\bf I} 	& 4 & 6 & 3 & {\bf 13} \\
			{\bf II} 	& 2 & 5 & 7 & {\bf 14}\\
			{\bf III} & 4 & 5 & 3 & {\bf 12} \\
			\hline
			{\bf Total} & {\bf 10} & {\bf 16} & {\bf 13}  & {\bf 39} \\
			\hline
			\end{tabular}
		\end{center}
		\end{scriptsize}\pause
		\begin{itemize}
		\item {\bf Basis:} Lets choose 3 and calculate the krest of the division through its basis:\pause
		\begin{scriptsize}
		\begin{center}
			\begin{tabular}{|r|lll|l|}
			\hline
			{\bf H} & {\bf A} & {\bf B} & {\bf C} & {\bf Total} \\
			\hline
			{\bf I} 	& 1 & 0 & 0 & {\bf 1} \\
			{\bf II} 	& 1 & 2 & 1 & {\bf 2}\\
			{\bf III} & 1 & 2 & 0 & {\bf 0} \\
			\hline
			{\bf Total} & {\bf 1} & {\bf 1} & {\bf 1}  & {\bf 0} \\
			\hline
			\end{tabular}
		\end{center}
		\end{scriptsize}\pause
		\item {\bf Weighting scheme:}\pause
		\begin{itemize}
			\item rest of division = 0: cell value stays untouched.\pause
			\item rest of division = 1: with probability $\frac{1}{3}$ we apply ceiling, with prob. $\frac{2}{3}$ floor.\pause
			\item rest of division = 2: with probability $\frac{2}{3}$ we apply ceiling, with prob. $\frac{1}{3}$ floor.
		\end{itemize}
	\end{itemize}
\end{frame}

\begin{frame}\frametitle{Random rounding - example}
	\begin{itemize}
		\item One possible solution:
		\begin{scriptsize}
		\begin{center}
			\begin{tabular}{|r|lll|l|}
			\hline
			{\bf H} & {\bf A} & {\bf B} & {\bf C} & {\bf Total} \\
			\hline
			{\bf I} 	& 6 & 6 & 3 & {\bf 15} \\
			{\bf II} 	& 3 & 3 & 6 & {\bf 12}\\
			{\bf III} & 3 & 6 & 3 & {\bf 12} \\
			\hline
			{\bf Total} & {\bf 9} & {\bf 15} & {\bf 15}  & {\bf 39} \\
			\hline
			\end{tabular}
		\end{center}
		\end{scriptsize}	\pause
		\item problem with additivity in colum 1 and 3. \pause
		\item another solution: \pause
		\begin{scriptsize}
		\begin{center}
			\begin{tabular}{|r|lll|l|}
			\hline
			{\bf H} & {\bf A} & {\bf B} & {\bf C} & {\bf Total} \\
			\hline
			{\bf I} 	& 3 & 6 & 3 & {\bf 15} \\
			{\bf II} 	& 0 & 6 & 6 & {\bf 15}\\
			{\bf III} & 3 & 3 & 3 & {\bf 12} \\
			\hline
			{\bf Total} & {\bf 12} & {\bf 15} & {\bf 15}  & {\bf 39} \\
			\hline
			\end{tabular}
		\end{center}
		\end{scriptsize}	\pause
		\item additivity in colum 1,3 and 4 and rows 1-4 stimmt is violated.\pause
		\item {\bf Attention:} this causes problems when the same cell is rounded different in linked tables.
	\end{itemize}
\end{frame}


\begin{frame}\frametitle{Controlled rounding}
	\begin{itemize}
		\item {\bf Idea:} each cell value is rounded on a specific manner, so that additivity of tables are not violated.\pause
		% \item {\bf Vielfache} der Basis werden (grunds?tzlich) nicht ver?ndert. \pause
		% \item {\bf Important:} unterschiedliche Gewichtungsschemata sind m?glich, jedoch soll keine Tendenz zum Auf- oder Abrunden durch das Gewichtungsschema implizit gegeben werden. \pause
		\item	{\bf Advantage:} tables stays (almost) additive. \pause
		\item	{\bf Disadvantage:} complex problem which is often practically unsolvable.
	\end{itemize}
\end{frame}

%\begin{frame}\frametitle{kontrolliertes Runden - Mathematisches Modell}
%	\begin{itemize}
%		\item {\bf Modell:}
%		\begin{eqnarray*}
%			min \sum_{i=1}^n c_i \cdot x_i \\
%			\sum_{i=1}^n m_{ji} (aRD_i + r_i \cdot x_i) &=& b_j \ldots j \in J \\
%			x_i &=& 0 \ldots \mbox{if} ~ aRD_i < lb_i ~ \mbox{or} ~ upl_i > aRU_i\\
%			x_i &=& 1 \ldots \mbox{if} ~ aRU_i > ub_i ~ \mbox{or} ~ lpl_i < aRD_i\\
%			x_i &\in& \{0,1\} \ldots \mbox{sonst}
%		\end{eqnarray*}
%		\item mit:
%		\begin{center}
%			\begin{tabular}{ll}
%				$RD_i \ldots \mbox{abgerundeter Wert f?r}~ a_i$ & $RU_i \ldots \mbox{aufgerundeter Wert f?r}~ a_i$ \\
%				$r_i = RU_i - RD_i$ & $c_i$ = costs\\
%				$upl_i \ldots \mbox{oberes Protection f?r}~ a_i$ & $lpl_i \ldots \mbox{untere Protection f?r}~ a_i$ \\
%				$ub_i \ldots \mbox{obere Grenze f?r}~ a_i$ & $lb_i \ldots \mbox{untere Grenze f?r}~ a_i$
%			\end{tabular}
%		\end{center}
%	\end{itemize}
%\end{frame}

\begin{frame}\frametitle{Controlled rounding - example}
	\begin{itemize}
		\item {\bf original table:}\\
		\begin{scriptsize}
		\begin{center}
			\begin{tabular}{|r|lll|l|}
			\hline
			{\bf H} & {\bf A} & {\bf B} & {\bf C} & {\bf Total} \\
			\hline
			{\bf I} 	& 4 & 6 & 3 & {\bf 13} \\
			{\bf II} 	& 2 & 5 & 7 & {\bf 14}\\
			{\bf III} & 4 & 5 & 3 & {\bf 12} \\
			\hline
			{\bf Total} & {\bf 10} & {\bf 16} & {\bf 13}  & {\bf 39} \\
			\hline
			\end{tabular}
		\end{center}
		\end{scriptsize} \pause
		\item Tabelle {\bf after controlled rounding:} \\
		\begin{scriptsize}
		\begin{center}
			\begin{tabular}{|r|lll|l|}
			\hline
			{\bf H} & {\bf A} & {\bf B} & {\bf C} & {\bf Total} \\
			\hline
			{\bf I} 	& 3 & 6 & 3 & {\bf 12} \\
			{\bf II} 	& 3 & 3 & 9 & {\bf 15}\\
			{\bf III} & 3 & 6 & 3 & {\bf 12} \\
			\hline
			{\bf Total} & {\bf 9} & {\bf 15} & {\bf 15}  & {\bf 39} \\
			\hline
			\end{tabular}
		\end{center}
		\end{scriptsize}	\pause
		\item All marginal totals are valid, the table is additive.
	\end{itemize}
\end{frame}

\subsection{Controlled tabular adjustment}
\begin{frame}\frametitle{Controlled tabular adjustment - CTA}
	\begin{itemize}
		\item {\bf Idea:}
		\begin{itemize}
			\item 1) each primary suppressed cell is replaced by an (large enough) interval.
			\item 2) All cells are adjusted in a way that the tables stays additive. \pause
		\end{itemize}
		\item {\bf Advantage:} no suppressions! Adustments for non-primary protected cells are often minor.\pause
		\item {\bf Additional advantage:} optimal algorithms exists. \pause
		\item {\bf Disadvantage:} optimal algorithms are only feasable in computational time for small tables. Again we need non-optimal heuristics which do not guarantee a solution of the problem.
	\end{itemize}
\end{frame}

% \begin{frame}\frametitle{Zellanpassung (CTA) -  Beispiel}
% 	\begin{itemize}
% 		\item {\bf urspr?ngliche Tabelle:} \pause
% 		\begin{scriptsize}
% 		\begin{center}
% 			\begin{tabular}{|r|lll|l|}
% 			\hline
% 			{\bf H} & {\bf A} & {\bf B} & {\bf C} & {\bf Total} \\ \hline
% 			{\bf I} 	& 74 & \cbw{17 [0:37]} & 85 & {\bf 176} \\
% 			{\bf II} 	& 71 & 51 & 30 & {\bf 152}\\
% 			{\bf III} & \cbw{1[0,21]} & \cbw{9[0,29]} & 36 & {\bf 46} \\ \hline
% 			{\bf Total} & {\bf 146} & {\bf 77} & {\bf 151}  & {\bf 374} \\ \hline
% 			\end{tabular}
% 		\end{center}
% 		\end{scriptsize} \pause
% 		\item {\bf Fixieren der Werte} f?r die sensitiven Zellen \pause
%
% 		\begin{scriptsize}
% 		\begin{center}
% 			\begin{tabular}{|r|lll|l|}
% 			\hline
% 			{\bf H} & {\bf A} & {\bf B} & {\bf C} & {\bf Total} \\ \hline
% 			{\bf I}   & \w{75*} & \w{0*}  & \w{85} & \wb{160*} \\
% 			{\bf II}  & \w{71}  & \w{51}  & \w{30} & \wb{152} \\
% 			{\bf III} & \w{0*}  & \w{29*} & \w{36} & \wb{65*} \\ \hline
% 			{\bf Total} & \wb{146} & \wb{80*} & \wb{151} & \wb{377*} \\ \hline
% 			\end{tabular}
% 		\end{center}
% 		\end{scriptsize}
% 		\end{itemize}
% \end{frame}

% \begin{frame}\frametitle{Zellanpassung (CTA) -  Beispiel}
% 	\begin{itemize}
% 		\item {\bf urspr?ngliche Tabelle:}
% 		\begin{scriptsize}
% 		\begin{center}
% 			\begin{tabular}{|r|lll|l|}
% 			\hline
% 			{\bf H} & {\bf A} & {\bf B} & {\bf C} & {\bf Total} \\ \hline
% 			{\bf I} 	& 74 & \cbw{17 [0:37]} & 85 & {\bf 176} \\
% 			{\bf II} 	& 71 & 51 & 30 & {\bf 152}\\
% 			{\bf III}   & \cbw{1[0,21]} & \cbw{9[0,29]} & 36 & {\bf 46} \\ \hline
% 			{\bf Total} & {\bf 146} & {\bf 77} & {\bf 151}  & {\bf 374} \\ \hline
% 			\end{tabular}
% 		\end{center}
% 		\end{scriptsize}
% 		\item {\bf Fixieren der Werte} f?r die sensitiven Zellen
%
% 		\begin{scriptsize}
% 		\begin{center}
% 			\begin{tabular}{|r|lll|l|}
% 			\hline
% 			{\bf H} & {\bf A} & {\bf B} & {\bf C} & {\bf Total} \\ \hline
% 			{\bf I}   & \w{75*} & \cbw{0*}  & \w{85} & \wb{160*} \\
% 			{\bf II}  & \w{71}  & \w{51}  & \w{30} & \wb{152} \\
% 			{\bf III} & \w{0*}  & \w{29*} & \w{36} & \wb{65*} \\ \hline
% 			{\bf Total} & \wb{146} & \wb{80*} & \wb{151} & \wb{377*} \\ \hline
% 			\end{tabular}
% 		\end{center}
% 		\end{scriptsize}
% 		\end{itemize}
% \end{frame}


% \begin{frame}\frametitle{Zellanpassung (CTA) -  Beispiel}
% 	\begin{itemize}
% 		\item {\bf urspr?ngliche Tabelle:}
% 		\begin{scriptsize}
% 		\begin{center}
% 			\begin{tabular}{|r|lll|l|}
% 			\hline
% 			{\bf H}   & {\bf A} & {\bf B} & {\bf C} & {\bf Total} \\ \hline
% 			{\bf I}   & 74 & \cbw{17 [0:37]} & 85 & {\bf 176} \\
% 			{\bf II}  & 71 & 51 & 30 & {\bf 152}\\
% 			{\bf III} & \cbw{1[0,21]} & \cbw{9[0,29]} & 36 & {\bf 46} \\ \hline
% 			{\bf Total} & {\bf 146} & {\bf 77} & {\bf 151}  & {\bf 374} \\ \hline
% 			\end{tabular}
% 		\end{center}
% 		\end{scriptsize}
% 		\item {\bf Fixieren der Werte} f?r die sensitiven Zellen
%
% 		\begin{scriptsize}
% 		\begin{center}
% 			\begin{tabular}{|r|lll|l|}
% 			\hline
% 			{\bf H} & {\bf A} & {\bf B} & {\bf C} & {\bf Total} \\ \hline
% 			{\bf I}   & \w{75*} & \cbw{0*} & \w{85} & \wb{160*} \\
% 			{\bf II}  & \w{71} & \w{51} & \w{30}  & \wb{152} \\
% 			{\bf III} & \w{0*} & \cbw{29*} & \w{36} & \wb{65*} \\ \hline
% 			{\bf Total} & \wb{146} & \wb{80*} & \wb{151}  & \wb{377*} \\ \hline
% 			\end{tabular}
% 		\end{center}
% 		\end{scriptsize}
% 		\end{itemize}
% \end{frame}

\begin{frame}\frametitle{CTA -  Example}
	\begin{itemize}
		\item {\bf original table:}
		\begin{scriptsize}
		\begin{center}
			\begin{tabular}{|r|lll|l|}
			\hline
			{\bf H} & {\bf A} & {\bf B} & {\bf C} & {\bf Total} \\ \hline
			{\bf I} 	& 74 & \cbw{17 [0:37]} & 85 & {\bf 176} \\
			{\bf II} 	& 71 & 51 & 30 & {\bf 152}\\
			{\bf III} & \cbw{1[0,21]} & \cbw{9[0,29]} & 36 & {\bf 46} \\ \hline
			{\bf Total} & {\bf 146} & {\bf 77} & {\bf 151}  & {\bf 374} \\ \hline
			\end{tabular}
		\end{center}
		\end{scriptsize} \pause
		\item {\bf fix the values} for the sensitive cells
		\pause

		\begin{scriptsize}
		\begin{center}
			\begin{tabular}{|r|lll|l|}
			\hline
			{\bf H} & {\bf A} & {\bf B} & {\bf C} & {\bf Total} \\ \hline
			{\bf I}   & \w{75*} & \cbw{0*}  & \w{85} & \wb{160*} \\
			{\bf II}  & \w{71}  & \w{51}    & \w{30} & \wb{152}\\
			{\bf III} & \cbw{0*}  & \cbw{29*} & \w{36} & \wb{65*} \\ \hline
			{\bf Total} & \wb{146} & \wb{80*} & \wb{151}  & \wb{377*} \\ \hline
			\end{tabular}
		\end{center}
		\end{scriptsize}
		\end{itemize}
\end{frame}


\begin{frame}\frametitle{CTA - example}
	\begin{itemize}
		\item {\bf original table:}
		\begin{scriptsize}
		\begin{center}
			\begin{tabular}{|r|lll|l|}
			\hline
			{\bf H} & {\bf A} & {\bf B} & {\bf C} & {\bf Total} \\ \hline
			{\bf I} 	& 74 & \cbw{17 [0:37]} & 85 & {\bf 176} \\
			{\bf II} 	& 71 & 51 & 30 & {\bf 152}\\
			{\bf III} & \cbw{1[0,21]} & \cbw{9[0,29]} & 36 & {\bf 46} \\ \hline
			{\bf Total} & {\bf 146} & {\bf 77} & {\bf 151}  & {\bf 374} \\ \hline
			\end{tabular}
		\end{center}
		\end{scriptsize}
		\item {\bf Adjustment} of non-sensitive cells

		\begin{scriptsize}
		\begin{center}
			\begin{tabular}{|r|lll|l|}
			\hline
			{\bf H} & {\bf A} & {\bf B} & {\bf C} & {\bf Total} \\ 	\hline
			{\bf I}   & \red{75*}  & \cbw{0*}  & \w{85} & \wb{160*} \\
			{\bf II}  & \w{71}   & \w{51}    & \w{30} & \wb{152} \\
			{\bf III} & \cbw{0*} & \cbw{29*} & \w{36} & \wb{65*} \\ \hline
			{\bf Total} & \wb{146} & \wb{80*} & \wb{151} & \wb{377*} \\ \hline
			\end{tabular}
		\end{center}
		\end{scriptsize}
		\end{itemize}
\end{frame}

\begin{frame}\frametitle{CTA - example}
	\begin{itemize}
		\item {\bf original table:}
		\begin{scriptsize}
		\begin{center}
			\begin{tabular}{|r|lll|l|}
			\hline
			{\bf H}     & {\bf A} & {\bf B} & {\bf C} & {\bf Total} \\ \hline
			{\bf I} 	& 74 & \cbw{17 [0:37]} & 85 & {\bf 176} \\
			{\bf II} 	& 71 & 51 & 30 & {\bf 152}\\
			{\bf III}   & \cbw{1[0,21]} & \cbw{9[0,29]} & 36 & {\bf 46} \\ \hline
			{\bf Total} & {\bf 146} & {\bf 77} & {\bf 151}  & {\bf 374} \\ \hline
			\end{tabular}
		\end{center}
		\end{scriptsize}
		\item {\bf Adjustment} of the non-sensitive cells

		\begin{scriptsize}
		\begin{center}
			\begin{tabular}{|r|lll|l|}
			\hline
			{\bf H} & {\bf A} & {\bf B} & {\bf C} & {\bf Total} \\ \hline
			{\bf I}   & \red{75*}  & \cbw{0*}  & 85 & \wb{160*} \\
			{\bf II}  & \w{71}   & \w{51}    & \w{30} & \wb{152}\\
			{\bf III} & \cbw{0*} & \cbw{29*} & \w{36} & \wb{65*} \\ \hline
			{\bf Total} & \wb{146} & \wb{80*} & \wb{151}  & \wb{377*} \\ \hline
			\end{tabular}
		\end{center}
		\end{scriptsize}
		\end{itemize}
\end{frame}

\begin{frame}\frametitle{CTA -  example}
	\begin{itemize}
		\item {\bf original table:}
		\begin{scriptsize}
		\begin{center}
			\begin{tabular}{|r|lll|l|}
			\hline
			{\bf H} & {\bf A} & {\bf B} & {\bf C} & {\bf Total} \\ \hline
			{\bf I} 	& 74 & \cbw{17 [0:37]} & 85 & {\bf 176} \\
			{\bf II} 	& 71 & 51 & 30 & {\bf 152}\\
			{\bf III} & \cbw{1[0,21]} & \cbw{9[0,29]} & 36 & {\bf 46} \\ \hline
			{\bf Total} & {\bf 146} & {\bf 77} & {\bf 151}  & {\bf 374} \\ \hline
			\end{tabular}
		\end{center}
		\end{scriptsize}
		\item {\bf Adjustment} of non-sensitive cells

		\begin{scriptsize}
		\begin{center}
			\begin{tabular}{|r|lll|l|}
			\hline
			{\bf H} & {\bf A} & {\bf B} & {\bf C} & {\bf Total} \\ \hline
			{\bf I} 	& \red{75*} & \cbw{0*} & 85 & \redb{160*} \\
			{\bf II} 	& \w{71} & \w{51} & \w{30} & \wb{152}\\
			{\bf III} & \cbw{0*} & \cbw{29*} & \w{36} & \wb{65*} \\ \hline
			{\bf Total} & \wb{146} & \wb{80*} & \wb{151}  & \wb{377*} \\ \hline
			\end{tabular}
		\end{center}
		\end{scriptsize}
		\end{itemize}
\end{frame}

\begin{frame}\frametitle{CTA - example}
	\begin{itemize}
		\item {\bf original table:}
		\begin{scriptsize}
		\begin{center}
			\begin{tabular}{|r|lll|l|}
			\hline
			{\bf H} & {\bf A} & {\bf B} & {\bf C} & {\bf Total} \\ \hline
			{\bf I} 	& 74 & \cbw{17 [0:37]} & 85 & {\bf 176} \\
			{\bf II} 	& 71 & 51 & 30 & {\bf 152}\\
			{\bf III} & \cbw{1[0,21]} & \cbw{9[0,29]} & 36 & {\bf 46} \\ \hline
			{\bf Total} & {\bf 146} & {\bf 77} & {\bf 151}  & {\bf 374} \\ \hline
			\end{tabular}
		\end{center}
		\end{scriptsize}
		\item {\bf Adjustment} of non-sensitive cells

		\begin{scriptsize}
		\begin{center}
			\begin{tabular}{|r|lll|l|}
			\hline
			{\bf H} & {\bf A} & {\bf B} & {\bf C} & {\bf Total} \\ \hline
			{\bf I} 	& \red{75*} & \cbw{0*} & 85 & \redb{160*} \\
			{\bf II} 	& 71 & \w{51} & \w{30} & \wb{152} \\
			{\bf III} & \cbw{0*} & \cbw{29*} & \w{36} & \wb{65*} \\ \hline
			{\bf Total} & \wb{146} & \wb{80*} & \wb{151}  & \wb{377*} \\ \hline
			\end{tabular}
		\end{center}
		\end{scriptsize}
		\end{itemize}
\end{frame}

\begin{frame}\frametitle{CTA - example}
	\begin{itemize}
		\item {\bf original table:}
		\begin{scriptsize}
		\begin{center}
			\begin{tabular}{|r|lll|l|}
			\hline
			{\bf H} & {\bf A} & {\bf B} & {\bf C} & {\bf Total} \\ \hline
			{\bf I} 	& 74 & \cbw{17 [0:37]} & 85 & {\bf 176} \\
			{\bf II} 	& 71 & 51 & 30 & {\bf 152}\\
			{\bf III} & \cbw{1[0,21]} & \cbw{9[0,29]} & 36 & {\bf 46} \\ \hline
			{\bf Total} & {\bf 146} & {\bf 77} & {\bf 151}  & {\bf 374} \\ \hline
			\end{tabular}
		\end{center}
		\end{scriptsize}
		\item {\bf Adjustment} of non-sensitive cells

		\begin{scriptsize}
		\begin{center}
			\begin{tabular}{|r|lll|l|}
			\hline
			{\bf H} & {\bf A} & {\bf B} & {\bf C} & {\bf Total} \\ \hline
			{\bf I} 	& \red{75*} & \cbw{0*} & 85 & \redb{160*} \\
			{\bf II} 	& 71 & 51 & \w{30} & \wb{152}\\
			{\bf III} & \cbw{0*} & \cbw{29*} & \w{36} & \wb{65*} \\ \hline
			{\bf Total} & \wb{146} & \wb{80*} & \wb{151}  & \wb{377*} \\ \hline
			\end{tabular}
		\end{center}
		\end{scriptsize}
		\end{itemize}
\end{frame}


\begin{frame}\frametitle{CTA - example}
	\begin{itemize}
		\item {\bf original table:}
		\begin{scriptsize}
		\begin{center}
			\begin{tabular}{|r|lll|l|}
			\hline
			{\bf H} & {\bf A} & {\bf B} & {\bf C} & {\bf Total} \\ \hline
			{\bf I} 	& 74 & \cbw{17 [0:37]} & 85 & {\bf 176} \\
			{\bf II} 	& 71 & 51 & 30 & {\bf 152}\\
			{\bf III} & \cbw{1[0,21]} & \cbw{9[0,29]} & 36 & {\bf 46} \\ \hline
			{\bf Total} & {\bf 146} & {\bf 77} & {\bf 151}  & {\bf 374} \\ \hline
			\end{tabular}
		\end{center}
		\end{scriptsize}
		\item {\bf Adjustment} of non-sensitive cells

		\begin{scriptsize}
		\begin{center}
			\begin{tabular}{|r|lll|l|}
			\hline
			{\bf H} & {\bf A} & {\bf B} & {\bf C} & {\bf Total} \\ \hline
			{\bf I} 	& \red{75*} & \cbw{0*} & 85 & \redb{160*} \\
			{\bf II} 	& 71 & 51 & 30 & \wb{152}\\
			{\bf III}   & \cbw{0*} & \cbw{29*} & \w{36} & \wb{65*} \\ \hline
			{\bf Total} & \wb{146} & \wb{80*} & \wb{151}  & \wb{377*} \\
			\hline
			\end{tabular}
		\end{center}
		\end{scriptsize}
		\end{itemize}
\end{frame}

\begin{frame}\frametitle{CTA - example}
	\begin{itemize}
		\item {\bf original table:}
		\begin{scriptsize}
		\begin{center}
			\begin{tabular}{|r|lll|l|}
			\hline
			{\bf H} & {\bf A} & {\bf B} & {\bf C} & {\bf Total} \\ \hline
			{\bf I} 	& 74 & \cbw{17 [0:37]} & 85 & {\bf 176} \\
			{\bf II} 	& 71 & 51 & 30 & {\bf 152}\\
			{\bf III} & \cbw{1[0,21]} & \cbw{9[0,29]} & 36 & {\bf 46} \\ \hline
			{\bf Total} & {\bf 146} & {\bf 77} & {\bf 151}  & {\bf 374} \\ \hline
			\end{tabular}
		\end{center}
		\end{scriptsize}
		\item {\bf Adjustment} of non-sensitive cells

		\begin{scriptsize}
		\begin{center}
			\begin{tabular}{|r|lll|l|}
			\hline
			{\bf H} & {\bf A} & {\bf B} & {\bf C} & {\bf Total} \\ \hline
			{\bf I} 	& \red{75*} & \cbw{0*} & 85 & \redb{160*} \\
			{\bf II} 	& 71 & 51 & 30 & {\bf 152}\\
			{\bf III} & \cbw{0*} & \cbw{29*} & \w{36} & \wb{65*} \\ \hline
			{\bf Total} & \wb{146} & \wb{80*} & \wb{151}  & \wb{377*} \\ \hline
			\end{tabular}
		\end{center}
		\end{scriptsize}
		\end{itemize}
\end{frame}

\begin{frame}\frametitle{CTA - example}
	\begin{itemize}
		\item {\bf original table:}
		\begin{scriptsize}
		\begin{center}
			\begin{tabular}{|r|lll|l|}
			\hline
			{\bf H} & {\bf A} & {\bf B} & {\bf C} & {\bf Total} \\ \hline
			{\bf I} 	& 74 & \cbw{17 [0:37]} & 85 & {\bf 176} \\
			{\bf II} 	& 71 & 51 & 30 & {\bf 152}\\
			{\bf III} & \cbw{1[0,21]} & \cbw{9[0,29]} & 36 & {\bf 46} \\ \hline
			{\bf Total} & {\bf 146} & {\bf 77} & {\bf 151}  & {\bf 374} \\ \hline
			\end{tabular}
		\end{center}
		\end{scriptsize}
		\item {\bf Adjustment} of non-sensitive cells

		\begin{scriptsize}
		\begin{center}
			\begin{tabular}{|r|lll|l|}
			\hline
			{\bf H} & {\bf A} & {\bf B} & {\bf C} & {\bf Total} \\ \hline
			{\bf I} 	& \red{75*} & \cbw{0*} & 85 & \redb{160*} \\
			{\bf II} 	& 71 & 51 & 30 & {\bf 152}\\
			{\bf III} & \cbw{0*} & \cbw{29*} & 36 & \wb{65*} \\ \hline
			{\bf Total} & \wb{146} & \wb{80*} & \wb{151}  & \wb{377*} \\ \hline
			\end{tabular}
		\end{center}
		\end{scriptsize}
		\end{itemize}
\end{frame}

\begin{frame}\frametitle{CTA - example}
	\begin{itemize}
		\item {\bf original table:}
		\begin{scriptsize}
		\begin{center}
			\begin{tabular}{|r|lll|l|}
			\hline
			{\bf H} & {\bf A} & {\bf B} & {\bf C} & {\bf Total} \\ \hline
			{\bf I} 	& 74 & \cbw{17 [0:37]} & 85 & {\bf 176} \\
			{\bf II} 	& 71 & 51 & 30 & {\bf 152}\\
			{\bf III} & \cbw{1[0,21]} & \cbw{9[0,29]} & 36 & {\bf 46} \\ \hline
			{\bf Total} & {\bf 146} & {\bf 77} & {\bf 151}  & {\bf 374} \\ \hline
			\end{tabular}
		\end{center}
		\end{scriptsize}
		\item {\bf Adjustment} of non-sensitive cells

		\begin{scriptsize}
		\begin{center}
			\begin{tabular}{|r|lll|l|}
			\hline
			{\bf H} & {\bf A} & {\bf B} & {\bf C} & {\bf Total} \\ \hline
			{\bf I} 	& \red{75*} & \cbw{0*} & 85 & \redb{160*} \\
			{\bf II} 	& 71 & 51 & 30 & {\bf 152}\\
			{\bf III} & \cbw{0*} & \cbw{29*} & 36 & \redb{65*} \\ \hline
			{\bf Total} & \wb{146} & \wb{80*} & \wb{151} & \wb{377*} \\ \hline
			\end{tabular}
		\end{center}
		\end{scriptsize}
		\end{itemize}
\end{frame}

\begin{frame}\frametitle{CTA - example}
	\begin{itemize}
		\item {\bf original table:}
		\begin{scriptsize}
		\begin{center}
			\begin{tabular}{|r|lll|l|}
			\hline
			{\bf H} & {\bf A} & {\bf B} & {\bf C} & {\bf Total} \\ \hline
			{\bf I} 	& 74 & \cbw{17 [0:37]} & 85 & {\bf 176} \\
			{\bf II} 	& 71 & 51 & 30 & {\bf 152}\\
			{\bf III} & \cbw{1[0,21]} & \cbw{9[0,29]} & 36 & {\bf 46} \\ \hline
			{\bf Total} & {\bf 146} & {\bf 77} & {\bf 151}  & {\bf 374} \\ \hline
			\end{tabular}
		\end{center}
		\end{scriptsize}
		\item {\bf Adjustment} of non-sensitive cells

		\begin{scriptsize}
		\begin{center}
			\begin{tabular}{|r|lll|l|}
			\hline
			{\bf H} & {\bf A} & {\bf B} & {\bf C} & {\bf Total} \\ \hline
			{\bf I} 	& \red{75*} & \cbw{0*} & 85 & \redb{160*} \\
			{\bf II} 	& 71 & 51 & 30 & {\bf 152}\\
			{\bf III} & \cbw{0*} & \cbw{29*} & 36 & \redb{65*} \\ \hline
			{\bf Total} & {\bf 146} & \wb{80*} & \wb{151} & \wb{377*} \\ \hline
			\end{tabular}
		\end{center}
		\end{scriptsize}
		\end{itemize}
\end{frame}

\begin{frame}\frametitle{CTA - example}
	\begin{itemize}
		\item {\bf original table:}
		\begin{scriptsize}
		\begin{center}
			\begin{tabular}{|r|lll|l|}
			\hline
			{\bf H} & {\bf A} & {\bf B} & {\bf C} & {\bf Total} \\ \hline
			{\bf I} 	& 74 & \cbw{17 [0:37]} & 85 & {\bf 176} \\
			{\bf II} 	& 71 & 51 & 30 & {\bf 152}\\
			{\bf III} & \cbw{1[0,21]} & \cbw{9[0,29]} & 36 & {\bf 46} \\ \hline
			{\bf Total} & {\bf 146} & {\bf 77} & {\bf 151}  & {\bf 374} \\ \hline
			\end{tabular}
		\end{center}
		\end{scriptsize}
		\item {\bf Adjustment} of non-sensitive cells

		\begin{scriptsize}
		\begin{center}
			\begin{tabular}{|r|lll|l|}
			\hline
			{\bf H} & {\bf A} & {\bf B} & {\bf C} & {\bf Total} \\ \hline
			{\bf I} 	& \red{75*} & \cbw{0*} & 85 & \redb{160*} \\
			{\bf II} 	& 71 & 51 & 30 & {\bf 152}\\
			{\bf III} & \cbw{0*} & \cbw{29*} & 36 & \redb{65*} \\ \hline
			{\bf Total} & {\bf 146} & \redb{80*} & \wb{151}  & \wb{377*} \\ \hline
			\end{tabular}
		\end{center}
		\end{scriptsize}
		\end{itemize}
\end{frame}

\begin{frame}\frametitle{CTA - example}
	\begin{itemize}
		\item {\bf original table:}
		\begin{scriptsize}
		\begin{center}
			\begin{tabular}{|r|lll|l|}
			\hline
			{\bf H} & {\bf A} & {\bf B} & {\bf C} & {\bf Total} \\ \hline
			{\bf I} 	& 74 & \cbw{17 [0:37]} & 85 & {\bf 176} \\
			{\bf II} 	& 71 & 51 & 30 & {\bf 152}\\
			{\bf III} & \cbw{1[0,21]} & \cbw{9[0,29]} & 36 & {\bf 46} \\ \hline
			{\bf Total} & {\bf 146} & {\bf 77} & {\bf 151}  & {\bf 374} \\ \hline
			\end{tabular}
		\end{center}
		\end{scriptsize}
		\item {\bf Adjustment} of non-sensitive cells

		\begin{scriptsize}
		\begin{center}
			\begin{tabular}{|r|lll|l|}
			\hline
			{\bf H} & {\bf A} & {\bf B} & {\bf C} & {\bf Total} \\ \hline
			{\bf I}   & \red{75*} & \cbw{0*} & 85 & \redb{160*} \\
			{\bf II}  & 71 & 51 & 30 & {\bf 152}\\
			{\bf III} & \cbw{0*} & \cbw{29*} & 36 & \redb{65*} \\ \hline
			{\bf Total} & {\bf 146} & \redb{80*} & {\bf 151}  & \wb{377*} \\ \hline
			\end{tabular}
		\end{center}
		\end{scriptsize}
		\end{itemize}
\end{frame}

\begin{frame}\frametitle{CTA - example}
	\begin{itemize}
		\item {\bf original table:}
		\begin{scriptsize}
		\begin{center}
			\begin{tabular}{|r|lll|l|}
			\hline
			{\bf H} & {\bf A} & {\bf B} & {\bf C} & {\bf Total} \\ \hline
			{\bf I} 	& 74 & \cbw{17 [0:37]} & 85 & {\bf 176} \\
			{\bf II} 	& 71 & 51 & 30 & {\bf 152}\\
			{\bf III} & \cbw{1[0,21]} & \cbw{9[0,29]} & 36 & {\bf 46} \\ \hline
			{\bf Total} & {\bf 146} & {\bf 77} & {\bf 151}  & {\bf 374} \\ \hline
			\end{tabular}
		\end{center}
		\end{scriptsize}
		\item {\bf Adjustment} of non-sensitive cells

		\begin{scriptsize}
		\begin{center}
			\begin{tabular}{|r|lll|l|}
			\hline
			{\bf H} & {\bf A} & {\bf B} & {\bf C} & {\bf Total} \\ \hline
			{\bf I} 	& \red{75*} & \cbw{0*} & 85 & \redb{160*} \\
			{\bf II} 	& 71 & 51 & 30 & {\bf 152}\\
			{\bf III} & \cbw{0*} & \cbw{29*} & 36 & \redb{65*} \\ \hline
			{\bf Total} & {\bf 146} & \redb{80*} & {\bf 151}  & \redb{377*} \\ \hline
			\end{tabular}
		\end{center}
		\end{scriptsize}
		\end{itemize}
\end{frame}


\begin{frame}\frametitle{CTA - example}
	\begin{itemize}
		\item {\bf original table:}\\
		\begin{scriptsize}
		\begin{center}
			\begin{tabular}{|r|lll|l|}
			\hline
			{\bf H} & {\bf A} & {\bf B} & {\bf C} & {\bf Total} \\ \hline
			{\bf I} 	& 74 & \cbw{17 [0:37]} & 85 & {\bf 176} \\
			{\bf II} 	& 71 & 51 & 30 & {\bf 152}\\
			{\bf III} & \cbw{1[0,21]} & \cbw{9[0,29]} & 36 & {\bf 46} \\ \hline
			{\bf Total} & {\bf 146} & {\bf 77} & {\bf 151}  & {\bf 374} \\ \hline
			\end{tabular}
		\end{center}
		\end{scriptsize}
		\item Table {\bf after CTA:} \\
		\begin{scriptsize}
		\begin{center}
			\begin{tabular}{|r|lll|l|}
			\hline
			{\bf H} & {\bf A} & {\bf B} & {\bf C} & {\bf Total} \\ \hline
			{\bf I} 	& \cbw{75*} & \cbw{0*} & 85 & \cbwb{160*} \\
			{\bf II} 	& 71 & 51 & 30 & {\bf 152}\\
			{\bf III} & \cbw{0*} & \cbw{29*} & 36 & \cbwb{65*} \\ \hline
			{\bf Total} & {\bf 146} & \cbwb{80*} & {\bf 151}  & \cbwb{377*} \\ \hline
			\end{tabular}
		\end{center}
		\end{scriptsize}
		\item {\bf Implementation} is based on linear optimization (complex formulas).
	\end{itemize}
\end{frame}

\begin{frame}\frametitle{Cell Adjustment by ABS}
	\begin{itemize}
		\item method from {\bf ABS} (Australian Bureau of Statistics) is a special case of {\bf table perturbation} \pause
		\item {\bf Idea:} konsistent and random perturbation of cells in a table based on
		\begin{itemize}
		    \item Record-keys
		    \item Cell-keys
		    \item LookUp-tables \pause
		\end{itemize}
		\item {\bf Advantage:} konsistent tables results \pause
		\item {\bf Disadvantage:} tables are non-additive anymore \pause
	\end{itemize}
\end{frame}

%%%%%%%%%%%%
%%%%%%%%%%%%%%%%%%%%%%%%%%%%%%%%%%%%%%%%%%%%%%%%%%%%

%%%%%%%%%%%%%%%%%%%%%%%%%%%%%%%%%%%%%%%%%%%%%%%%%%%%
%%%%% Methoden zum Schutz von sensiblen Zellen %%%%%
\section{Software}
\begin{frame}\frametitle{Software for tabular protection}
	\begin{itemize}
	  \item {\tt $\tau$-Argus:}
	  \begin{itemize}
	    \item $\tau$-Argus resulted from research funds of the EU \pause
	  	\item Code ist partly opened after $\approx$ 15 years of the development of the code
	  	\pause
	  	\item Well established software, but mostly it works only for Windows 32-bit version and it includes a lot of workarounds
	  	(cause the orginal code nobody touched so far) 
	  	\pause
	  \end{itemize}
	  \item {\tt sdcTable:}
	  \begin{itemize}
	    \item open-source in {\tt R} \pause
	  	\item easySdcTable package as front-end (we will hear a presentation, I do not know if its implemented well) \pause
	  	\item adaptable, flexible and cool, but still in heavy development \pause
	  \end{itemize}
	\end{itemize}
\end{frame}

% Was kann das Paket, was nicht, Einschr?nkungen\ldots
%\begin{frame}\frametitle{sdcTable: ein Kurz?berblick}
%	\begin{itemize}
%	  \item
%	\end{itemize}
%\end{frame}

% Kurzes Beispiel (Code)
%\begin{frame}\frametitle{sdcTable: ein einfaches Beispiel}
%	\begin{itemize}
%	  \item
%	\end{itemize}
%\end{frame}

%%%%%%%%%%%%
%%%%%%%%%%%%%%%%%%%%%%%%%%%%%%%%%%%%%%%%%%%%%%%%%%%%

\begin{frame}[fragile]\frametitle{Minimal example}

For cell suppression:

\begin{itemize}
\item Problem must be specified in a structured manner (the most difficult task) (\textbf{makeProblem()})
\item Primary suppression according to a primary suppression rule (\textbf{primarySuppression()})
\item Apply one of the tabular protection rules (\textbf{protectTable()})
\end{itemize}

\end{frame}


\begin{frame}[fragile]\frametitle{Minimal example}

\begin{Schunk}
\begin{Sinput}
> library(sdcTable)
> data("microData1", package="sdcTable")
> # having a look at the data structure
> str(microData1)
\end{Sinput}
\begin{Soutput}
'data.frame':	100 obs. of  3 variables:
 $ region: chr  "C" "C" "A" "A" ...
 $ gender: chr  "male" "male" "male" "male" ...
 $ val   : num  9 11 10 11 18 10 6 9 5 12 ...
\end{Soutput}
\end{Schunk}

$\rightarrow$ two spanning variables ('region' and 'gender'), one
numeric ('val').

Specify hierarchical structure of 'region', levels 'A' to 'D' sum up to a total
\begin{Schunk}
\begin{Sinput}
> dim.region <- data.frame(
+  levels=c('@','@@','@@','@@','@@'),
+  codes=c('Total', 'A','B','C','D'),
+  stringsAsFactors=FALSE)
\end{Sinput}
\end{Schunk}
\end{frame}

\begin{frame}[fragile]\frametitle{Minimal example}

Specify structure of hierarchical variable 'gender'


\begin{Schunk}
\begin{Sinput}
> dim.gender <- hier_create(root = "Total", 
+                 nodes = c("male", "female"))
> hier_display(dim.gender) # see result in R
\end{Sinput}
\begin{Soutput}
Total

\end{Soutput}
\end{Schunk}

\end{frame}

\begin{frame}[fragile]\frametitle{Minimal example}

create a named list with each element being a data-frame
 containing information on one dimensional variable 
 
\begin{Schunk}
\begin{Sinput}
> dimList <- list(region = dim.region, gender = dim.gender)
> numVarInd <- 3
\end{Sinput}
\end{Schunk}

\end{frame}

\begin{frame}[fragile]\frametitle{Minimal example}

In this example, no variables holding counts, numeric values, weights or sampling

\begin{Schunk}
\begin{Sinput}
> p1 <- makeProblem(
+   data = microData1,
+   dimList = dimList,
+   numVarInd = "val" # third variable in `data`
+ )
> print(class(p1))
\end{Sinput}
\begin{Soutput}
[1] "sdcProblem"
attr(,"package")
[1] "sdcTable"
\end{Soutput}
\end{Schunk}
\end{frame}

\begin{frame}[fragile]\frametitle{Minimal example}

\begin{Schunk}
\begin{Sinput}
> p1 <- primarySuppression(
+   object = p1,
+   type = "freq",
+   maxN = 2
+ )
\end{Sinput}
\end{Schunk}

\end{frame}

\begin{frame}[fragile]\frametitle{Minimal example}

Problem is set up

\begin{Schunk}
\begin{Sinput}
> df1 <- sdcProb2df(p1, addDups = TRUE,
+   addNumVars = TRUE, dimCodes = "original")
> print(df1)
\end{Sinput}
\begin{Soutput}
    strID freq sdcStatus  val region gender
 1:  0000  100         s 1284  Total  Total
 2:  0001   45         s  482  Total female
 3:  0002   55         s  802  Total   male
 4:  0100   20         s  198      A  Total
 5:  0101    2         u   20      A female
 6:  0102   18         s  178      A   male
 7:  0200   33         s  344      B  Total
 8:  0201   19         s  204      B female
 9:  0202   14         s  140      B   male
10:  0300   22         s  224      C  Total
11:  0301   10         s  106      C female
12:  0302   12         s  118      C   male
13:  0400   25         s  518      D  Total
14:  0401   14         s  152      D female
15:  0402   11         s  366      D   male
\end{Soutput}
\end{Schunk}
\end{frame}

\begin{frame}[fragile]\frametitle{Minimal example}

We now can apply an algorithms (several can be chosen) to receive protected tables

\begin{Schunk}
\begin{Sinput}
> protectedData <- protectTable(p1, 
+                    method='HITAS')
\end{Sinput}
\end{Schunk}

\end{frame}

\begin{frame}[fragile]\frametitle{Minimal example}


\begin{Schunk}
\begin{Sinput}
> summary(protectedData)
\end{Sinput}
\begin{Soutput}
######################################################
### Summary of the result object of class 'safeObj' ###
######################################################
--> The input data have been protected using algorithm HITAS.
--> The algorithm ran for 1 second.
--> To protect 1 primary sensitive cells, 3 cells need to be additionally suppressed.
--> A total of 11 cells may be published.

###################################
### Structure of protected Data ###
###################################
Classes 'data.table' and 'data.frame':	15 obs. of  5 variables:
 $ region   : chr  "Total" "Total" "Total" "A" ...
 $ gender   : chr  "Total" "female" "male" "Total" ...
 $ Freq     : num  100 45 55 20 2 18 33 19 14 22 ...
 $ val      : num  1284 482 802 198 20 ...
 $ sdcStatus: chr  "s" "s" "s" "s" ...
 - attr(*, ".internal.selfref")=<externalptr> 
 - attr(*, "index")= int 
  ..- attr(*, "__sdcStatus")= int  1 2 3 4 7 8 9 10 13 14 ...
NULL
\end{Soutput}
\begin{Sinput}
> print(getInfo(protectedData, type='finalData'))
\end{Sinput}
\begin{Soutput}
    region gender Freq  val sdcStatus
 1:  Total  Total  100 1284         s
 2:  Total female   45  482         s
 3:  Total   male   55  802         s
 4:      A  Total   20  198         s
 5:      A female    2   20         u
 6:      A   male   18  178         x
 7:      B  Total   33  344         s
 8:      B female   19  204         s
 9:      B   male   14  140         s
10:      C  Total   22  224         s
11:      C female   10  106         x
12:      C   male   12  118         x
13:      D  Total   25  518         s
14:      D female   14  152         s
15:      D   male   11  366         s
\end{Soutput}
\end{Schunk}

\end{frame}

\section{Conclusion}
\begin{frame}\frametitle{Conclusio}
	\begin{itemize}
		\item Anonymization of tabular data is very {\bf complex}. \pause
		\item Methods: \pause
		\begin{itemize} 
			\item cell suppression
			\item rounding
			\item CTA 
			\item ABS 
			\pause
		\end{itemize}
		\item all methods have its advantages and disadvantages
	\end{itemize}
	
	Guide for sdcTable:
	https://cran.r-project.org/web/packages/sdcTable/vignettes/sdcTable.html
\end{frame}



\end{document}
